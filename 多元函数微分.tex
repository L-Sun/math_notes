\part{多元函数微分}
\section{多元函数的概念}
在坐标平面上,如果$P_0(x_0,y_0)$是一个点,$\delta > 0$,则称到点$P_0$点距离小于$\delta$的
点$P(x,y)$的全体为$P_0$的$\delta$邻域,记作$B(P_0,\delta)$,即
\begin{align*}
    B(P_0,\delta) & = \{P:|PP_0|<\delta\}                              \\
                  & = \{(x,y)\,|\, (x-x_0)^2 + (y-y_0)^2 < \delta^2 \}
\end{align*}

一个平面点集$E$中的点$P$称为\textcolor{red}{\textbf{\textsf{内点}}},是指存在一个邻域$B(P_0,\delta)$,
满足
\[ B(P_0,\delta) \subset E \]
每个点都是内殿的平面点集称为\textcolor{red}{\textbf{\textsf{开集}}}。如果一个平面点集包含了每个边界点,
则称为\textcolor{red}{\textbf{\textsf{闭集}}}。若平面点集中的任意两个点可以用点集中的一条连续取许连接,
则称该点集为\textcolor{red}{\textbf{\textsf{连通点集}}}。
连通的开集称为\textcolor{red}{\textbf{\textsf{开区域}}},常简称为\textcolor{red}{\textbf{\textsf{区域}}};
连通的闭集称为\textcolor{red}{\textbf{\textsf{闭区域}}},有时也简称区域。

从$n$维向量空间$\mathbb{R}^n$的非空子集$X$到实数域$\mathbb{R}$的映射$f:X\to \mathbb{R}$,称为$n$元函数。

\section{多元函数的极限}
按照向量表示法,多元函数的极限可以统一定义为
\begin{definition}
    设函数$f(x)$在$\bm{x}_0\in\mathbb{R}^n$点的某邻域内有定义,称$\bm{x}\to\bm{x}_0$时,
    $f(\bm{x})$的极限为$A$,记作
    \[ \lim_{\bm{x}\to\bm{x}_0}f(\bm{x})=A \]
    如果对充分小的$\varepsilon>0$,都存在$\delta>0$,使当$0<|\bm{x}-\bm{x}_0|<\delta$时,恒有
    \[ |f(\bm{x}-A|<\varepsilon \]
\end{definition}

\subsection{二重极限}
按照分量表示法,二元函数在内点的极限可以定义如下
\begin{definition}
    设函数$f(x,y)$在$(x_0,y_0)$点的某邻域内有定义,称$(x,y)\to(x_0,y_0)$时,$f(x,y)$的极限为$A$,记作
    \[ \lim_{(x,y)\to(x_0,y_0)} f(x,y) = A \]
    如果对任意小的$\varepsilon>0$,都存在$\delta>0$,使当
    \[ 0<|x-x_0|^2 +|y-y_0|^2<\delta \]
    时,恒有
    \[ |f(x,y)-A| < \varepsilon \]
\end{definition}

多元函数的极限具有唯一性、局部有界性、局部保号性、四则运算法则、两边夹法则、变量代换法则和海因定理等。

\begin{theorem}
    如果对任意的$\varepsilon>0$,当$(x,y)\to(x_0,y_0)$时,恒有
    \[ A-\varepsilon \leftarrow f(x,y,\varepsilon) \leq h(x,y)\leq g(x,y,\varepsilon) \to A + \varepsilon \]
    则
    \[ \lim_{(x,y)\to(x_0,y_0)} h(x,y) = A \]
\end{theorem}

变量代换法则可以表述为
\begin{theorem}
    设$\lim_{(x,y)\to(x_0,y_0)} f(x,y) = A, \lim_{t\to A} g(t) = B$,若$(x,y)\to(x_0,y_0)$时,恒有
    $f(x,y)\to,\neq A$,则
    \[ \lim_{(x,y)\to(x_0,y_0)} g[f(x,y)] = \lim_{t\to A}g(t) = B \]
\end{theorem}
更一般地,有
\begin{theorem}
    设$\lim_{(x,y)\to(x_0,y_0)} f(x,y) = A$,若$(u,v)\to(u_0,v_0)$时,恒有
    \[ (x(u,v),y(u,v))\to,\neq(x_0,y_0) \]
    则
    \[ \lim_{(u,v)\to(u_0,v_0)}f(x(u,v),y(u,v)) = \lim_{(x,y)\to(x_0,y_0)} f(x,y) = A \]
\end{theorem}

海因定理可以描述为\textbf{\textsf{二重极限存在=数列话极限恒存在且相等}},即
\begin{theorem}
    $\lim_{(x,y)\to(x_0,y_0)}f(x,y) = A \iff $当$n\to\infty$时$(x_n,y_n)\to,\neq(x_0,y_0)$,恒有
    \[ \lim_{n\to\infty} f(x_n,y_n) = A \]
\end{theorem}
\begin{situation}
    海因定理常用来否定极限的存在性,更常用的是一元化方法:\textbf{\textsf{二重极限存在=一元化极限存在且相等}}。
\end{situation}
\begin{theorem}
    $\lim_{(x,y)\to(x_0,y_0)}f(x,y) = A \iff $当$x=x(t),y=y(t)$为连续曲线且$(x(t),y(t))=(x_0,y_0)$只有$t=t_0$解时,
    恒有
    \[ \lim_{t\to t_0}f(x(t),y(t)) = A \]
\end{theorem}

对于二元函数,其在$(x_0,y_0)$处的极限存在,根据上述定理可知,对于任意一条参数化曲线$x=x(t),y=y(t)$,其逼近$(x_0,y_0)$时,
所得极限值需要相等。

\begin{example}
    证明函数$f(x,y)=(1+xy)^{\dfrac{1}{x+y}}$在原点$(0,0)$处的极限不存在。
\end{example}
\begin{proof}
    \[ \lim_{(x,y)\to(0,0)}f(x,y) = \exp\left(\lim_{(x,y)\to(0,0)}\frac{\ln(1+xy)}{x+y}\right) \]
    令$y=\lambda x$,
    则
    \[
        \lim_{(x,y)\to(0,0)}f(x,y)
        = \exp\left(\lim_{x\to 0}\frac{\ln(1+\lambda x^2)}{(1+\lambda)x}\right)
        = \exp\left(\lim_{x\to 0}\frac{\lambda x}{1+\lambda}\right)
    \]
    其中$\lambda \neq -1$时,原极限为$1$。当$\lambda = -1$时,需要考察一条曲线在$(0,0)$处切线为$y=-x$时的逼近,
    不妨令$y=x^2-x$,则有
    \[
        \lim_{(x,y)\to(0,0)}f(x,y)
        = \lim_{x\to 0} f(x,x^2-x)
        = \exp\left(\lim_{x\to 0}\frac{\ln(1+x^3-x^2)}{x^2}\right)
        = \frac{1}{\mathrm{e}}
    \]
    故沿两个曲线逼近$(0,0)$的极限不相等,所以愿继续不存在。
\end{proof}

\subsection{二次极限}
形如
\[ \lim_{y\to y_0} \lim_{x\to x_0}f(x,y),\qquad \lim_{x\to x_0}\lim_{y\to y_0}f(x,y) \]
的极限称为二次极限。

二重极限与二次极限没有直接关系,但存在间接关系。
\begin{theorem}
    若$\lim_{(x,y)\to(x_0,y_0)}f(x,y)$存在且当$x\to,\neq x_0$时,$\lim_{y\to y_0}f(x,y)$存在,
    则$\lim_{x\to x_0} \lim_{y\to y_0}f(x,y)$存在且
    \[ \lim_{(x,y)\to(x_0,y_0)}f(x,y) = \lim_{x\to x_0} \lim_{y\to y_0}f(x,y) \]
\end{theorem}

\begin{theorem}
    若
    \[\lim_{(x,y)\to(x_0,y_0)}f(x,y)\]
    \[\lim_{x\to x_0} \lim_{y\to y_0}f(x,y)\]
    \[\lim_{y\to y_0}\lim_{x\to x_0}f(x,y)\]
    存在,则三者相等。
\end{theorem}

\begin{theorem}
    若二次极限
    \[ \lim_{x\to x_0} \lim_{y\to y_0}f(x,y),\lim_{y\to y_0}\lim_{x\to x_0}f(x,y) \]
    都存在,但不相等,则二重极限$\lim_{(x,y)\to(x_0,y_0)}f(x,y)$不存在。
\end{theorem}

\subsection{多元函数的连续性}
如果函数的极限值等于函数值,即
\[ \lim_{(x,y)\to(x_0,y_0)} f(x,y) = f(x_0,y_0) \]
则称$f(x,y)$在$(x_0,y_0)$处连续。如果函数$f(x,y)$在区域或闭区域$E$上的每个点都连续,则称$f(x,y)$是$E$上的连续函数,
记作$f\in C(E)$

连续函数的四则运算一定是连续函数,连续函数的复合函数也一定是连续函数。对于二元初等函数,其在有定义的区域内都是连续函数,
且满足有限次四则运算、有限次复合运算得到的二元函数也称为二元初等函数。

在有界闭区域上的连续函数具有有界性、最值性、介值性和一直连续性。

\section{多元函数的偏导数}
\subsection{一阶偏导数}
\subsubsection{偏导数的概念}
一元函数$x\mapsto f(x,y)$的导数称为二元函数$(x,y)\mapsto f(x,y)$的偏导数,记为$f'_x(x,y)$,用极限语言描述为
\begin{definition}
    设函数$z=f(x,y)$在$(x_0,y_0)$点的某个邻域内有定义,如果
    \[ \lim_{\delta x\to 0} \frac{f(x_0+\delta x,y_0)-f(x_0,y_0)}{\delta} \]
    存在,则称此极限为函数$z=f(x,y)$在$(x_0,y_0)$点对$x$的偏导数,记作
    \[
        f'_x(x_0,y_0),\,
        f'_1(x_0,y_0),\,
        \frac{\partial f}{\partial x}(x_0,y_0),\,
        \left.\frac{\partial f}{\partial x}\right|_{(x_0,y_0)},\,
        \left.\frac{\partial z}{\partial x}\right|_{(x_0,y_0)},\,
        z'_{(x_0,y_0)}
    \]
\end{definition}

由此定义可知,二元函数是否可导与是否连续没有直接关系。即\textcolor{red}{“可导$\centernot\implies$连续”}

如果函数$z=f(x,y)$在区域$D$内的每一个点对$x$的偏导数都存在,那么这个偏导数也是$x,y$的导数,称为偏导函数,也简称偏导数。
如果两个偏导数都存在时,则称二原函数可导。

\begin{situation}
    求一个给定点的偏导数,通常时先求偏导函数,再求骗到函数的函数值。对于特殊点的偏导数,大多需要用定义计算,
    或用非求导变量先代入的一元化方法。
\end{situation}

\subsubsection{偏导数的几何意义}
设函数$z=f(x,y)$在区域$D$上连续可导,则它的图像时一个曲面,记为$\Sigma$。那么对于曲面上任意一点$(x_0,y_0,z_0)$,
其延$x,y$轴方向的切向量为
\[ (1,0,f'_x(x,y)),\quad(0,1,f'_y(x,y)) \]
则法向量为
\begin{equation}
    \bm{n} = (1,0,f'_x(x,y)\times(0,1,f'_y(x,y))) = (-z'_x, -z'_y, 1)
\end{equation}
得到了法向量,就可以构建此点的切平面方程和法线方程。

更一般地,偏导数可以用来表示参数曲面的坐标线切下来和曲面法向量。设参数曲面
\[ \Sigma : \bm{r}=(x(u,v),y(u,v),z(u,v)) \]
可以看成时$u$-曲线
\[ u \mapsto (x(u,v),y(u,v),z(u,v)) \]
和$v$-曲线编织的图形。根据切向量的定义,两条曲线的切向量分别为
\[
    \frac{\partial\bm{r}}{\partial u} = \left(\frac{\partial x}{\partial u}, \frac{\partial y}{\partial u}, \frac{\partial z}{\partial u}\right)
    ,\quad
    \frac{\partial\bm{r}}{\partial v} = \left(\frac{\partial x}{\partial v}, \frac{\partial y}{\partial v}, \frac{\partial z}{\partial v}\right)
\]
则法向量为
\begin{equation}
    \bm{n} = \frac{\partial\bm{r}}{\partial u} \times \frac{\partial\bm{r}}{\partial v} =
    \begin{vmatrix}
        \bm{i} & \bm{j} & \bm{k} \\
        x'_u   & y'_u   & z'_u   \\
        x'_v   & y'_v   & z'_v
    \end{vmatrix}
    =
    \left(\frac{\partial(y,z)}{\partial(u,v)},\frac{\partial(z,x)}{\partial(u,v)}, \frac{\partial(x,y)}{\partial(u,v)}\right)
\end{equation}
其中
\[
    \frac{\partial(y,z)}{\partial(u,v)} =
    \begin{vmatrix}
        y'_u & z'_u \\
        y'_v & z'_v
    \end{vmatrix},\quad
    \frac{\partial(z,x)}{\partial(u,v)} =
    \begin{vmatrix}
        z'_u & x'_u \\
        z'_v & x'_v
    \end{vmatrix},\quad
    \frac{\partial(x,y)}{\partial(u,v)} =
    \begin{vmatrix}
        x'_u & y'_u \\
        x'_v & y'_v
    \end{vmatrix}
\]
并称为\textcolor{red}{\textbf{\textsf{雅可比行列式}}}。

\subsection{高阶偏导数}
对于函数$z=f(x,y)$,其偏导数$f'_x(x,y), f'_y(x,y)$的偏导数称为$z=f(x,y)$的二阶偏导数,记为
\begin{align*}
    \frac{\partial}{\partial x}\left(\frac{\partial z}{\partial x}\right) & = \frac{\partial^2 z}{\partial x^2} = \frac{\partial^2 f}{\partial x^2} = z''_{xx}=f''_x(x,y)                     \\
    \frac{\partial}{\partial y}\left(\frac{\partial z}{\partial x}\right) & = \frac{\partial^2 z}{\partial y \partial x} = \frac{\partial^2 f}{\partial y\partial x} = z''_{xy}=f''_{xy}(x,y) \\
    \frac{\partial}{\partial x}\left(\frac{\partial z}{\partial y}\right) & = \frac{\partial^2 z}{\partial x\partial y} = \frac{\partial^2 f}{\partial x\partial y} = z''_{yx}=f''_{yx}(x,y)  \\
    \frac{\partial}{\partial y}\left(\frac{\partial z}{\partial y}\right) & = \frac{\partial^2 z}{\partial y^2} = \frac{\partial^2 f}{\partial y^2} = z''_{yy}=f''_{yy}(x,y)                  \\
\end{align*}
其中$\dfrac{\partial^z}{\partial y\partial x}$称为$z$先对$x$后对$y$的二阶混合偏导数。

对于二阶混合偏导数能否交换偏导顺序,有如下定理
\begin{theorem}
    设$f''_{xy}(x,y),f''_{yx}(x,y)$在$(x_0,y_0)$点连续,则
    \[ f''_{xy}(x_0,y_0) = f''_{yx}(x_0,y_0) \]
\end{theorem}

\section{多元函数的全微分}
\subsection{一阶全微分}
\subsubsection{全微分的概念}
对于二元函数$z=f(x,y)$的增量
\[ \Delta z = f(x+\Delta x,y+\Delta y) - f(x,y) \]
用线性逼近的形式应为$\diff z = A \Delta x + B \Delta y$,最佳逼近条件为
\[ \Delta z- \diff z = \Delta z - A\Delta x - B \Delta y = o(\rho) \quad (\rho \to 0) \]
其中$\rho = \sqrt{\Delta x^2 + \Delta y^2}$。

根据上述描述,可定义
\begin{definition}
    设函数$z=f(x,y)$在$(x,y)$点的某邻域内有定义。如果存在常数$A,B$使当
    \[ \rho = \sqrt{\Delta x^2 + \Delta y^2} \to 0 \]
    时,恒有
    \[ \Delta z - A\Delta x - B\Delta y = o(\rho) \]
    则称$A\Delta x + B\Delta y$为函数$z=f(x,y)$在$(x,y)$点的微分,记作
    \[ \diff z = \diff f(x,y) = A\Delta + B\Delta \]
\end{definition}
微分存在时称函数可微,当函数在区域$D$内每个点都可微时,称函数时区域$D$内的可微函数。

要判断函数的可微性,首先需要微分$A\Delta x+ B\Delta y$的系数;其次需要证明极限关系
\begin{equation}
    \lim_{\substack{\Delta x\to 0\\ \Delta y \to 0}}
    = \frac{\Delta z - (A\Delta x + B\Delta y)}{\sqrt{\Delta x^2 + \Delta y^2}}
    =0
\end{equation}

\subsubsection{微分中值定理}
对于一元可微函数$y=f(x)$,拉格朗日中值定理\ref{th:拉格朗日中值定理},
\[ f(x+\Delta x) = f(x) + f'(x+\theta x)\Delta x,\quad (0<\theta<1) \]
将这一定理推广至二元函数,则有
\begin{theorem}
    (微分中值定理)
    \label{th:微分中值定理}
    设函数$z=f(x,y)$在矩形域
    \[ D = \{(x,y)\,|\,|x-x_0|<\delta_1, |y-y_0| < \delta_2 \} \]
    上有定义,若$f(x,y)$在$D$内可导,则当$|\Delta x|<\delta_1, |\Delta y|<\delta_2$时,
    必存在$\theta_1,\theta_2\in(0,1)$,使得
    \begin{align*}
        \Delta z & = f(x_0+\Delta x, y_0 +\Delta y) - f(x_0,y_0)                                                 \\
                 & =f'_x(x_0+\theta_1\Delta x, y_0 + \Delta y) \Delta x + f'_y(x_0,y_0+\theta_2\Delta y)\Delta y
    \end{align*}
\end{theorem}
可以理解为函数先沿$x$-曲线变化到$f(x_0,y_0+\Delta y)$,这个变化与$f'_y$相关;
之后再延$y$-曲线变化到$f(x_0+\Delta x, y_0+\Delta y)$,这个变化与$f'_x$相关。

作为推论有
\begin{theorem}
    对于微分中值定理的推论,有
    \begin{enumerate}[(1)]
        \item 设函数$z=f(x,y)$在区域$D$内可导且两个偏导数恒为零,则$f(x,y)$在$D$内必为常数。
        \item 设函数$z=f(x,y)$在$(x_0,y_0)$点的某邻域内可导且两偏导数有界,则$f(x,y)$在$(x_0,y_0)$点连续。
    \end{enumerate}
\end{theorem}

\subsubsection{可微性条件}
用定义直接判断可微性是比较困难的,因此有必要探究可微的条件。
\begin{theorem}
    (可微的必要条件)
    若函数$z=f(x,y)$在$(x,y)$点可微,$\implies$
    \begin{enumerate}[(1)]
        \item $f(x,y)$在$(x,y)$点连续;
        \item $f(x,y)$在$(x,y)$点可导,且$\diff z = f'_x(x,y)\Delta x+ f'_y(x,y)\Delta y$;
        \item 当$x,y$为自变量时,有
              \[ \diff z = f'_x(x,y)\,\diff x, f'_y(x,y)\,\diff y \]
    \end{enumerate}
\end{theorem}

可微可以推出连续,也可以推出可导,但可导不能推出连续,自然不能推出可微。判断可微一般通过下面的定理来得出。
\begin{theorem}
    (可微的充分条件)
    \label{th:可微的充分条件}
    若函数$z=f(x,y)$在$(x,y)$点的偏导数连续,则$z=f(x,y)$在$(x,y)$点处可微。
\end{theorem}
这里要注意偏导是可微的\textcolor{red}{充分条件但非必要条件}。例如下面的例子中$f(x,y)$在$(0,0)$点
偏导不连续,但却可微。
\begin{example}
    讨论函数
    \[
        f(x,y) =
        \begin{cases}
            (x^2+y^2)\sin \frac{1}{x^2+y^2}, & x^2+y^2 \neq 0, \\
            0,                               & x^2+y^2=0
        \end{cases}
    \]
    在$(0,0)$点是否可微,其偏导数是否连续。
\end{example}
\begin{solution}
    \begin{align*}
          & \lim_{\substack{\Delta x\to 0  \\ \Delta y \to 0}} \frac{f(\Delta x, \Delta y)-(A\Delta x+ B\Delta y)}{\sqrt{\Delta x^2 + \Delta y^2}}\\
        = & \lim_{\substack{\Delta x\to 0  \\ \Delta y \to 0}} \frac{(\Delta x^2+\Delta y^2)\sin \frac{1}{\Delta x^2+\Delta y^2} - (A\Delta x + B \Delta y)}{\sqrt{\Delta x^2 + \Delta y^2}}\\
        = & \lim_{\substack{\Delta x\to 0  \\ \Delta y \to 0}} \sqrt{\Delta x^2 + \Delta y^2}\sin \frac{1}{\Delta x^2 +\Delta y^2} - \lim_{\substack{\Delta x\to 0 \\ \Delta y \to 0}} \frac{A\Delta x + B\Delta y}{\sqrt{\Delta x^2 + \Delta y^2}}\\
        = & -\lim_{\substack{\Delta x\to 0 \\ \Delta y \to 0}} \frac{A\Delta x + B\Delta y}{\sqrt{\Delta x^2 + \Delta y^2}}
    \end{align*}
    其中$A,B$为常数,当$A=B=0$时原极限存在且为零,此时可得$f(x,y)$在$(0,0)$处可微。

    当$ x^2+y^2 \neq 0$时,
    \begin{align*}
        f'_x(x,y) & = 2x\sin\frac{1}{x^2+y^2} -\frac{2x}{x^2+y^2}\cos\frac{1}{x^2+y^2} \\
        f'_y(x,y) & = 2y\sin\frac{1}{x^2+y^2} -\frac{2y}{x^2+y^2}\cos\frac{1}{x^2+y^2}
    \end{align*}
    当$x^2+y^2=0$,即$x=y=0$时,
    \begin{align*}
        f'_x(0,0) & = \lim_{x\to 0} \frac{f(x,0)-f(0,0)}{x-0} = \lim_{x\to 0}x\sin\frac{1}{x^2} = 0 \\
        f'_y(0,0) & = \lim_{y\to 0} \frac{f(0,y)-f(0,0)}{y-0} = \lim_{y\to 0}y\sin\frac{1}{y^2} = 0
    \end{align*}
    由于
    \[ \lim_{x\to 0} f'_x(x,0) = \lim_{x\to 0}\frac{-2}{x}\cos\frac{1}{x^2} = \text{不存在} \]
    \[ \lim_{y\to 0} f'_y(0,y) = \lim_{y\to 0}\frac{-2}{y}\cos\frac{1}{y^2} = \text{不存在} \]
    所以偏导函数$f'_x(x,y),f'_y(x,y)$在$(0,0)$点不连续。
\end{solution}

\begin{theorem}
    若$f'_x(x,y)$在$(x_0,y_0)$处连续,$f'_y(x,y)$在$(x_0,y_0)$点存在,则
    $f(x,y)$在$(x_0,y_0)$点可微。
\end{theorem}

利用一元函数的可微与可导的等价性,可以证明一元可微函数与二元可微函数的复合函数仍是可微函数,即
\begin{theorem}
    设函数$u=f(x,y)$在$(x_0,y_0)$点可微,函数$z=g(u)$在$u_0=f(x_0,y_0)$点可微,
    则函数$z=g(f(x,y))$在$(x_0,y_0)$点可微并且有
    \[ \diff z|_{(x_0,y_0)} = g'(u_0)\,\diff f(x_0,y_0) = g'(u_0)\left[f'_x(x_0,y_0)\,\diff x + f'_y(x_0,y_0)\,\diff y\right] \]
\end{theorem}

\subsubsection{一阶全微分不变性}
对于可微函数的复合形式
\[ z= f(\varphi(x,y),\psi(x,y)) \]
或联立形式
\[ z = f(u,v), u=\varphi(x,y), v=\psi(x,y) \]
可以证明微分公式
\begin{equation}
    \diff z
    = \frac{\partial f}{\partial u}\,\diff \varphi + \frac{\partial f}{\partial v}\,\diff \psi
    = f'_1\,\diff \varphi + f'_2\,\diff \psi
\end{equation}
这一性质称为\textcolor{red}{\textbf{\textsf{一阶全微分的不变性}}}。
\begin{proof}
    设可微函数$z=F(x,y) = f(\varphi(x,y), \psi(x,y))$,则
    \begin{align*}
        \diff F & = \frac{\partial F}{\partial x}\,\diff x + \frac{\partial F}{\partial y}\,\diff y                                                                      \\
                & = \left(\frac{\partial f}{\partial u}\frac{\partial\varphi}{\partial x} + \frac{\partial f}{\partial v}\frac{\partial\psi}{\partial x}\right)\,\diff x
        +\left(\frac{\partial f}{\partial u}\frac{\partial\varphi}{\partial y} + \frac{\partial f}{\partial v}\frac{\partial\psi}{\partial y}\right)\,\diff y            \\
                & =\frac{\partial f}{\partial u}\,\diff \varphi + \frac{\partial f}{\partial v}\,\diff \psi
    \end{align*}
\end{proof}

对于隐函数的求导委托,可以方程两端同时微分后解出方程即可。
\begin{example}
    设
    \[
        \begin{cases}
            x=-u^2+v+z, \\
            y=u+vz
        \end{cases}
    \]
    求$\dfrac{\partial u}{\partial x}, \dfrac{\partial v}{\partial x},\dfrac{\partial u}{\partial y}$
\end{example}
\begin{solution}
    方程组两端同时微分可得
    \[
        \begin{cases}
            \diff x = -2u\,\diff u + \diff v + \diff z \\
            \diff y = \diff u + z\,\diff v + v\,\diff z
        \end{cases}
    \]
    解得
    \[ \diff u = \frac{-z\,\diff x + (z-v)\,\diff z + \diff y}{2uz + 1} \]
    \[ \diff v = \frac{\diff x + 2u\,\diff y - (1+2uv)\,\diff z}{2uz + 1} \]
    由此可得
    \[
        \frac{\partial u}{\partial x} = -\frac{z}{2uv+1},\quad
        \frac{\partial v}{\partial x} = \frac{1}{2uz+1},\quad
        \frac{\partial u}{\partial y} = \frac{1}{2uz+1}
    \]
\end{solution}

\subsubsection{微分的几何意义}
微分的几何意义为:
\[ \text{可微~} \iff \text{ 可导~}+\text{ 切平面存在} \]
微分就是切平面函数的增量。

\subsection{高阶全微分}
二元函数$z=f(x,y)$的全微分时一个四元函数
\[ (x,y,\diff x,\diff y)\mapsto f'_x(x,y)\,\diff x + f'_y(x,y)\,\diff y \]
如果$x,y$是自变量,则$\diff x,\diff y$也是自变量,此时$\diff x,\diff y$可以视为相对常量。在这种约定下,二元函数
$F(x,y)=f'_x(x,y)\,\diff x+ f'_y(x,y)\,\diff y$的微分$\diff F$称为二元函数$z=f(x,y)$的二阶全微分$\diff^2f(x,y)$,即
\begin{equation}
    \diff^2 f
    = \diff F
    = \frac{\partial^2 f}{\partial x^2}\,\diff x\diff x
    + \frac{\partial^2 f}{\partial x\partial y}\,\diff x\diff y
    + \frac{\partial^2 f}{\partial y\partial x}\,\diff y\diff x
    + \frac{\partial^2 f}{\partial y^2}\,\diff y\diff y
\end{equation}
作交换性约定
\[
    \diff x\diff x = \diff x^2,\, \diff x\diff y= \diff y\diff x,\, \diff y\diff y=\diff y^2
\]
则有全微分的二项式公式
\begin{equation}
    \diff^2 f
    = \frac{\partial^2 f}{\partial x^2}\,\diff x^2
    + 2\frac{\partial^2 f}{\partial x\partial y}\,\diff x\diff y
    + \frac{\partial^2 f}{\partial y^2}\,\diff y^2
    = \left(\frac{\partial}{\partial x}\,\diff x + \frac{\partial}{\partial y}\,\diff y\right)^2 f
\end{equation}
值得注意的是当$x,y$不是自变量时,全微分二项式公式不成立。而应为
\begin{equation}
    \begin{split}
        \diff^2 z
        &= \diff \left(\frac{\partial f}{\partial x}\,\diff x + \frac{\partial f}{\partial y}\,\diff y\right)                    \\
        &= \diff\left(\frac{\partial f}{\partial x}\right)\,\diff x
        + \frac{\partial f}{\partial x}\,\diff^2 x
        + \diff\left(\frac{\partial f}{\partial y}\right)\,\diff y
        + \frac{\partial f}{\partial y}\,\diff^2 y                                                                                 \\
        &= \left(\frac{\partial^2 f}{\partial x^2}\,\diff x + \frac{\partial^2 f}{\partial y\partial x}\,\diff y\right)\,\diff x
        + \frac{\partial f}{\partial x}\,\diff^2 x
        + \left(\frac{\partial^2 f}{\partial x\partial y}\,\diff x + \frac{\partial^2 f}{\partial^2 y}\,\diff y\right)
        + \frac{\partial f}{\partial y}\,\diff^2 y                                                                                 \\
        &= \left(\frac{\partial}{\partial x}\,\diff x + \frac{\partial}{\partial y}\,\diff y\right)^2 f
        + \frac{\partial f}{\partial x}\,\diff^2 x
        + \frac{\partial f}{\partial y}\,\diff^2 y
    \end{split}
\end{equation}
其中$\diff^2 x$为$x$的二阶全微分,$\diff^2 y$为$y$的二阶全微分。


\section{多元函数的微分法}
\subsection{复合函数微分法}

\begin{theorem}
    (链法则)
    \label{th:链法则}
    设函数$u=\varphi(x,y),v=\psi(x,y)$在$(x,y)$点可导,函数$z=f(u,v)$在对应点$(u,v)$可微,则复合函数$z=f(u(x,y),v(x,y))$在$(x,y)$点可导
    且
    \[
        \frac{\partial z}{\partial x} = \frac{\partial f}{\partial u}\cdot\frac{\partial\varphi}{\partial x} + \frac{\partial f}{\partial v}\cdot\frac{\partial\psi}{\partial x}
    \]
    \[
        \frac{\partial z}{\partial y} = \frac{\partial f}{\partial u}\cdot\frac{\partial\varphi}{\partial y} + \frac{\partial f}{\partial v}\cdot\frac{\partial\psi}{\partial y}
    \]
\end{theorem}

\begin{theorem}
    更一般的链法则:设函数$u=f(x_1,x_2,\cdots,x_n)$可微,$x_i=x_i(t_1,t_2,\cdots,t_m)$可导,则复合函数$u=f(x_1(t_1,t_2,\cdots,t_m),\cdots,x_n(t_1,t_2,\cdots,t_m))$可导,且有
    \[
        \frac{\partial u}{\partial t_j} = \sum_{i=1}^n \frac{\partial f}{\partial x_i}\cdot\frac{\partial x_i}{\partial t_j}, \quad j = 1,2,...,m
    \]
\end{theorem}

\subsection{隐函数微分法}
\subsubsection{隐函数定理}
设$F(x,y)=0$时一个二元函数,如果方程
\[ F(x,y) = 0 \]
存在一个解函数$y=f(x)$,则称这个解函数微隐函数。
\begin{theorem}
    (隐函数定理)
    \label{th:隐函数定理}
    设函数$F(x,y),F'_y(x,y)$在$(x_0,y_0)$附近连续且
    \[ F(x_0,y_0) = 0, \quad F'_y(x_0,y_0)\neq 0 \]
    则在$(x_0,y_0)$点的某个邻域内,唯一存在连续函数$y=f(x)$,使得
    \[ F(x,f(x)) \equiv 0 \]
    并当$F'_x(x,y)$在该邻域连续时,$y=f(x)$连续可导且
    \[ f'(x) = - \frac{F'_x(x,y)}{F'_y(x,y)} \]
\end{theorem}

更一般的隐函数定理是
\begin{theorem}
    设$x=(x_1,x_2,\cdots,x_n),y=(y_1,y_2,\cdots,y_m)$,若函数
    \[ F(x,y) = (F_1(x,y),\cdots,F_m(x,y)) \]
    在$(x_0,y_0)$附近连续可导,且
    \[ F(x_0,y_0) = 0, \quad\left.\frac{\partial(F_1,\cdots,F_m)}{\partial(y_1,\cdots,y_m)}\right|_{(x_0,y_0)} \neq 0 \]
    则在$(x_0,y_0)$点的某个邻域内,唯一存在连续可导函数$y=f(x)$,使得
    \[ F(x,f(x))\equiv 0 \]
    且当$1\leq i \leq n, 1\leq j\leq m$时,恒有
    \[
        \frac{\partial y_j}{\partial x_i}
        =
        \left.
        - \frac{\partial(F_1,\cdots,F_m)}{\partial(y_1,\cdots,y_{j-1},x_i,y_{j+1},\cdots,y_m)}
        \middle/
        \frac{\partial(F_1,\cdots,F_m)}{\partial(y_1,\cdots,y_m)}
        \right.
    \]
\end{theorem}

隐函数求导公式的特征可概括为:\textbf{\textsf{雅可比,变分子;$A$导$B$,$A$换$B$}}。
其中$F_1,F_2,\cdots,F_m$为方程组
\[
    \begin{cases}
        F_1(x,y) = 0, \\
        F_2(x,y) = 0, \\
        \cdots        \\
        F_m(x,y) = 0
    \end{cases}
\]
分母为
\begin{equation}
    \frac{\partial(F_1,\cdots,F_m)}{\partial(y_1,\cdots,y_m)}
    =
    \begin{vmatrix}
        \frac{\partial F_1}{\partial y_1} & \dots  & \frac{\partial F_1}{\partial y_m} \\
        \vdots                            & \ddots & \vdots                            \\
        \frac{\partial F_m}{\partial y_1} & \dots  & \frac{\partial F_m}{\partial y_m}
    \end{vmatrix}
\end{equation}
计算$y_j$对$x_i$的偏导时,只需将分母雅可比中的$y_j$替换成$x_i$,即可得出$\dfrac{\partial y_j}{\partial x_i}$对应的分子雅可比。

\subsubsection{反函数定理}
.
\begin{theorem}
    (雅可比法则)
    \label{th:雅可比法则}
    设函数
    \[ z_i = z_i(y_i,\cdots,y_n),\quad y_i=y_i(x_1,\cdots,x_n),\quad (i=1,\cdots n) \]
    在拟讨论点附近连续可导,则
    \[
        \frac{\partial(z_1,\cdots,z_n)}{\partial(x_1,\cdots,x_m)}
        = \frac{\partial(z_1,\cdots,z_n)}{\partial(y_1,\cdots,y_n)} \cdot \frac{\partial(y_1,\cdots,y_n)}{\partial(x_1,\cdots,x_n)}
    \]
\end{theorem}

\begin{theorem}
    (反函数定理)
    \label{th:反函数定理}
    设函数
    \[ y_i=y_i(x_1,\cdots,x_n),\quad (i=1,\cdots,n) \]
    在拟讨论点附近可导且$\dfrac{\partial(y_1,\cdots,y_n)}{\partial(x_1,\cdots,x_n)}\neq 0$,则在该点附近反函数
    \[ x_i = x_i(y_1,\cdots,y_n),\quad (i=1,\cdots,n) \]
    存在且连续可导,并有
    \begin{enumerate}[(1)]
        \item
              $
                  \dfrac{\partial x_i}{\partial y_j}
                  = \left.(-1)^{i+j} \dfrac{\partial(y_1,\cdots,y_{j-1},y_{j+1},\cdots,y_n)}{\partial(x_1,\cdots,x_{i-1},x_{i+1},\cdots,x_n)}
                  \middle/
                  \dfrac{\partial(y_1,\cdots,y_n)}{\partial(x_1,\cdots,x_n)}
                  \right.
              $
        \item
              $
                  \dfrac{\partial(x_1,\cdots,x_n)}{\partial(y_1,\cdots,y_n)} = \left.1\middle/\dfrac{\partial(y_1,\cdots,y_n)}{\partial(x_1,\cdots,x_n)}\right.
              $
    \end{enumerate}
\end{theorem}

\section{偏导数的应用}