\part{行列式}
\section{行列式的性质}
行列式有六个性质常用于计算当中。
\begin{property}
    行列式与它的转置行列式相等,即
    \[
        \begin{vmatrix}
            a_{11} & a_{12} & \cdots & a_{1n} \\
            a_{21} & a_{22} & \cdots & a_{2n} \\
            \vdots & \vdots & \ddots & \vdots \\
            a_{n1} & a_{n2} & \cdots & a_{nn}
        \end{vmatrix}
        =
        \begin{vmatrix}
            a_{11} & a_{21} & \cdots & a_{n1} \\
            a_{12} & a_{22} & \cdots & a_{n2} \\
            \vdots & \vdots & \ddots & \vdots \\
            a_{1n} & a_{2n} & \cdots & a_{nn}
        \end{vmatrix}
    \]
\end{property}
上面这个性质表达了行列式的行列对称性,因此行的性质也就等价于列的性质。此后讨论的行列式性质,只研究行的性质。

\begin{property}
    行列式的某行公因子可以提到行列式外,即
    \[
        \begin{vmatrix}
            a_{11}  & a_{12}  & \cdots & a_{1n}  \\
            \vdots  & \vdots  & \ddots & \vdots  \\
            ka_{t1} & ka_{t2} & \cdots & ka_{tn} \\
            \vdots  & \vdots  & \ddots & \vdots  \\
            a_{n1}  & a_{n2}  & \cdots & a_{nn}
        \end{vmatrix}
        = k
        \begin{vmatrix}
            a_{11} & a_{12} & \cdots & a_{1n} \\
            \vdots & \vdots & \ddots & \vdots \\
            a_{t1} & a_{t2} & \cdots & a_{tn} \\
            \vdots & \vdots & \ddots & \vdots \\
            a_{n1} & a_{n2} & \cdots & a_{nn}
        \end{vmatrix}
        , \qquad t = 1, 2, \cdots, n
    \]
\end{property}

\begin{property}
    交换行列式中的任意两行,行列式改变符号,即
    \[
        \begin{vmatrix}
            a_{11} & a_{12} & \cdots & a_{1n} \\
            \cdots & \cdots & \cdots & \cdots \\
            a_{t1} & a_{t2} & \cdots & a_{tn} \\
            \cdots & \cdots & \cdots & \cdots \\
            a_{s1} & a_{s2} & \cdots & a_{sn} \\
            \cdots & \cdots & \cdots & \cdots \\
            a_{n1} & a_{n2} & \cdots & a_{nn}
        \end{vmatrix}
        = -
        \begin{vmatrix}
            a_{11} & a_{12} & \cdots & a_{1n} \\
            \cdots & \cdots & \cdots & \cdots \\
            a_{s1} & a_{s2} & \cdots & a_{sn} \\
            \cdots & \cdots & \cdots & \cdots \\
            a_{t1} & a_{t2} & \cdots & a_{tn} \\
            \cdots & \cdots & \cdots & \cdots \\
            a_{n1} & a_{n2} & \cdots & a_{nn}
        \end{vmatrix}
    \]
\end{property}

\begin{property}
    行列式有两行成比例,则行列式等于零,即
    \[
        \begin{vmatrix}
            a_{11}  & a_{12}  & \cdots & a_{1n}  \\
            \cdots  & \cdots  & \cdots & \cdots  \\
            a_{t1}  & a_{t2}  & \cdots & a_{tn}  \\
            \cdots  & \cdots  & \cdots & \cdots  \\
            ka_{t1} & ka_{t2} & \cdots & ka_{tn} \\
            \cdots  & \cdots  & \cdots & \cdots  \\
            a_{n1}  & a_{n2}  & \cdots & a_{nn}
        \end{vmatrix}
        =
        0
    \]
\end{property}

\begin{property}
    行列式中,某行元素若可以表示为两个数相加,则可将其拆为两个行列式,即
    \[
        \begin{vmatrix}
            a_{11}        & a_{12}        & \cdots & a_{1n}        \\
            \vdots        & \vdots        & \ddots & \vdots        \\
            a_{t1}+b_{t1} & a_{t2}+b_{t2} & \cdots & a_{tn}+b_{tn} \\
            \vdots        & \vdots        & \ddots & \vdots        \\
            a_{n1}        & a_{n2}        & \cdots & a_{nn}
        \end{vmatrix}
        =
        \begin{vmatrix}
            a_{11} & a_{12} & \cdots & a_{1n} \\
            \vdots & \vdots & \ddots & \vdots \\
            a_{t1} & a_{t2} & \cdots & a_{tn} \\
            \vdots & \vdots & \ddots & \vdots \\
            a_{n1} & a_{n2} & \cdots & a_{nn}
        \end{vmatrix}
        +
        \begin{vmatrix}
            a_{11} & a_{12} & \cdots & a_{1n} \\
            \vdots & \vdots & \ddots & \vdots \\
            b_{t1} & b_{t2} & \cdots & b_{tn} \\
            \vdots & \vdots & \ddots & \vdots \\
            a_{n1} & a_{n2} & \cdots & a_{nn}
        \end{vmatrix}
    \]
\end{property}

\begin{property}
    行列式中,某行元素加上另一行的倍数,行列式不变
    \[
        \begin{vmatrix}
            a_{11}         & a_{12}         & \cdots & a_{1n}         \\
            \cdots         & \cdots         & \cdots & \cdots         \\
            a_{i1}         & a_{i2}         & \cdots & a_{in}         \\
            \cdots         & \cdots         & \cdots & \cdots         \\
            a_{j1}+ka_{i1} & a_{j2}+ka_{i2} & \cdots & a_{jn}+ka_{in} \\
            \cdots         & \cdots         & \cdots & \cdots         \\
            a_{n1}         & a_{n2}         & \cdots & a_{nn}
        \end{vmatrix}
        =
        \begin{vmatrix}
            a_{11} & a_{12} & \cdots & a_{1n} \\
            \cdots & \cdots & \cdots & \cdots \\
            a_{i1} & a_{i2} & \cdots & a_{in} \\
            \cdots & \cdots & \cdots & \cdots \\
            a_{j1} & a_{j2} & \cdots & a_{jn} \\
            \cdots & \cdots & \cdots & \cdots \\
            a_{n1} & a_{n2} & \cdots & a_{nn}
        \end{vmatrix}
    \]
\end{property}

\subsection{行列式的展开}
行列式可以按某行展开为其元素和与之对应的代数余子式乘积的和。如按第一行展开,则有
\[
    \begin{vmatrix}
        a_{11} & a_{12} & \cdots & a_{1n} \\
        a_{21} & a_{22} & \cdots & a_{2n} \\
        \vdots & \vdots & \ddots & \vdots \\
        a_{n1} & a_{n2} & \cdots & a_{nn}
    \end{vmatrix}
    =
    a_{11}A_{11} + a_{12}A_{12} + \cdots + a_{1n}A_{1n}
\]
其中$A_{ij}$为代数余子式,和余子式$M_{ij}$的关系为
\[ A_{ij} = (-1)^{i+j}M_{ij} \]
$M_{ij}$为去掉第$i$行,第$j$列元素后的行列式。

显然某个元素$a_{ij}$的值与其对应的代数余子式$A_{ij}$的值没有关系。
\begin{example}
    设行列式
    \[
        D =
        \begin{vmatrix}
            a_{11} & a_{12} & a_{13} \\
            a_{21} & a_{22} & a_{23} \\
            a_{31} & a_{32} & a_{33}
        \end{vmatrix}
    \]
    证明$a_{31}A_{11} + a_{32}A_{12} + a_{33}A_{13} = 0$
\end{example}
\begin{proof}
    由于某个元素$a_{ij}$的值与其对应的代数余子式$A_{ij}$的值没有关系,
    因此可以将$D$的第一行替换为第三行得出命题等式。
    \[
        a_{31}A_{11} + a_{32}A_{12} + a_{33}A_{13}
        =
        \begin{vmatrix}
            a_{31} & a_{32} & a_{33} \\
            a_{21} & a_{22} & a_{23} \\
            a_{31} & a_{32} & a_{33}
        \end{vmatrix}
        =
        0
    \]
\end{proof}
由上面例题,不难得出下述重要公式
\begin{equation}
    \sum_{k=1}^n a_{ik}A_{jk} =
    \begin{cases}
        D, & i = j    \\
        0, & i \neq j
    \end{cases}
\end{equation}
和
\begin{equation}
    \sum_{k=1}^n a_{ki}A_{kj} =
    \begin{cases}
        D, & i = j    \\
        0, & i \neq j
    \end{cases}
\end{equation}

\section{特殊的行列式}
\subsection{三角行列式}
\begin{equation}
    \begin{vmatrix}
        a_{11} &        &        &               &        \\
        a_{21} & a_{22} &        & \text{\huge0} &        \\
        a_{31} & a_{32} & a_{33} &               &        \\
        \vdots & \vdots & \vdots & \ddots        &        \\
        a_{n1} & a_{n2} & a_{n3} & \cdots        & a_{nn}
    \end{vmatrix}
    = a_{11}a_{22}a_{33}\cdots a_{nn}
\end{equation}

\begin{equation}
    \begin{vmatrix}
        a_{11} & a_{12}        & a_{13} & \cdots & a_{1n} \\
               & a_{22}        & a_{23} & \cdots & a_{2n} \\
               &               & a_{33} & \cdots & a_{3n} \\
               & \text{\huge0} &        & \ddots & \vdots \\
               &               &        &        & a_{nn}
    \end{vmatrix}
    = a_{11}a_{22}a_{33}\cdots a_{nn}
\end{equation}

\begin{equation}
    \begin{vmatrix}
               &               &           &           & a_{1n} \\
               & \text{\huge0} &           & a_{2,n-1} & a_{2n} \\
               &               & a_{3,n-2} & a_{3,n-1} & a_{3n} \\
               & \udots        & \vdots    & \vdots    & \vdots \\
        a_{n1} & \cdots        & a_{n,n-2} & a_{n,n-1} & a_{nn}
    \end{vmatrix}
    = (-1)^{\frac{1}{2}n(n-1)} a_{1n}a_{2,n-1}\cdots a_{n1}
\end{equation}

\begin{equation}
    \begin{vmatrix}
        a_{11} & \cdots & a_{1,n-2} & a_{1,n-1}     & a_{1n} \\
        a_{21} & \cdots & a_{2,n-2} & a_{2,n-1}     &        \\
        a_{31} & \cdots & a_{3,n-2} &               &        \\
        \vdots & \udots &           & \text{\huge0} &        \\
        a_{n1} &        &           &               &
    \end{vmatrix}
    = (-1)^{\frac{1}{2}n(n-1)} a_{1n}a_{2,n-1}\cdots a_{n1}
\end{equation}

\subsection{拉普拉斯行列式}
\begin{align}
    \label{eq:拉普拉斯行列式}
    \begin{vmatrix}
        A & * \\
        0 & B
    \end{vmatrix}
     & =
    \begin{vmatrix}
        A & 0 \\
        * & B
    \end{vmatrix}
    = |A| \cdot |B|
    \\
    \begin{vmatrix}
        * & A \\
        B & 0
    \end{vmatrix}
     & =
    \begin{vmatrix}
        0 & A \\
        B & *
    \end{vmatrix}
    = (-1)^{mn} |A| \cdot |B|
\end{align}
其中$|A|,|B|$分别为$n$阶行列式,$m$阶行列式。

\subsection{范德蒙行列式}
称
\begin{equation}
    \label{eq:范德蒙行列式}
    V_n =
    \begin{vmatrix}
        1         & 1         & 1         & \cdots & 1         \\
        a_1       & a_2       & a_3       & \cdots & a_n       \\
        a_1^2     & a_2^2     & a_3^2     & \cdots & a_n^2     \\
        \vdots    & \vdots    & \vdots    & \ddots & \vdots    \\
        a_1^{n-1} & a_2^{n-1} & a_3^{n-1} & \cdots & a_n^{n-1}
    \end{vmatrix}
    =
    \prod_{1\leq j < i \leq n} (a_i-a_j)
\end{equation}
为$n$阶范德蒙行列式。

\section{克拉默法则}
设线性非齐次方程组
\[
    \begin{cases}
        a_{11}x_1 + a_{12}x_2 + \cdots + a_{1n}x_n = b_1 \\
        a_{21}x_1 + a_{22}x_2 + \cdots + a_{2n}x_n = b_2 \\
        \cdots                                           \\
        a_{n1}x_1 + a_{n2}x_2 + \cdots + a_{nn}x_n = b_n
    \end{cases}
\]
和
\[
    D =
    \begin{vmatrix}
        a_{11} & a_{12} & \cdots & a_{1n} \\
        a_{21} & a_{22} & \cdots & a_{2n} \\
        \vdots & \vdots & \ddots & \vdots \\
        a_{n1} & a_{n2} & \cdots & a_{nn}
    \end{vmatrix}
    \qquad
    D_i =
    \begin{vmatrix}
        a_{11} & \cdots & a_{1,i-1} & b_1    & a_{1,i+1} & \cdots & a_{1n} \\
        a_{21} & \cdots & a_{2,i-1} & b_2    & a_{2,i+1} & \cdots & a_{2n} \\
        \vdots & \ddots & \vdots    & \vdots & \vdots    & \ddots & \vdots \\
        a_{n1} & \cdots & a_{n,i-1} & b_n    & a_{n,i+1} & \cdots & a_{nn}
    \end{vmatrix}
\]
那么有
\begin{theorem}
    (克拉默法则)
    \label{th:克拉默法则}
    若方程组的系数行列式$D\neq 0$,则方程组有唯一解。且有
    \[ x_i = \frac{D_i}{D} \]
\end{theorem}

对于齐次方程组
\[
    \begin{cases}
        a_{11}x_1 + a_{12}x_2 + \cdots + a_{1n}x_n = 0 \\
        a_{21}x_1 + a_{22}x_2 + \cdots + a_{2n}x_n = 0 \\
        \cdots                                         \\
        a_{n1}x_1 + a_{n2}x_2 + \cdots + a_{nn}x_n = 0
    \end{cases}
\]
显然有$0$解,根据克拉默法则有如下推论
\begin{enumerate}[(1)]
    \item 当方程组的系数行列式$D\neq 0$时,齐次方程组只有$0$解;
    \item 当齐次方程组有非$0$解时,则系数行列式$D=0$
\end{enumerate}

\section{行列式计算的化简方法}
下面通过一个例题来说明一些计算方法
\begin{example}
    计算$n$阶行列式
    \[
        D_n =
        \begin{vmatrix}
            x      & a      & a      & \cdots & a      \\
            a      & x      & a      & \cdots & a      \\
            \vdots & \vdots & \vdots & \ddots & \vdots \\
            a      & a      & a      & \cdots & x
        \end{vmatrix}
    \]
\end{example}
\subsection{化三角方法}
\begin{solution}
    从最后一行开始,每一行减去前面一行,则有
    \[
        D_n =
        \begin{vmatrix}
            x      & a      & a      & \cdots & a      & a      \\
            a      & x      & a      & \cdots & a      & a      \\
            \vdots & \vdots & \vdots & \ddots & \vdots & \vdots \\
            0      & 0      & 0      & \cdots & a-x    & x-a
        \end{vmatrix}
        =
        \cdots
        =
        \begin{vmatrix}
            x      & a      & a      & \cdots & a      & a      \\
            a-x    & x-a    & 0      & \cdots & 0      & 0      \\
            \vdots & \vdots & \vdots & \ddots & \vdots & \vdots \\
            0      & 0      & 0      & \cdots & a-x    & x-a
        \end{vmatrix}
    \]
    第一列加上后面所有列,得
    \[
        D_n =
        \begin{vmatrix}
            x + (n-1)a & a      & a      & \cdots & a      & a      \\
            0          & x-a    & 0      & \cdots & 0      & 0      \\
            \vdots     & \vdots & \vdots & \ddots & \vdots & \vdots \\
            0          & 0      & 0      & \cdots & 0      & x-a
        \end{vmatrix}
        =
        [x+(n-1)a](x-a)^{n-1}
    \]
\end{solution}
\subsection{递推公式法}
\begin{solution}
    \[
        D_n =
        \begin{vmatrix}
            x - a + a & a      & a      & \cdots & a      \\
            0 + a     & x      & a      & \cdots & a      \\
            \vdots    & \vdots & \vdots & \ddots & \vdots \\
            0 + a     & a      & a      & \cdots & x
        \end{vmatrix}
        =
        \begin{vmatrix}
            x - a  & a      & a      & \cdots & a      \\
            0      & x      & a      & \cdots & a      \\
            \vdots & \vdots & \vdots & \ddots & \vdots \\
            0      & a      & a      & \cdots & x
        \end{vmatrix}
        +
        \begin{vmatrix}
            a      & a      & a      & \cdots & a      \\
            a      & x      & a      & \cdots & a      \\
            \vdots & \vdots & \vdots & \ddots & \vdots \\
            a      & a      & a      & \cdots & x
        \end{vmatrix}
        =
        (x-a)D_{n-1} +
        \begin{vmatrix}
            a      & a      & a      & \cdots & a      \\
            a      & x      & a      & \cdots & a      \\
            \vdots & \vdots & \vdots & \ddots & \vdots \\
            a      & a      & a      & \cdots & x
        \end{vmatrix}
    \]
    其中
    \[
        \begin{vmatrix}
            a      & a      & a      & \cdots & a      \\
            a      & x      & a      & \cdots & a      \\
            \vdots & \vdots & \vdots & \ddots & \vdots \\
            a      & a      & a      & \cdots & x
        \end{vmatrix}
        =
        \begin{vmatrix}
            a      & a      & a      & \cdots & a      \\
            0      & x - a  & 0      & \cdots & 0      \\
            \vdots & \vdots & \vdots & \ddots & \vdots \\
            0      & 0      & 0      & \cdots & x-a
        \end{vmatrix}
        =
        a(x-a)^{n-1}
    \]
    得出递推公式
    \[ D_n = (x-a)D_{n-1} + a(x-a)^{n-1} \]
    同时有
    \begin{align*}
        D_{n-1} & = (x-a)D_{n-2} + a(x-a)^{n-2} \\
        D_{n-2} & = (x-a)D_{n-3} + a(x-a)^{n-3} \\
        \cdots                                  \\
        D_2     & = (x-a)D_1 + a(x-a)
    \end{align*}
    将$(x-a)$乘到第二个式子,$(x-a)^2$乘到第三个式子,以此类推,然后将所有的等式相加,得
    \[ D_n = (x-a)^{n-1}D_1 + (n-1)a(x-a)^{n-1} = (x-a)^{n-1}[x +(n-1)a] \]
\end{solution}
\subsection{加边法}
\begin{solution}
    \[
        D_n =
        \begin{vmatrix}
            1 & a      & a      & a      & \cdots & a      \\
            0 & x      & a      & a      & \cdots & a      \\
            0 & a      & x      & a      & \cdots & a      \\
            0 & \vdots & \vdots & \vdots & \ddots & \vdots \\
            0 & a      & a      & a      & \cdots & x
        \end{vmatrix}
        =
        \begin{vmatrix}
            1  & a      & a      & a      & \cdots & a      \\
            -1 & x-a    & 0      & 0      & \cdots & 0      \\
            -1 & 0      & x-a    & 0      & \cdots & 0      \\
            -1 & \vdots & \vdots & \vdots & \ddots & \vdots \\
            -1 & 0      & 0      & 0      & \cdots & x-a
        \end{vmatrix}
    \]
    当$x\neq a$时,有
    \begin{align*}
        D_n & = \frac{1}{x-a}
        \begin{vmatrix}
            x-a & a      & a      & a      & \cdots & a      \\
            a-x & x-a    & 0      & 0      & \cdots & 0      \\
            a-x & 0      & x-a    & 0      & \cdots & 0      \\
            a-x & \vdots & \vdots & \vdots & \ddots & \vdots \\
            a-x & 0      & 0      & 0      & \cdots & x-a
        \end{vmatrix}             \\
            & = \frac{1}{x-a}
        \begin{vmatrix}
            x+(n-1)a & a      & a      & a      & \cdots & a      \\
            0        & x-a    & 0      & 0      & \cdots & 0      \\
            0        & 0      & x-a    & 0      & \cdots & 0      \\
            0        & \vdots & \vdots & \vdots & \ddots & \vdots \\
            0        & 0      & 0      & 0      & \cdots & x-a
        \end{vmatrix}             \\
            & = \frac{1}{x-a} [x+(n-1)](x-a)^n \\
            & = [x+(n-1)](x-a)^{n-1}
    \end{align*}
    当$x=a$时,$D_n=0$且$[x+(n-1)](x-a)^{n-1}=0$

    综上
    $D_n = [x+(n-1)](x-a)^{n-1}$
\end{solution}