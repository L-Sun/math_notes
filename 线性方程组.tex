\part{线性方程组}
\section{线性方程组的基本概念}
称
\[
    \begin{cases}
        a_{11}x_1 + a_{12}x_2 + \cdots + a_{1n}x_n = 0, \\
        a_{21}x_1 + a_{22}x_2 + \cdots + a_{2n}x_n = 0, \\
        \cdots                                          \\
        a_{m1}x_1 + a_{m2}x_2 + \cdots + a_{mn}x_n = 0
    \end{cases}
\]
为齐次线性方程组,记为$\mat{A}_{m\times n}\mat{X}=\bm{0}$,其中
\[
    \mat{A} =
    \begin{pmatrix}
        a_{11} & a_{12} & \cdots & a_{1n} \\
        a_{21} & a_{22} & \cdots & a_{2n} \\
        \vdots & \vdots & \ddots & \vdots \\
        a_{m1} & a_{m2} & \cdots & a_{mn}
    \end{pmatrix},
    \qquad
    \mat{X}=
    \begin{pmatrix}
        x_1 \\x_2\\\vdots\\x_n
    \end{pmatrix}
\]

称
\[
    \begin{cases}
        a_{11}x_1 + a_{12}x_2 + \cdots + a_{1n}x_n = b_1, \\
        a_{21}x_1 + a_{22}x_2 + \cdots + a_{2n}x_n = b_2, \\
        \cdots                                            \\
        a_{m1}x_1 + a_{m2}x_2 + \cdots + a_{mn}x_n = b_m
    \end{cases}
\]
为非齐次线性方程组,简记为$\mat{A}_{m\times n}\mat{X}=\bm{b}$,其中
\[
    \mat{A} =
    \begin{pmatrix}
        a_{11} & a_{12} & \cdots & a_{1n} \\
        a_{21} & a_{22} & \cdots & a_{2n} \\
        \vdots & \vdots & \ddots & \vdots \\
        a_{m1} & a_{m2} & \cdots & a_{mn}
    \end{pmatrix},
    \qquad
    \mat{X}=
    \begin{pmatrix}
        x_1 \\x_2\\\vdots\\x_m
    \end{pmatrix},
    \qquad
    \bm{b}=
    \begin{pmatrix}
        b_1 \\b_2\\\vdots\\b_m
    \end{pmatrix},
    \qquad
    \overline{\mat{A}}=
    \left(\begin{array}{cccc|c}
            a_{11} & a_{12} & \cdots & a_{1n} & b_1    \\
            a_{21} & a_{22} & \cdots & a_{2n} & b_2    \\
            \vdots & \vdots & \ddots & \vdots & \vdots \\
            a_{m1} & a_{m2} & \cdots & a_{mn} & b_m
        \end{array}\right)
\]
$\overline{\mat{A}}$为增广矩阵。

\section{齐次线性方程组的有解判定定理}
齐次线性方程组的解只有两种可能:只有零解、有无穷多解。
\begin{theorem}
    齐次线性方程组$\mat{A}_{m\times n}\mat{X}=\bm{0}$有非零解$\iff r(\mat{A})<n \iff A$的列向量组线性相关。
\end{theorem}
作为推论有
\begin{theorem}
    齐次线性方程组$\mat{A}_{m\times n}\mat{X}=\bm{0}$中
    \begin{enumerate}[(1)]
        \item 当$m<n$时,$\mat{AX}=\bm{0}$必有非零解;
        \item 当$m=n$时,$\mat{AX}=\bm{0}$有非零解$\iff |\mat{A}|=0$
    \end{enumerate}
\end{theorem}

\begin{theorem}
    如果$r(\mat{A})=r<n$,则$\mat{AX}=\bm{0}$有$n-r$个线性无关解$\bm{\eta}_1,\bm{\eta}_2,\cdots,\bm{\eta}_{n-r}$称为基础解系,
    且$\mat{AX}=\bm{0}$的任意一个解$\bm{\eta}$一定能由$\bm{\eta}_1,\bm{\eta}_2,\cdots,\bm{\eta}_{n-r}$线性表出。即通解为
    \[ k_1\bm{\eta}_1 + k_2\bm{\eta}_2 + \cdots + k_{n-r}\bm{\eta}_{n-r} \]
    其中$k_1,k_2,\cdots,k_n$为任意常数。
\end{theorem}

基础解系的定义如下
\begin{definition}
    如果$\bm{\eta}_1,\bm{\eta}_2,\cdots,\bm{\eta}_t$是$\mat{AX}=\bm{0}$的解,且满足
    \begin{enumerate}[(1)]
        \item $\bm{\eta}_1,\bm{\eta}_2,\cdots,\bm{\eta}_t$线性无关;
        \item $\mat{AX}=\bm{0}$的任意一个解都可以由$\bm{\eta}_1,\bm{\eta}_2,\cdots,\bm{\eta}_t$线性表出。
    \end{enumerate}
    则称$\bm{\eta}_1,\bm{\eta}_2,\cdots,\bm{\eta}_t$是$\mat{AX}=\bm{0}$的一个基础解系,且$t=n-r(\mat{A})$
\end{definition}

\begin{example}
    解方程组
    \[
        \systeme{
            x_1 - 2x_2 + x_3 + 2x_4 = 0 ,
            x_1 + x_2 - 2x_3 - 4x_4 = 0 ,
            x_1 - 5x_2 + 4x_3 + 8x_4 = 0
        }
    \]
\end{example}
\begin{solution}
    对系数矩阵作初等行变换
    \[
        \begin{pmatrix}
            1 & -2 & 1  & 2  \\
            1 & 1  & -2 & -4 \\
            1 & -5 & 4  & 8  \\
        \end{pmatrix}
        \longrightarrow
        \begin{pmatrix}
            1 & -2 & 1  & 2  \\
            0 & 3  & -3 & -6 \\
            0 & -3 & 3  & 6  \\
        \end{pmatrix}
        \longrightarrow
        \begin{pmatrix}
            1 & -2 & 1  & 2  \\
            0 & 1  & -1 & -2 \\
            0 & 0  & 0  & 0  \\
        \end{pmatrix}
        \longrightarrow
        \begin{pmatrix}
            1 & 0 & -1 & -2 \\
            0 & 1 & -1 & -2 \\
            0 & 0 & 0  & 0  \\
        \end{pmatrix}
    \]
    因此方程组变为
    \[
        \systeme{
            x_1 - x_3 - 2x_4 = 0,
            x_2 - x_3 - 2x_4 = 0
        }
    \]
    \begin{enumerate}[itemindent=1em,label=\textbf{\textsf{方法}}\arabic*]
        \item 令自由变量$x_3=t,x_4=u$,则有
              \[
                  \systeme{
                      t+2u = x_1,
                      t+2u = x_2,
                      t = x_3,
                      u = x_4
                  }
                  \text{或~}
                  x = t
                  \begin{pmatrix}
                      1 \\1\\1\\0
                  \end{pmatrix}
                  + u
                  \begin{pmatrix}
                      2 \\2\\0\\1
                  \end{pmatrix}
              \]
        \item 令$x_3=1,x_4=0$得解$\bm{\eta}_1 = (1,1,1,0)^\intercal$;令$x_3=0,x_4=1$得解$\bm{\eta}_2=(2,2,0,1)^\intercal$
              所以方程组的解为
              \[
                  x= k_1\bm{\eta}_1+k_2\bm{\eta}_2 =
                  k_1
                  \begin{pmatrix}
                      1 \\1\\1\\0
                  \end{pmatrix}
                  +
                  k_2
                  \begin{pmatrix}
                      2 \\2\\0\\1
                  \end{pmatrix}
              \]
    \end{enumerate}
\end{solution}

\section{非齐次线性方程组的有解判断}
对于非齐次线性方程组,它的解有三种可能:无解、唯一解、无穷多解。
\begin{theorem}
    且三种有解判断如下:
    \begin{enumerate}[(1)]
        \item 有解$\iff r(\mat{A})=r(\overline{\mat{A}})$,反过来,无解$\iff r(\mat{A})\neq r(\overline{\mat{A}}) \iff r(\mat{A})+1 = r(\overline{\mat{A}})$;
        \item 唯一解$\iff r(\mat{A})=r(\overline{\mat{A}}) = n$;
        \item 无穷多解$\iff r(\mat{A})=r(\overline{\mat{A}})<n$。
    \end{enumerate}
\end{theorem}

当$\mat{AX}=\bm{b}$有无穷多解时,引入\textbf{\textsf{解的结构}}概念。
\begin{theorem}
    设$\bm{\alpha}$是方程组$\mat{AX}=\bm{b}$的解,$\bm{\eta}_1,\bm{\eta}_2,\cdots,\bm{\eta}_t$是导出组$\mat{AX}=\bm{0}$的基础解系,则方程组$\mat{AX}=\bm{b}$的通解为
    \[ \bm{\alpha} + k_1\bm{\eta}_1 + k_2\bm{\eta}_2 + \cdots + k_t\bm{\eta}_t \]
    其中$k_1,k_2,\cdots,k_t$为任意常数。
\end{theorem}

\begin{theorem}
    (解的性质)
    \begin{enumerate}[(1)]
        \item 方程组$\mat{AX}=\bm{b}$的两个解$\bm{\gamma}_1,\bm{\gamma}_2$的差$\bm{\gamma}_1 - \bm{\gamma}_2$是$\mat{AX}=\bm{0}$的解;
        \item 方程组$\mat{AX}=\bm{b}$的一个解$\bm{\gamma}_0$与$\mat{AX}=\bm{0}$的一个解$\bm{\eta}$之和$\bm{\gamma}_0+\bm{\eta}$仍是$\mat{AX}=\bm{b}$的解。
    \end{enumerate}
\end{theorem}

\begin{example}
    解方程组
    \[
        \systeme{
            x_1 +  x_2 +  x_3 = 5,
            3x_1 + 2x_2 +  x_3 = 13,
            x_2 + 2x_3 =2
        }
    \]
\end{example}
\begin{solution}
    对增广矩阵作初等行变换
    \[
        \overline{\mat{A}}=
        \left(\begin{array}{ccc|c}
                1 & 1 & 1 & 5  \\
                3 & 2 & 1 & 13 \\
                0 & 1 & 2 & 2
            \end{array}\right)
        \longrightarrow
        \left(\begin{array}{ccc|c}
                1 & 1  & 1  & 5  \\
                0 & -1 & -2 & -2 \\
                0 & 1  & 2  & 2
            \end{array}\right)
        \longrightarrow
        \left(\begin{array}{ccc|c}
                1 & 1  & 1  & 5  \\
                0 & -1 & -2 & -2 \\
                0 & 0  & 0  & 0
            \end{array}\right)
        \longrightarrow
        \left(\begin{array}{ccc|c}
                1 & 0 & -1 & 3 \\
                0 & 1 & 2  & 2 \\
                0 & 0 & 0  & 0
            \end{array}\right)
    \]
    因此方程组变为
    \[
        \systeme{
            x_1 - x_3 = 3,
            x_2 + 2x_3 = 2
        }
    \]
    所以自由变量为$x_3$,方程组有无穷个解,则
    \[
        \begin{cases}
            x_1 = 3+t  \\
            x_2 = 2-2t \\
            x_3 = t
        \end{cases}
        \text{或~}
        x =
        \begin{pmatrix}
            3 \\2\\0
        \end{pmatrix}
        +t
        \begin{pmatrix}
            1 \\-2\\1
        \end{pmatrix}
    \]
    对于上个题的方法二,可以令自由变量$x_3=0$,则得特解$\bm{\alpha}=(3,2,0)^\intercal$。

    再令化简后的方程组$\mat{AX}=\bm{b}$变为齐次方程组$\mat{AX}=\bm{0}$然后计算齐次方程组的解向量,即令$x_3=1$,得$\bm{\eta}=(1,-2)^\intercal$,
    所以基础解系为
    \[ x= \bm{\alpha} + k\bm{\eta} \]
\end{solution}

\section{方程组的应用}
利用方程组求矩阵
\begin{example}
    \[
        \begin{pmatrix}
            1 & 2 & 3 \\
            2 & 3 & 4
        \end{pmatrix}
        \mat{X}
        =
        \begin{pmatrix}
            4 & 5 \\
            5 & 6
        \end{pmatrix}
    \]
    求$\mat{X}$
\end{example}
\begin{solution}
    显然$\mat{X}$是$3\times 2$的矩阵。同时$\mat{AX}=\mat{B}$中\mat{$A}$不可逆,所以要解方程组
    \[
        \begin{pmatrix}
            1 & 2 & 3 \\
            2 & 3 & 4
        \end{pmatrix}
        \begin{pmatrix}
            x_1 \\ x_2 \\ x_3
        \end{pmatrix}
        =
        \begin{pmatrix}
            4 \\5
        \end{pmatrix},\qquad
        \begin{pmatrix}
            1 & 2 & 3 \\
            2 & 3 & 4
        \end{pmatrix}
        \begin{pmatrix}
            x_4 \\ x_5 \\ x_6
        \end{pmatrix}
        =
        \begin{pmatrix}
            5 \\6
        \end{pmatrix}
    \]
    那么有增广矩阵
    \[
        \left(\begin{array}{ccc|c}
                1 & 2 & 3 & 4 \\
                2 & 3 & 4 & 5
            \end{array}\right)
        ,\qquad
        \left(\begin{array}{ccc|c}
                1 & 2 & 3 & 5 \\
                2 & 3 & 4 & 6
            \end{array}\right)
    \]
    为了简便,这里可以直接合并两个增广矩阵,此时就是$(A|B)$
    \[
        \left(\begin{array}{ccc|cc}
                1 & 2 & 3 & 4 & 5 \\
                2 & 3 & 4 & 5 & 6
            \end{array}\right)
        \longrightarrow
        \left(\begin{array}{ccc|cc}
                1 & 2  & 3  & 4  & 5  \\
                0 & -1 & -2 & -3 & -4
            \end{array}\right)
        \longrightarrow
        \left(\begin{array}{ccc|cc}
                1 & 2 & 3 & 4 & 5 \\
                0 & 1 & 2 & 3 & 4
            \end{array}\right)
        \longrightarrow
        \left(\begin{array}{ccc|cc}
                1 & 0 & -1 & -2 & -3 \\
                0 & 1 & 2  & 3  & 4
            \end{array}\right)
    \]
    设$x_3=s,x_6=t$因此可得
    \[
        \begin{pmatrix}
            x_1 \\ x_2 \\ x_3
        \end{pmatrix}
        =
        \begin{pmatrix}
            -2+s \\ 3-2s \\ s
        \end{pmatrix}
        ,\qquad
        \begin{pmatrix}
            x_4 \\ x_5 \\ x_6
        \end{pmatrix}
        =
        \begin{pmatrix}
            -3+t \\ 4-2t \\ t
        \end{pmatrix}
    \]
    因此
    \[
        \mat{X} =
        \begin{pmatrix}
            -2+s & -3+t \\
            3-2s & 4-2t \\
            s    & t
        \end{pmatrix}
    \]
    其中$s,t$为任意常数。
\end{solution}
对于$\mat{A},\mat{B}$都是$n$阶矩阵时,上面这种方法也可以使用,而且比求$\mat{A}^{-1}$要简单,这是因为
\[ \mat{A}^{-1}(\mat{A}|\mat{B}) = (\mat{E}|\mat{A}^{-1}\mat{B}) \]
这样就简化了$\mat{A}^{-1}\mat{B}$的矩阵乘法,其中$\mat{A}^{-1}$为一系列的初等行变换。

\begin{example}
    求一个齐次线性方程组,使其基础解系为
    \[ \bm{\eta}_1 = (4,3,1,2)^\intercal,\qquad \bm{\eta}_2 = (0,1,3,-2)^\intercal \]
\end{example}
\begin{solution}
    由于是求齐次方程组$\mat{AX}=\bm{0}$,其实也就是求矩阵$\mat{A}$,那么根据解向量的维数,可知$\mat{A}$矩阵的列数为$4$,同时基础解系只有两个向量,那么有
    $n-r=2$,因此$r(\mat{A}) = 4-2 = 2$,不妨设$\mat{A}$为$2\times 4$的矩阵,则有
    \[ \mat{A}\bm{\eta}_1 = \bm{0}, \qquad \mat{A}\bm{\eta}_2 = \bm{0} \]
    为了简便,将$\bm{\eta}_1,\bm{\eta}_2$合为一个矩阵,那么有
    \[
        \begin{pmatrix}
            a_{11} & a_{12} & a_{13} & a_{14} \\
            a_{21} & a_{22} & a_{23} & a_{24} \\
        \end{pmatrix}
        \begin{pmatrix}
            4 & 0  \\
            3 & 1  \\
            1 & 3  \\
            2 & -2
        \end{pmatrix}
        =\mat{0}
    \]
    利用转置可得
    \[
        \begin{pmatrix}
            4 & 3 & 1 & 2  \\
            0 & 1 & 3 & -2
        \end{pmatrix}
        \begin{pmatrix}
            a_{11} & a_{21} \\
            a_{12} & a_{22} \\
            a_{13} & a_{23} \\
            a_{14} & a_{24} \\
        \end{pmatrix}
        =\mat{0}
    \]
    对其进行初等行变换
    \[
        \begin{pmatrix}
            4 & 3 & 1 & 2  \\
            0 & 1 & 3 & -2
        \end{pmatrix}
        \longrightarrow
        \begin{pmatrix}
            4 & 0 & -8 & 8  \\
            0 & 1 & 3  & -2
        \end{pmatrix}
        \longrightarrow
        \begin{pmatrix}
            1 & 0 & -2 & 2  \\
            0 & 1 & 3  & -2
        \end{pmatrix}
    \]
    那么有
    \[
        \systeme{
            x_1 -2x_3 + 2x_4 = 0,
            x_2 + 3x_3 - 2x_4 = 0
        }
    \]
    令$x_3 = s, x_4=t$,则得通解为
    \[
        x =
        \begin{pmatrix}
            2s - 2t \\
            -3s+2t  \\
            s       \\
            t
        \end{pmatrix}
    \]
    分别令$s=0,t=1$和$s=1,t=0$得一个矩阵$\mat{A}$
    \[
        \mat{A} =
        \begin{pmatrix}
            -2 & 2  & 0 & 1 \\
            2  & -3 & 1 & 0
        \end{pmatrix}
    \]
\end{solution}