\part{导数}
\section{基本概念}
.
\begin{definition}
    若函数$f(x)$在某点$x_0$的领域内有定义,并且有
    \[ f'(x) = \lim_{x \to x_0} \frac{f(x)-f(x_0)}{x-x_0} \]
    则函数$f(x)$在$x_0$处可导。
\end{definition}

\begin{theorem}
    函数$f(x)$在$x_0$处可导且$f(x_0) = A \iff$函数$f(x)$在$x_0$点单侧可导
    且$f'_-(x_0)=f'_+(x_0)=A$
\end{theorem}


\section{求导法则}
导数的四则运算如下
\begin{theorem}
    设函数$u(x), v(x)$均在$(a,b)$内可导,则函数$u(x), v(x)$的和、差、积、商(分母不为零)均在$(a,b)$内可导,并且有
    \begin{align}
        \left[u(x) \pm v(x) \right]'      & = u'(x) \pm v'(x)                      \\
        \left[ u(x) \cdot v(x) \right]'   & = u'(x)v(x) + u(x)v'(x)                \\
        \left[ \frac{u(x)}{v(x)} \right]' & = \frac{u'(x)v(x) - u(x)v'(x)}{v^2(x)}
    \end{align}
\end{theorem}

\begin{theorem}
    (反函数求导法则)
    \label{th:反函数求导法则}
    设函数$\varphi (y)$可导,如果$\varphi'(y) \neq 0$,则反函数$y=f(x)$可导.
    \[ f'(x) = \dv{y}{x} = \left. 1 \middle / \dv{x}{y}\right. = \frac{1}{\varphi'(y)} = \frac{1}{\varphi'(f(x))} \]
\end{theorem}
复合函数求导如下
\begin{theorem}
    (链式法则)
    \label{th:链式法则}
    设函数$y=f(u)$与函数$u=\varphi(x)$均可导,则符合函数$y=f(\varphi(x))$在其定义域内可导,并且有$(f\circ\varphi)'(x)=f'(\varphi(x))\cdot\varphi'(x)$即
    \[ \dv{y}{x} = \dv{y}{u} \cdot \dv{u}{x} \]
\end{theorem}

\subsection{高阶导数}
在求高阶导数的通式和值时,主要使用Leibiz公式,对于不是很好求的函数时,可对部分项进行泰勒展开。
\begin{theorem}
    (Leibniz公式)
    \label{th:Leibniz公式}
    设$u=u(x),v=v(x)$在$x$点处有$n$阶导数,则
    \[ (uv)^{(n)} = \sum^n_{k=0}C^k_n u^{(n-k)}v^{(k)} \]
\end{theorem}

\begin{example}
    已知$y=x^2\sin 2x$,求$y^{(50)}$
\end{example}
\begin{solution}
    根据Leibniz公式\ref{th:Leibniz公式},可得
    \begin{align*}
        y^{(50)} & = \sum_{n=0}^{50} C_{50}^n (x^2)^{(n)}\cdot(\sin 2x)^{(50-n)}                                                       \\
                 & = C_{50}^0(x^2)\cdot(\sin 2x)^{(50)} + C_{50}^1(x^2)'\cdot(\sin 2x)^{(49)} + C_{50}^2(x^2)''\cdot(\sin 2x)^{(48)}+0 \\
                 & = 2^{50}\cdot \left[ x^2 \sin(2x+25\pi) + 50x\sin(2x+\frac{49\pi}{2})+\frac{25 \cdot 49}{2}\sin(2x+24\pi) \right]   \\
                 & = 2^{50}\cdot\left[ -x^2\sin 2x + 50x\cos 2x + \frac{1225}{2}\sin 2x \right]
    \end{align*}
\end{solution}

\begin{example}
    已知
    \begin{math}
        y =
        \begin{cases}
            \frac{\sin x}{x}, & x \neq 0 \\
            1,                & x = 0
        \end{cases}
    \end{math}
    求$y^{(n)}(0)$
\end{example}
\begin{solution}
    当$x\neq 0$时,由泰勒展开可得
    \begin{align*}
        y & = \frac{1}{x}(x - \frac{x^3}{3!} + \frac{x^5}{5!} - \cdots) \\
          & = 1-\frac{x^2}{3!}+\frac{x^4}{5!}-\cdots                    \\
          & =\sum_{k=0}^\infty \frac{(-1)^{k}x^{2k}}{(2k+1)!}
    \end{align*}
    同时有
    \[ \lim_{x \to 0} \sum_{k=0}^\infty \frac{(-1)^{k}x^{2k}}{(2k+1)!} = 1 \]
    所以对于$(-\infty,+\infty)$,泰勒展开恒成立。
    又由于
    \[ y = \sum_{n=0}^{\infty}\frac{y^{(n)}}{n!}x^{n} \]
    比较系数可得
    \[
        y^{(n)}(0) =
        \begin{cases}
            0,                      & n = 2k+1, \\
            \dfrac{(-1)^{k}}{2k+1}, & n=2k
        \end{cases}
    \]
\end{solution}

\section{微分中值定理}
微分中值定理属于重要且必考的定理,常出现于证明题中。
\begin{theorem}
    (罗尔定理)
    \label{th:罗尔定理}
    设函数$f(x)$在区间$[a,b]$上连续,$(a,b)$内可导,
    且$f(a)=f(b)$,则有
    \[ f'(\xi) = 0 \qquad \xi \in (a,b)\]
\end{theorem}

\begin{theorem}
    (拉格朗日中值定理)
    \label{th:拉格朗日中值定理}
    设函数$f(x)$在区间$[a,b]$上连续,$(a,b)$内可导,则有
    \[ \frac{f(b)-f(a)}{b-a} = f'(\xi) \qquad \xi \in (a,b)\]
\end{theorem}

\begin{theorem}
    (柯西中值定理)
    \label{th:柯西中值定理}
    设函数$f(x),g(x)$在区间$[a,b]$上连续,$(a,b)$内可导,且$g'(x)\neq0$恒成立,则有
    \[ \frac{f(b)-f(a)}{g(b)-g(a)} = \frac{f'(\xi)}{g'(\xi)} \qquad \xi \in (a,b)\]
\end{theorem}

\begin{example}
    设$f(x)$在$[a,b]$上连续,$f''(x)$在$(a,b)$存在。连接点$(a,f(a)),(b,f(b))$交曲线$y=f(x)$于点$(c,f(c))$,且$a<c<b$,
    证明存在$\xi\in(a,b)$使得$f''(\xi)=0$.
\end{example}
\begin{proof}
    根据拉格朗日中值定理得
    \begin{align*}
        \frac{f(c)-f(a)}{c-a} & = f'(\xi_1) & \xi_1 \in (a,c) \\
        \frac{f(b)-f(c)}{b-c} & = f'(\xi_2) & \xi_2 \in (c,b)
    \end{align*}
    而由于点$(a,f(a)),(c,f(c)),(b,f(b))$共线,所以有
    \[ f'(\xi_1) = f'(\xi_2) \qquad \xi_1,\xi_2 \in (a,b) \]
    根据罗尔定理可知
    \[ f''(\xi) = 0  \qquad \xi \in (\xi_1,\xi_2) \subseteq (a,b) \]
\end{proof}


\begin{example}
    设函数$f(x)$在区间$[a,b]$上连续,$(a,b)$内可导,且$f'(x)\neq 0$,试证存在$\xi,\eta \in (a,b)$使得
    \[ \frac{f'(\xi)}{f'(\eta)} = \frac{\mathrm{e}^b-\mathrm{e}^a}{b-a}\mathrm{e}^{-\eta} \]
\end{example}

\begin{proof}
    根据柯西中值定理可得
    \[ \frac{f(b)-f(a)}{\mathrm{e}^b-\mathrm{e}^a} = \frac{f'(\eta)}{\mathrm{e}^\eta} \qquad \eta \in (a,b) \]
    根据拉格朗日中值定理得
    \[ \frac{f(b)-f(a)}{b-a} = f'(\xi) \qquad \xi \in (a,b) \]
    因此有
    \[ f(b)-f(a) = \left( \mathrm{e}^b - \mathrm{e}^a \right)f'(\eta)\mathrm{e}^{-\eta}=(b-a)f'(\xi)  \qquad \xi,\eta \in (a,b) \]
    变形得
    \[ \frac{f'(\xi)}{f'(\eta)} = \frac{\mathrm{e}^b-\mathrm{e}^a}{b-a}\mathrm{e}^{-\eta} \qquad \xi,\eta \in (a,b) \]
\end{proof}

\begin{example}
    设函数$f(x)$在$[a,b]$上连续,在$(a,b)$内可导,且$f(a)=f(b)=0$,求证:
    \begin{enumerate}[(1)]
        \item 存在$\xi\in(a,b)$,使$f(\xi)+\xi f'(\xi)=0$;
        \item 存在$\eta\in(a,b)$,使$\eta f(\eta) + f'(\eta)=0$;
    \end{enumerate}
\end{example}
\begin{proof}
    \begin{enumerate}[(1)]
        \item 设$g(x)=xf(x)$,则由罗尔定理得,存在$\xi\in(a,b)$使得
              \[ g'(\xi)=f(\xi)+\xi f'(\xi)=0 \]
        \item 设$h(x)=\mathrm{e}^{\frac{x^2}{2}}f(x)$,则由罗尔定理得,存在$\eta\in(a,b)$使得
              \[ h'(\eta) = \mathrm{e}^{\frac{\eta^2}{2}}(\eta f(\eta)+f'(\eta)) = 0 \]
              即
              \[ \eta f(\eta) + f'(\eta)=0 \]
    \end{enumerate}
\end{proof}
常见的构造函数如下:
\begin{alignat*}{2}
     & g(x)=xf(x),                           & \qquad & g'(x)=f(x)+xf'(x)                                  \\
     & g(x)=\dfrac{f(x)}{x},                 & \qquad & g'(x) = \dfrac{xf'(x)-f(x)}{x^2}, (x\neq 0)        \\
     & g(x)=\mathrm{e}^{m(x)}f(x),           & \qquad & g'(x) = \mathrm{e}^{m(x)}(m'(x)f(x)+f'(x))         \\
     & g(x)=\dfrac{f(x)}{\mathrm{e}^{m(x)}}, & \qquad & g'(x) = \dfrac{f'(x)-m'(x)f(x)}{\mathrm{e}^{m(x)}} \\
\end{alignat*}
前两个的$f(x)$系数为$\pm1$,后两个则是$f'(x)$的系数为$1$。

当无法猜出$g(x)$时,可以考虑$\frac{m'(x)}{n'(x)}$,根据柯西中值定理来推出$m(x)$和$n(x)$。

\begin{example}
    设函数$f(x)$在$[a,b]$上连续,在$(a,b)$内可导,且$f(a)\neq f(b)$,求证:$\exists \xi,\eta \in (a,b)$使得
    \[ \frac{f'(\xi)}{2\xi} = \frac{f'(\eta)}{b+a} \]
\end{example}
\begin{proof}
    因为
    \[ \frac{f'(\xi)}{2\xi}, \qquad \frac{f'(\eta)}{b+a} = \frac{f'(\eta)(b-a)}{b^2-a^2} \]
    所以可以猜测需要柯西中值定理以及涉及两个函数$f(x)$和$x^2$,所以有
    \[ \frac{f(b)-f(a)}{b^2-a^2} = \frac{f'(\xi)}{2\xi}, \qquad \xi\in(a,b) \]
    注意到
    \[ \frac{f(b)-f(a)}{b^2-a^2} = \frac{f(b)-f(a)}{(b-a)(b+a)} = \frac{f'(\eta)}{b+a}, \qquad \eta\in(a,b)\]
    所以有
    \[ \frac{f'(\xi)}{2\xi} = \frac{f'(\eta)}{b+a}, \qquad \xi,\eta\in(a,b) \]
\end{proof}


\section{泰勒展开}
泰勒公式不仅可以用于求解极限,还能证明与导数相关的题目。
\begin{example}
    设$f(x)$在$x_0$处$n$阶可导,且$f^{(m)}(x)=0,(m=1,2,\cdots,n-1),f^{(n)}(x)\neq 0(n\geq 2)$,证明:
    \begin{enumerate}[(1)]
        \item 当$n$为偶数且$f^{(n)}(x_0)<0$时,$f(x)$在$x_0$取得极小值;
        \item 当$n$为偶数且$f^{(n)}(x_0)>0$时,$f(x)$在$x_0$取得极大值;
    \end{enumerate}
\end{example}
\begin{proof}
    根据泰勒公式可知:
    \begin{align*}
        f(x) & = f(x_0) + \sum_{k=1}^{n-1} \frac{f^{(k)}(x_0)}{k!}(x-x_0)^k + \frac{f^{(n)}(\xi)}{n!}(x-x_0)^n \\
             & = f(x_0) +  \frac{f^{(n)}(\xi)}{n!}(x-x_0)^n
    \end{align*}
    其中$\xi$位于$x_0$的去心领域内。

    当$n$为偶数且$f^{(n)}(x_0)<0$时,$x_0$的去心领域内有,$f(x) \leq f(x_0)$,则$f(x)$在$x_0$取得极小值;

    当$n$为偶数且$f^{(n)}(x_0)>0$时,$x_0$的去心领域内有,$f(x) \geq f(x_0)$,则$f(x)$在$x_0$取得极大值。
\end{proof}

\begin{example}
    若函数$\varphi(x)$及$\psi(x)$是$n$阶可微的,且$\varphi^{(k)}(x_0)=\psi^{(k)}(x_0),k=0,1,2,\cdots,n-1$,又$x>x_0$时,
    $\varphi^{(n)}(x)>\psi^{(n)}(x)$。试证:当$x>x_0$时,$\varphi(x)>\psi(x)$
\end{example}
\begin{proof}
    根据泰勒公式可知
    \[ \varphi(x) = \sum_{k=0}^{n-1}\frac{\varphi^{(k)}(x_0)}{k!}(x-x_0)^k + \frac{\varphi^{(n)}(\xi_1)}{n!}(x-x_0)^n \]
    \[ \psi(x) = \sum_{k=0}^{n-1}\frac{\psi^{(k)}(x_0)}{k!}(x-x_0)^k + \frac{\psi^{(n)}(\xi_2)}{n!}(x-x_0)^n \]
    当$x>x_0$时,$\xi_1,\xi_2\in(x_0,x)$,此时由上述两个等式可得
    \[ \varphi(x) - \psi(x) = \frac{(x-x_0)^n}{n!}[\varphi^{(n)}(\xi_1) - \psi^{(n)}(\xi_2)] > 0\]
    因此当$x>x_0$时,$\varphi(x)>\psi(x)$
\end{proof}

\section{导数的应用}
\subsection{极值求解}
极值的定义如下
\begin{definition}
    极大值与极小值(统称极值)是指在一个域上函数取得最大值(或最小值)的点的函数值。而使函数取得极值的点(的横坐标)被称作极值点。这个域既可以是一个邻域,又可以是整个函数域(这时极值称为最值)
    若某点连续且左右两侧的单调性相异,则该点为极值点。
\end{definition}
\begin{theorem}
    当$f'(x_0)=0$时
    \begin{enumerate}
        \item 当$f(x)$在$x_0$连续,且$f'_-(x_0)<0,f'_+(x_0)>0$时, $x_0$为极小值点
        \item 当$f(x)$在$x_0$连续,且$f'_-(x_0)>0,f'_+(x_0)<0$时, $x_0$为极大值点
        \item $f''(x_0)>0,\ x_0$为极小值点
        \item $f''(x_0)<0,\ x_0$为极大值点
    \end{enumerate}
\end{theorem}

\subsection{零点的求解}
对于函数$f(x)$,零点的存在性由介值定理得出。零点位置、个数要根据函数的变化(单调性,水平渐近线)来进行求解。
\begin{theorem}
    介值定理
    \label{th:介值定理}
    若函数$f(x)$在$[a,b]$上连续,且$f(a)<f(b)$,若$\forall u$满足$f(a)<u<f(b)$,则至少存在一点$x=c$且$a<c<b$,使得$f(c)=u$,$f(a)>f(b)$同理。
\end{theorem}

\begin{example}
    讨论方程$ax\mathrm{e}^x+b=0,(a>0)$的实根情况。
\end{example}
\begin{marginfigure}
    \begin{tikzpicture}
        \begin{axis}[
                ticks=none,
                axis x line=none,
                axis y line=middle,
                ymax=5,
                ymin=1,
                xmin=-10,
                xmax=10,
                domain=-10:10,
                xlabel=$x$,
                ylabel=$y$,
                scale only axis,
                width=\textwidth
            ]
            \addplot[CarnationPink,smooth,domain=-10:1] {exp(1)*x*exp(x)+3} node [right,pos=0.65,scale=0.7] {$ax\mathrm{e}^x+b$};
            \addplot[Gray,dashed,name path = L1] {3.05};
            \addplot[Gray,name path = L2] {2};
            \path[name path=YMax] (-10,5) -- (10,5);
            \path[name path=YMin] (-10,1) -- (10,1);
            \addplot [thick,color=red,fill=red,fill opacity=0.05]fill between[of=YMax and L1];
            \addplot [thick,color=blue,fill=blue,fill opacity=0.05]fill between[of=L1 and L2];
            \addplot [thick,color=green,fill=green,fill opacity=0.05]fill between[of=L2 and YMin];
            \node [Gray, scale=0.7] at (8,1.5) {无实根};
            \node [Gray, scale=0.7] at (8,2.5) {两个实根};
            \node [Gray, scale=0.7] at (8,4) {一个实根};
            \node [Gray, scale=0.7] at (8,2) {一个实根};
        \end{axis}
    \end{tikzpicture}
\end{marginfigure}
\begin{solution}
    令$f(x) = ax\mathrm{e}^x+b$,则有
    \[
        \lim_{x\to-\infty}f(x)
        = a\lim_{x\to-\infty} x\mathrm{e}^x + b
        = a\lim_{x\to-\infty} \frac{x}{\mathrm{e}^{-x}} + b
        = a\lim_{x\to-\infty} \frac{1}{-\mathrm{e}^{-x}} + b
        = b
    \]
    \[
        \lim_{x\to+\infty}f(x) = +\infty
    \]
    对函数$f'(x)=a\mathrm{e}^x(1+x)=0$进行求解,得$x=-1$,则函数得单调性如下
    \begin{center}
        \begin{tabular}{|c|c|c|c|}
            \hline
            $x$     & $(-\infty,-1)$ & $-1$                      & $(-1,+\infty)$ \\ \hline
            $f'(x)$ & $-$            & $0$                       & $+$            \\ \hline
            $f(x)$  & $\searrow$     & $-\frac{a}{\mathrm{e}}+b$ & $\nearrow$     \\ \hline
        \end{tabular}
    \end{center}
    所以
    \begin{enumerate}
        \item 当$0 < -\frac{a}{\mathrm{e}}+b$时,即$b\in(\frac{a}{\mathrm{e}},+\infty)$,方程无实根。
        \item 当$0 = -\frac{a}{\mathrm{e}}+b$或$0 \geq b$时,即$b\in(\infty,0]\cup \left\{ -\frac{a}{\mathrm{e}} \right\}$,方程有一实根。
        \item 当$-\frac{a}{\mathrm{e}}+b < 0 < b$时,即$b\in(0,\frac{a}{\mathrm{e}})$,方程有两个不同的实根。
    \end{enumerate}
\end{solution}

对于求$f(x)$的极值步骤如下
\begin{enumerate}
    \item 确定定义域
    \item 计算导函数$f'(x)$
    \item 求解方程$f'(x)=0$(根可能为临界点)
    \item 确定可能的极值点(临界点、\textcolor{red}{不可导点})
    \item 按照极值的充分性判断极值点
    \item 计算极值点的函数值,得出极值
\end{enumerate}

\begin{example}
    求函数$y=(x^2-1)^{2/3}$的极值。
\end{example}
\begin{solution}
    由
    \[ y'=\frac{4}{3}\frac{x}{\sqrt[3]{x^2-1}}=0 \]
    解得临界点$x=0$和不可导点$x=\pm 1$
    因此可得如下情况:
    \begin{center}
        \begin{tabular}{|c|c|c|c|c|c|c|c|}
            \hline
            $x$  & $(-\infty,1)$ & $-1$     & $(-1,0)$   & $0$      & $(0,1)$    & $1$    & $(1,+\infty)$ \\ \hline
            $y'$ & $-$           & 不存在   & $+$        & $0$      & $-$        & 不存在 & $+$           \\ \hline
            $y$  & $\searrow$    & $\smile$ & $\nearrow$ & $\frown$ & $\searrow$ & 极小值 & $\nearrow$    \\ \hline
        \end{tabular}
    \end{center}
    故函数$y=(x^2-1)^{2/3}$有极大值$\eval{y}_{x=0}=1$,有极小值$\eval{y}_{x=\pm 1}=0$
\end{solution}

\subsection{函数的凹凸性}
当函数弦在上,曲在下,函数称为凹函数(反之则为凸函数)。以数学的语言,即如下不等式成立时,函数为凹函数
\begin{definition}
    设函数$f(x)$在区间$(a,b)$内有定义,如果$x_1,x_2\in(a,b), t \in [0,1]$时,恒有
    \[ f((1-t)x_1+tx_2)\leq (1-t)f(x_1)+tf(x_2) \]
    则函数$f(x)$在区间$(a,b)$为凹函数。
\end{definition}
换句话说,组合的像小于等于像的组合时,函数为凹函数

\begin{theorem}
    设函数$f(x)$在区间$[a,b]$上连续,在$(a,b)$内二阶可导,则
    \begin{enumerate}
        \item 当$f''(x)>0$于$(a,b)$时,$f(x)$是$[a,b]$上的严格凹函数
        \item 当$f''(x)<0$于$(a,b)$时,$f(x)$是$[a,b]$上的严格凸函数
        \item 若该函数在某点的二阶导数为零或不存在,且二阶导数在该点两侧符号相反,该点即为函数的拐点。
    \end{enumerate}
\end{theorem}

\subsection{函数的曲率}
曲率的几何概念可由右图给出
\begin{marginfigure}
    \centering
    \begin{tikzpicture}
        \begin{axis}[
                ticks=none,
                disabledatascaling,
                axis lines=middle,
                xmin=0,
                xmax=8,
                ymin=0,
                ymax=8,
                domain=0:8,
                xlabel=$x$,
                ylabel=$y$,
                scale only axis,
                unit vector ratio=1 1,
                width=\textwidth
            ]
            \pgfmathsetmacro{\cx}{4}
            \pgfmathsetmacro{\cy}{4}
            \pgfmathsetmacro{\r}{3}
            \pgfmathsetmacro{\a}{30}
            \pgfmathsetmacro{\da}{30}

            \coordinate (C) at ({\cx},{\cy});
            \coordinate (A) at ({\r*sin(\a)+\cx},{\cy-\r*cos(\a)});
            \coordinate (B) at ({\r*sin(\a+\da)+\cx},{\cy-\r*cos(\a+\da)});
            \node at (C) [above right] {$\textcolor{red}{\Delta\alpha}$};
            \draw [name path=circle] (C) circle [radius={\r}];
            \draw (A) -- (C) node [left,pos=0.3] {$R$} -- (B) pic [draw=Gray] {angle=A--C--B};
            \addplot [name path =L1, Gray] {tan(\a) * (x-\r*sin(\a)-\cx) + \cy - \r*cos(\a)};
            \addplot [name path =L2, Gray] {tan(\a+\da) * (x-\r*sin(\a+\da)-\cx) + \cy - \r*cos(\a+\da)};
            \path [name intersections={of=L1 and L2}];
            \coordinate (D) at (intersection-1);
            \coordinate (E) at (8, {tan(\a) * (8-\r*sin(\a)-\cx) + \cy - \r*cos(\a)});
            \pic [draw=Gray,"$\textcolor{red}{\Delta\alpha}$", angle radius=15, angle eccentricity=1.5] {angle=E--D--B};
            \coordinate (X) at (8,0);
            \path [name path=XAxis] (0,0)--(X);
            \path [name intersections={of=L1 and XAxis}];
            \coordinate (F) at (intersection-1);
            \path [name intersections={of=L2 and XAxis}];
            \coordinate (G) at (intersection-1);
            \pic [draw=Gray,"$\alpha$", angle radius=15, angle eccentricity=1.5] {angle=X--F--D};
            \pic [draw=Gray,"$\alpha+\textcolor{red}{\Delta\alpha}$", angle radius=15, angle eccentricity=2] {angle=X--G--D};
            \draw [very thick, blue] (A) arc[radius={\r}, start angle={\a-90}, end angle={\a+\da-90}] node [left, pos=0.8] {$\Delta s$};
        \end{axis}
    \end{tikzpicture}
    \caption{某段曲线的曲率变化}
\end{marginfigure}

\begin{definition}
    (曲率$K$和曲率半径$R$)
    \begin{align*}
        R & = \lim_{\Delta\alpha\to 0} \abs{\frac{\Delta s}{\Delta \alpha}} = \abs{\dv{s}{\alpha}} &                 \\
        \\
        K & = \frac{1}{R} = \abs{\dv{\alpha}{s}}                                                   & \text{定义式}   \\
          & =\frac{\abs{y''}}{[1+y'^2]^{3/2}}                                                      & \text{函数形式} \\
          & =\frac{\abs{x'y''-x''y'}}{[x'^2+y'^2]^{3/2}}                                           & \text{参数形式}
    \end{align*}
\end{definition}

\begin{proof}
    如果曲线$L$的方程$y=f(x)$具有二阶导数,则
    \[ \alpha = \arctan y',\dd{\alpha}=\frac{y''}{1+y'^2}\dd{x} \]
    根据弧长公式$\dd{s}=\sqrt{\dd{x^2} + \dd{y^2}} = \sqrt{1+y'^2}\dd{x}$,可得
    \[ K = \frac{\abs{y''}}{[1+y'^2]^{3/2}} \]
\end{proof}

\begin{example}
    已知$f(x)$二阶可导,且$f(x)>0,f(x)f''(x)-[f'(x)]^2 \geq 0 (x\in \mathbf{R})$,
    证明
    \[f(x_1)f(x_2)\geq f^2\left(\frac{x_1+x_2}{2}\right)\ (x_1,x_2\in \mathbf{R})\];
\end{example}

\begin{proof}
    令$g(x)=\ln f(x)$,则
    \[g''(x) = \frac{f(x)f''(x)-[f'(x)]^2}{f^2(x)} \geq 0 \]
    所以$g(x)$为凹函数,所以有
    \[ \frac{1}{2}[g(x_1)+g(x_2)] \geq g\left(\frac{x_1+x_2}{2}\right) \]
    即
    \[f(x_1)f(x_2)\geq f^2\left(\frac{x_1+x_2}{2}\right)\]
\end{proof}


\subsection{渐近线}
渐近线分为\textcolor{red}{水平渐近线}、\textcolor{red}{垂直渐近线}、\textcolor{red}{斜渐近线}。如下图
\begin{figure}[htbp]
    \centering
    \begin{subfigure}[b]{0.3\linewidth}
        \begin{tikzpicture}
            \begin{axis}[
                    ticks=none,
                    axis lines=middle,
                    ymin=-pi,
                    ymax=pi,
                    xlabel=$x$,
                    ylabel=$y$,
                    scale only axis,
                    width=\textwidth
                ]
                \addplot[CarnationPink, smooth] {rad(atan(x))};
                \addplot[Gray, dashed] {pi/2} node [midway, left, text=black] {$\frac{\pi}{2}$};
                \addplot[Gray, dashed] {-pi/2} node [midway, left, text=black] {$-\frac{\pi}{2}$};
            \end{axis}
        \end{tikzpicture}
        \subcaption{$\textcolor{CarnationPink}{\arctan x}$的水平渐近线}
    \end{subfigure}
    \begin{subfigure}[b]{0.3\linewidth}
        \begin{tikzpicture}
            \begin{axis}[
                    ticks=none,
                    axis lines=middle,
                    ymin=-4.5,ymax=4.5,
                    xmin=-pi,xmax=pi,
                    xlabel=$x$,
                    ylabel=$y$,
                    scale only axis,
                    width=\textwidth
                ]
                \addplot[CarnationPink, smooth, domain=-pi/2:pi/2] {tan(deg(x))};
                \addplot[CarnationPink, smooth, domain=-pi:-pi/2] {tan(deg(x))};
                \addplot[CarnationPink, smooth, domain=pi/2:pi] {tan(deg(x))};
                \addplot[Gray, dashed, variable=\y, domain=-4.5:4.5] (pi/2, {\y}) node [midway, below left,text=black] {$\frac{\pi}{2}$};
                \addplot[Gray, dashed, variable=\y, domain=-4.5:4.5] (-pi/2, {\y}) node [midway, below left,text=black] {$-\frac{\pi}{2}$};
            \end{axis}
        \end{tikzpicture}
        \subcaption{$\textcolor{CarnationPink}{\tan x}$的垂直渐近线}
    \end{subfigure}
    \begin{subfigure}[b]{0.3\linewidth}
        \begin{tikzpicture}
            \begin{axis}[
                    ticks=none,
                    axis lines=middle,
                    ymin=-pi,
                    ymax=2*pi,
                    xlabel=$x$,
                    ylabel=$y$,
                    scale only axis,
                    unit vector ratio=1 1,
                    width=\textwidth
                ]
                \addplot[CarnationPink, smooth] {x+rad(90-atan(x))};
                \addplot[Gray, dashed] {x} node [below right, pos=0.7, text=black] {$y=x$};
                \addplot[Gray, dashed] {x+pi} node [above left, pos=0.5, text=black] {$y=x+\pi$};
            \end{axis}
        \end{tikzpicture}
        \subcaption{$\textcolor{CarnationPink}{x + \arccot x}$的斜渐近线}
    \end{subfigure}
    \caption{三种渐近线}
\end{figure}

寻找渐进线的步骤为:
\begin{enumerate}
    \item 确定函数的间断点,判断间断点处是否存在垂直渐近线。
    \item 当函数在$x\to\infty$、$x\to-\infty$、$x\to+\infty$存在极限值时,函数存在水平渐近线(此时对应的$x$则没有斜渐近线)。
    \item 若$x\to\infty$、$x\to-\infty$、$x\to+\infty$时下式成立,则斜渐近线存在。
          \sidenote{其中函数曲线到斜渐近线的距离为\[d(x) = \dfrac{\abs{f(x)-kx-b}}{\sqrt{1+k^2}} \]}
          \[ \frac{f(x)}{x}\to k, \qquad f(x)-kx\to b \]
\end{enumerate}
