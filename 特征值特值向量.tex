\part{特征值、特值向量}
特征值、特征向量无法避开的两个等式如下
\begin{enumerate}[(1)]
    \item $\mat{A}\bm{\alpha} = \lambda\bm{\alpha}, \bm{\alpha} \neq \bm{0}$;
    \item $|\lambda \mat{E} - \mat{A}| = 0$;
    \item $(\lambda_i \mat{E} - \mat{A} )\mat{X} = \bm{0}$;
\end{enumerate}

以及相似矩阵的概念$\mat{A}\sim \mat{B} : \exists \text{可逆矩阵}\mat{P}, \mat{P}^{-1}\mat{A}\mat{P} = \mat{B}$,其中的重点是$\mat{A}$是否相似于对角矩阵。

还有实对称矩阵的概念及相关定理。

\section{特征值、特值向量}
.
\begin{definition}
    设$\mat{A}$是$n$阶矩阵,$\bm{\alpha}$是$n$维非零列向量,且
    \[ \mat{A}\bm{\alpha} = \lambda\bm{\alpha} \]
    则称$\lambda$是矩阵$\mat{A}$的特征值,$\bm{\alpha}$是矩阵$\mat{A}$是对应于特征值$\lambda$的特征向量。
\end{definition}
特征值、特征向量的求法主要有定义法、行列式法、基础解系法。


设$n$阶矩阵$\mat{A}$,非零列向量$\bm{\alpha}$,则有等式
\[ \mat{A}\bm{\alpha} = \lambda\bm{\alpha} \implies (\lambda \mat{E} - \mat{A})\bm{\alpha} = \bm{0} \]

\paragraph{行列式法}
这里同齐次线性方程组联系起来,由于$\bm{\alpha}\neq 0$,所以方程组有非零解,故有行列式法
\begin{equation}
    |\lambda \mat{E}- \mat{A}| = 0
\end{equation}
称之为特征多项式。根据代数基本定理,特征多项式一定有$n$个解(允许有重根、复数),即$\mat{A}$有$n$个特征值。

\paragraph{基础解系法}
另一方面,行列式法得到的特征值$\lambda_i$,带入到等式中,即
\[  (\lambda_i \mat{E} -\mat{A})\bm{\alpha} = \bm{0} \]
求其基础解系,则可以得到特征值$\lambda_i$对应的一组线性无关的特征向量。
然后得出特征向量的通解(即通过基础解系得出通解)。

\begin{theorem}
    设$\bm{\alpha}_1,\bm{\alpha}_2,\cdots,\bm{\alpha}_t$都是矩阵$\mat{A}$对应于特征值$\lambda$的特征向量,
    那么当$\bm{\alpha} = k_1\bm{\alpha}_1+k_2\bm{\alpha}_2 + \cdots + k_t\bm{\alpha}_t \neq 0$时,$\bm{\alpha}$仍是$\mat{A}$对应于特征值$\lambda$的特征向量。
\end{theorem}

\begin{theorem}
    如果$\lambda_1,\lambda_2,\cdots,\lambda_m$时矩阵$\mat{A}$互不相同的特征值,对应的特征向量分别为$\bm{\alpha}_1,\bm{\alpha}_2,\cdots,\bm{\alpha}_m$,
    则$\bm{\alpha}_1,\bm{\alpha}_2,\cdots,\bm{\alpha}_m$\textcolor{red}{线性无关}。
\end{theorem}
这个定理要注意的是,仅仅是线性无关,而不是正交。(正交是实对称矩阵的性质。)

\begin{theorem}
    设$\mat{A}$为$n$矩阵,特征值为$\lambda_1,\lambda_2,\cdots,\lambda_n$,则
    \begin{enumerate}
        \item $\sum_i^n \lambda_i = \sum_i^n a_{ii}$
        \item $\prod_i^n \lambda_i = |\mat{A}|$
    \end{enumerate}
\end{theorem}

\begin{example}
    求下面矩阵的特征值,特征向量。
    \[
        \mat{A} =
        \begin{pmatrix}
            17 & -2 & -2 \\
            -2 & 14 & -4 \\
            -2 & -4 & 14
        \end{pmatrix}
    \]
\end{example}
\begin{solution}
    有$\mat{A}$的特征多项式
    \begin{align*}
        |\lambda \mat{E} - \mat{A}| & =
        \begin{vmatrix}
            \lambda-17 & 2            & 2           \\
            2          & \lambda - 14 & 4           \\
            2          & 4            & \lambda -14
        \end{vmatrix}
        =
        \begin{vmatrix}
            \lambda-17  & 2            & 2           \\
            36-2\lambda & \lambda - 18 & 0           \\
            36-2\lambda & 0            & \lambda -18
        \end{vmatrix}
        =
        \begin{vmatrix}
            \lambda-9 & 2            & 2           \\
            0         & \lambda - 18 & 0           \\
            0         & 0            & \lambda -18
        \end{vmatrix}                                \\
                                    & = (\lambda-9)(\lambda-18)^2
    \end{align*}
    得特征值$\lambda_1 = 9, \lambda_2 = 18(\text{重根})$

    当$\lambda=9$时,齐次方程组
    \[
        (9\mat{E} - \mat{A})\mat{X} =
        \begin{pmatrix}
            -8 & 2  & 2  \\
            2  & -5 & 4  \\
            2  & 4  & -5
        \end{pmatrix}\mat{X}
        =\bm{0}
    \]
    则
    \[
        \begin{pmatrix}
            -8 & 2  & 2  \\
            2  & -5 & 4  \\
            2  & 4  & -5
        \end{pmatrix}
        \longrightarrow
        \begin{pmatrix}
            4 & -1  & -1  \\
            4 & -10 & 8   \\
            4 & 8   & -10
        \end{pmatrix}
        \longrightarrow
        \begin{pmatrix}
            4 & -1 & -1 \\
            0 & -9 & 9  \\
            0 & 9  & -9
        \end{pmatrix}
        \longrightarrow
        \begin{pmatrix}
            4 & -1 & -1 \\
            0 & 1  & -1 \\
            0 & 0  & 0
        \end{pmatrix}
        \longrightarrow
        \begin{pmatrix}
            1 & 0 & -\frac{1}{2} \\
            0 & 1 & -1           \\
            0 & 0 & 0
        \end{pmatrix}
    \]
    所以$x_3$是自由变量,令$x_3=1$得$x_1=\frac{1}{2},x_2=1$
    那么基础解系为
    \[
        \bm{\alpha} = (\frac{1}{2},1,1)^\intercal
    \]
    因此$\lambda = 9$时对应得特征向量为
    \[ t\bm{\alpha} (t\neq 0) \]
    当$\lambda=18$时,齐次方程组
    \[
        (18\mat{E} - \mat{A})\mat{X} =
        \begin{pmatrix}
            1 & 2 & 2 \\
            2 & 4 & 4 \\
            2 & 4 & 4
        \end{pmatrix}\mat{X}
        =\bm{0}
    \]
    则
    \[
        \begin{pmatrix}
            1 & 2 & 2 \\
            2 & 4 & 4 \\
            2 & 4 & 4
        \end{pmatrix}
        \longrightarrow
        \begin{pmatrix}
            1 & 2 & 2 \\
            0 & 0 & 0 \\
            0 & 0 & 0
        \end{pmatrix}
    \]
    所以$x_2,x_3$是自由变量,令$x_2=1,x_3=0$得$x_1=-2$,令$x_2=0,x_3=1$得$x_1=-2$
    那么基础解系为
    \[
        \bm{\alpha}_1 = (-2,1,0)^\intercal,\qquad \bm{\alpha}_2 = (-2,0,1)^\intercal
    \]
    因此$\lambda = 18$时对应得特征向量为
    \[ k_1\bm{\beta}_1 + k_2\bm{\beta}_2 (k_1,k_2\text{不全为}0) \]
\end{solution}

特征向量更快的解法由下面给出
\begin{solution}
    当$\lambda=9$时,
    \[
        (9\mat{E}-\mat{A}) =
        \begin{pmatrix}
            -8 & 2  & 2  \\
            2  & -5 & 4  \\
            2  & 4  & -5
        \end{pmatrix}
    \]
    由于$|\lambda \mat{E} - \mat{A}|=0$,所以它的秩$r<3$,这里可以\textcolor{red}{明显}看出$9\mat{E}-\mat{A}$的秩是$2$(若无法明显看出,那么按原始方法作,以减少不必要的麻烦),
    所以一定能经过初等行变换变成两行非零的矩阵,由于$9\mat{E}-\mat{A}$的任何两行都不成比例(找出两行不成比例即可,剩余的变为$0$),
    所以直接让某一行变为全$0$,这是因为这一行能被另外两行表示。因此
    \[
        \begin{pmatrix}
            -8 & 2  & 2  \\
            2  & -5 & 4  \\
            2  & 4  & -5
        \end{pmatrix}
        \longrightarrow
        \begin{pmatrix}
            2 & -5 & 4  \\
            2 & 4  & -5 \\
            0 & 0  & 0
        \end{pmatrix}
        \longrightarrow
        \begin{pmatrix}
            2 & -5 & 4  \\
            0 & 9  & -9 \\
            0 & 0  & 0
        \end{pmatrix}
        \longrightarrow
        \begin{pmatrix}
            2 & 0 & -1 \\
            0 & 1 & -1 \\
            0 & 0 & 0
        \end{pmatrix}
        \longrightarrow
        \begin{pmatrix}
            1 & 0 & -\frac{1}{2} \\
            0 & 1 & -1           \\
            0 & 0 & 0
        \end{pmatrix}
    \]
    那么其基础解系为$\bm{\alpha} = (\frac{1}{2},1,1)^\intercal$
    所以对应的特征向量为$t\bm{\alpha} (t\neq 0)$
\end{solution}

\begin{example}
    求
    \[
        \mat{A} =
        \begin{pmatrix}
            -1 & 1 & 0 \\
            -4 & 3 & 0 \\
            1  & 0 & 2
        \end{pmatrix}
    \]
    的特征值、特征向量。
\end{example}
\begin{solution}
    特征多项式
    \[
        |\lambda \mat{E} - \mat{A}|
        =
        \begin{vmatrix}
            \lambda+1 & -1        & 0         \\
            4         & \lambda-3 & 0         \\
            -1        & 0         & \lambda-2
        \end{vmatrix}
        =(\lambda-2)
        \begin{vmatrix}
            \lambda+1 & -1        \\
            4         & \lambda-3
        \end{vmatrix}
        = (\lambda-2)(\lambda-1)^2
    \]
    因此特征值为$2,1,1$,
    当$\lambda = 2$时,
    \[
        \lambda \mat{E} - \mat{A} =
        \begin{pmatrix}
            3  & -1 & 0 \\
            4  & -1 & 0 \\
            -1 & 0  & 0
        \end{pmatrix}
        \longrightarrow
        \begin{pmatrix}
            1 & 0  & 0 \\
            3 & -1 & 0 \\
            0 & 0  & 0
        \end{pmatrix}
        \longrightarrow
        \begin{pmatrix}
            1 & 0 & 0 \\
            0 & 1 & 0 \\
            0 & 0 & 0
        \end{pmatrix}
    \]
    则$(2\mat{E}-\mat{A})\mat{X}=\bm{0}$的基础解系为$\bm{\alpha} = (0,0,1)^\intercal$
    因此对应的特征向量为$k\bm{\alpha} (k\neq 0)$

    当$\lambda = 1$时
    \[
        \lambda \mat{E} - \mat{A} =
        \begin{pmatrix}
            2  & -1 & 0  \\
            4  & -2 & 0  \\
            -1 & 0  & -1
        \end{pmatrix}
        \longrightarrow
        \begin{pmatrix}
            1 & 0  & 1 \\
            2 & -1 & 0 \\
            0 & 0  & 0
        \end{pmatrix}
        \longrightarrow
        \begin{pmatrix}
            1 & 0  & 1  \\
            0 & -1 & -2 \\
            0 & 0  & 0
        \end{pmatrix}
        \longrightarrow
        \begin{pmatrix}
            1 & 0 & 1 \\
            0 & 1 & 2 \\
            0 & 0 & 0
        \end{pmatrix}
    \]
    所以$(\mat{E}-\mat{A})\mat{X}=\bm{0}$基础解系为$\bm{\beta}=(-1,-2,1)^\intercal$,则对应的特征向量为$t\bm{\beta} (t\neq 0)$
\end{solution}

注意到这两个例题中,\textcolor{red}{前者重根的特征向量的自由变量有两个,而后者重根的特征向量的自由变量只有一个。}
这个区别关系到矩阵与对角矩阵相似的问题。

根据定义,可以得到下面这个公式
\begin{equation}
    (\mat{A}+k\mat{E})\bm{\alpha} = (\lambda + k)\bm{\alpha}
\end{equation}
这表明$(\mat{A}+k\mat{E})$和$\mat{A}$的特征向量相同,特征值相差$k$

还有下面这个公式
\begin{equation}
    \mat{A}^n\bm{\alpha} = \lambda^n\bm{\alpha}
\end{equation}

\begin{example}
    已知$\bm{\alpha}=(1,1,-1)^\intercal$是
    \[
        \mat{A} =
        \begin{pmatrix}
            2  & -1 & 2  \\
            5  & a  & 3  \\
            -1 & b  & -2
        \end{pmatrix}
    \]
    的一个特征向量,求$a,b$
\end{example}
\begin{solution}
    由于$\bm{\alpha}$是$\mat{A}$的特征向量,不妨设$\lambda$为$\bm{\alpha}$对应的特征值,那么有
    \[ \mat{A}\bm{\alpha} = \lambda\bm{\alpha} \]
    即
    \[
        \begin{pmatrix}
            2  & -1 & 2  \\
            5  & a  & 3  \\
            -1 & b  & -2
        \end{pmatrix}
        \begin{pmatrix}
            1 \\1\\-1
        \end{pmatrix}
        =
        \begin{pmatrix}
            -1 \\ 2+a \\ 1+b
        \end{pmatrix}
        =
        \begin{pmatrix}
            \lambda \\ \lambda \\ -\lambda
        \end{pmatrix}
    \]
    因此有
    \[
        \begin{cases}
            \lambda = -1 \\
            a = -3       \\
            b=0
        \end{cases}
    \]
\end{solution}

\subsection{特征值个数与秩的关系}
考虑$n$阶矩阵$\mat{A}$,根据定义可知特征值满足下面等式
\[ \mat{A}\bm{\alpha} = \lambda\bm{\alpha} \]
同时,由初等变换可知存在$n$阶可逆矩阵$P,Q$使得
\[
    \mat{PAQ} =
    \begin{pmatrix}
        \mat{E_r} & \mat{0} \\
        \mat{0}   & \mat{0}
    \end{pmatrix}
\]
其中$r\leq n$,将等式变形可得
\[
    \mat{A} =
    \mat{P}^{-1}
    \begin{pmatrix}
        \mat{E_r} & \mat{0} \\
        \mat{0}   & \mat{0}
    \end{pmatrix}
    \mat{Q}^{-1}
    =
    \mat{P}^{-1}
    \begin{pmatrix}
        \mat{E_r} \\ \mat{0}
    \end{pmatrix}
    \begin{pmatrix}
        \mat{E_r} & \mat{0}
    \end{pmatrix}
    \mat{Q}^{-1}
    =\mat{BC}
\]
其中$\mat{B}=\mat{P}^{-1}\begin{pmatrix}\mat{E_r} & \mat{0}\end{pmatrix}^\intercal, \mat{C} = \begin{pmatrix}\mat{E_r} & \mat{0}\end{pmatrix}\mat{Q}^{-1}$,
则$\mat{B}$为列满秩矩阵,$\mat{C}$为行满秩矩阵,且$r(\mat{A}) = r(\mat{B}) = r(\mat{C}) = r$。

设$\lambda$为$\mat{A}$的任意一非零特征值,$\bm{\alpha}$为对应的特征向量,则有
\[ \mat{A}\bm{\alpha} = \mat{BC}\bm{\alpha} = \lambda\bm{\alpha} \]
两边同时乘以矩阵$\mat{C}$,则有
\[ \mat{CBC}\bm{\alpha} = \mat{CB}(\mat{C}\bm{\alpha}) = \lambda(\mat{C}\bm{\alpha}) \]
其中$\mat{C}\bm{\alpha}\neq \bm{0}$,(若$\mat{C}\bm{\alpha} = \bm{0}$,则有$\mat{BC}\bm{\alpha} = \bm{0} = \lambda\bm{\alpha} = \bm{0}$,又$\lambda\neq 0$,则$\bm{\alpha}=\bm{0}$,与$\bm{\alpha}\neq \bm{0}$矛盾),
因此矩阵$\mat{CB}$特征值为$\lambda$,对应特征向量为$\mat{C}\bm{\alpha}$。所以$\mat{BC}$的任意一非零特征值为$\mat{CB}$的特征值。同理可证$\mat{CB}$的任意一非零特征值也为$\mat{BC}$的特征值。
因此$\mat{BC},\mat{CB}$的非零特征值相同。又因为特征值个数不超过矩阵的阶数,所以非零特征值个数不超过$\mat{CB}$的阶数$r$,即$\mat{A}=\mat{BC}$的非零特征值不超过$r$。

综上可得,
\begin{theorem}
    设矩$n$阶矩阵$\mat{A}$,且$r(\mat{A})=r$,则$\mat{A}$的非零特征值个数小于等于$r$。
\end{theorem}
进一步的,有
\begin{enumerate}[(1)]
    \item 若$\mat{A}$为可对角化矩阵,根据$\mat{A}\sim \operatorname{diag}(\lambda_1,\lambda_2,\cdots)$,知非零特征值个数等于矩阵的秩;
    \item 若$\mat{A}$为满秩矩阵,根据$|\mat{A}|=\prod \lambda \neq 0$,知$\lambda\neq 0$,则非零特征值等于$n$;
    \item 若$\mat{A}$为不可对角化的降秩矩阵,则非零特征值个数小于等于它的秩。
\end{enumerate}

\begin{example}
    设$\mat{A}=\mat{E}+\bm{\alpha}\bm{\beta}^\intercal$,其中$\bm{\alpha},\bm{\beta}$均为$n$阶非零列向量,且有$\bm{\alpha}^\intercal\bm{\beta}=3$,求$|\mat{A}+2\mat{E}|$
\end{example}
\begin{solution}
    要求$|\mat{A}+2\mat{E}|$,则要考虑$\mat{A}$的特征值,根据特征值的乘积求出行列式。又$\mat{A}=\mat{E}+\bm{\alpha}\bm{\beta}^\intercal$,因此转为考虑$\bm{\alpha}\bm{\beta}^\intercal$的特征值。
    注意到
    \[ r(\bm{\alpha}\bm{\beta}^\intercal) \leq \min(r(\bm{\alpha}),r(\bm{\beta}^\intercal)) = 1 \]
    且$\bm{\alpha}\neq \bm{0},\bm{\beta}\neq \bm{0}$,则有$\bm{\alpha}\bm{\beta}^\intercal\neq 0$,因此$r(\bm{\alpha}\bm{\beta}^\intercal)=1$,所以$\bm{\alpha}\bm{\beta}^\intercal$非零特征值不超过$1$。

    假设非零特征值个数为$0$,则特征值为$0$的个数等于$n$,又因为$\Tr(\bm{\alpha}\bm{\beta}^\intercal)=\bm{\beta}^\intercal\bm{\alpha} = 3\neq 0$,所以非零特征值个数不等于$0$;

    因此非零特征值个数为$1$,此时特征值为$0$的个数等于$n-1$,即$0$为$n-1$重特征值。根据$\Tr(\bm{\alpha}\bm{\beta}^\intercal)=3$,得出$3$为单重特征值。综上可得
    \[ |\mat{A}+2\mat{E}| = |3\mat{E} + \bm{\alpha}\bm{\beta}^\intercal| = \prod_i^n (3+\lambda_i) = 6\cdot 3^{n-1} = 2\cdot 3^n \]

\end{solution}

\section{相似矩阵}
\subsection{相似矩阵的基本概念和性质}
.
\begin{definition}
    设$\mat{A},\mat{B}$都是$n$阶矩阵,如果存在可逆矩阵$\mat{P}$,使得
    \[ \mat{P}^{-1}\mat{A}\mat{P} = \mat{B} \]
    就称矩阵$\mat{A}$相似于矩阵$\mat{B}$,$\mat{B}$是$\mat{A}$的相似矩阵,记为$\mat{A}\sim \mat{B}$
\end{definition}
这里注意与等价概念的区别,等价是$\mat{Q\mat{A}\mat{P}}=\mat{B}$,其中$\mat{Q},\mat{P}$均为可逆矩阵,但$\mat{Q},\mat{P}$可无关系。

下面介绍几个相似的基本性质
\begin{enumerate}[(1)]
    \item (自反性)$\mat{A}\sim \mat{A}$;
    \item (对称性)如果$\mat{A}\sim \mat{B}$则$\mat{B} \sim \mat{A}$;
    \item (传递性)如果$\mat{A}\sim \mat{B}, \mat{B} \sim \mat{C}$,则$\mat{A}\sim \mat{C}$;
    \item (特征值相同)$\mat{A}\sim \mat{B} \implies \lambda_{\mat{A}} = \lambda_{\mat{B}}$;
    \item (秩相等)$\mat{A}\sim \mat{B} \implies r(\mat{A})=r(\mat{B})$;
    \item (行列式相等)$\mat{A}\sim \mat{B} \implies |\mat{A}|=|\mat{B}|$;
    \item (迹相等)$\mat{A}\sim \mat{B} \implies \Tr(\mat{A}) = \Tr(\mat{B})$
\end{enumerate}

下面为相似的运算关系:假设$\mat{A}\sim \mat{B}$
\begin{enumerate}[(1)]
    \item $\mat{A}^2 \sim \mat{B}^2$;
    \item $\mat{A}+k\mat{E} \sim \mat{B} + k\mat{E}$;
    \item 如果$\mat{A}$可逆,则$\mat{A}^{-1}\sim \mat{B}^{-1}$
    \item 如果$\mat{A}_1\sim \mat{B}_1, \mat{A}_2\sim \mat{B}_2$,则
          $
              \begin{pmatrix}
                  \mat{A}_1 &           \\
                            & \mat{A}_2
              \end{pmatrix}
              \sim
              \begin{pmatrix}
                  \mat{B}_1 &           \\
                            & \mat{B}_2
              \end{pmatrix}
          $
\end{enumerate}

\begin{example}
    已知
    \[
        \mat{A} =
        \begin{pmatrix}
            2 & 0 & 0 \\
            0 & 0 & 1 \\
            0 & 1 & x
        \end{pmatrix}
        ,\qquad
        \mat{B} =
        \begin{pmatrix}
            2 & 0 & 0  \\
            0 & y & 0  \\
            0 & 0 & -1
        \end{pmatrix}
    \]
    相似,求$x,y$
\end{example}
\begin{solution}
    由于$\mat{A}\sim \mat{B}$,所以$\Tr(\mat{A})=\Tr(\mat{B}),|\mat{A}|=|\mat{B}|$,即
    \begin{align*}
        2+x & = 1+y \\
        -2  & = -2y
    \end{align*}
    解得$x=0,y=1$
\end{solution}

\subsection{相似对角化}
.
\begin{definition}
    如果$\mat{A}$和对角矩阵相似,则称矩阵$\mat{A}$可相似对角化。
\end{definition}

\begin{theorem}
    $\mat{A}$和对角矩阵相似$\iff \mat{A}$有$n$个线性无关特征向量。
\end{theorem}
作为推论有
\begin{theorem}
    $\mat{A}$有$n$个不同的特征值$\implies \mat{A}$可相似对角化。
\end{theorem}
这是因为不同的特征值对应的特征向量,一定线性无关。注意这里不是充要条件,这是因为特征值有重根的$\mat{A}$也是有可能相似对角化。

\begin{theorem}
    $\mat{A}$可相似对角化$\iff \lambda$是$\mat{A}$的$k$重根特征值,则$\lambda$有$k$个线性无关的特征向量。
\end{theorem}
简而言之,特征值相同的个数等于对应的特征向量的个数时,矩阵能相似对角化。

相似对角化的解题步骤:假设$3$阶矩阵$\mat{A}$可相似对角化
\begin{enumerate}[(1)]
    \item 求特征值$\lambda_1, \lambda_2, \lambda_3$;
    \item 求特征向量$\bm{\alpha}_1,\bm{\alpha}_2,\bm{\alpha}_3$;
    \item 构造可逆矩阵$P=\begin{pmatrix} \bm{\alpha}_1 & \bm{\alpha}_2& \bm{\alpha}_3 \end{pmatrix}$;
    \item 最后有对角化矩阵$\operatorname{diag}(\lambda_1,\lambda_2,\lambda_3)$
\end{enumerate}

利用相似对角化来求$\mat{A}^n$:
\[
    \mat{P}^{-1}\mat{A}\mat{P} = \mat{B} \implies    \mat{P}^{-1}\mat{A}^n\mat{P} = \mat{B}^n \implies \mat{A}^n = \mat{P}\mat{B}^n\mat{P}^{-1}
\]
其中$\mat{B}$为对角矩阵,因此有
\begin{equation}
    \mat{A}^n = \mat{P}\mat{B}^n\mat{P}^{-1} =
    \mat{P}
    \begin{pmatrix}
        b_1 &        &     \\
            & \ddots &     \\
            &        & b_m
    \end{pmatrix}^n\mat{P}^{-1}
    =\mat{P}
    \begin{pmatrix}
        b_1^n &        &       \\
              & \ddots &       \\
              &        & b_m^n
    \end{pmatrix}\mat{P}^{-1}
\end{equation}

\begin{example}
    已知
    \[
        \mat{A} =
        \begin{pmatrix}
            2 & 0 & 0 \\
            0 & 0 & 1 \\
            0 & 1 & x
        \end{pmatrix}
        ,\qquad
        \mat{B} =
        \begin{pmatrix}
            2 & 0 & 0  \\
            0 & y & 0  \\
            0 & 0 & -1
        \end{pmatrix}
    \]
    相似,求$x,y$,求可逆矩阵$\mat{P}$,使得$\mat{P}^{-1}\mat{A}\mat{P}=\mat{B}$
\end{example}
\begin{solution}
    此例题在相似矩阵部分中解得$x=0,y=1$。因此
    \[
        \mat{A} =
        \begin{pmatrix}
            2 & 0 & 0 \\
            0 & 0 & 1 \\
            0 & 1 & 0
        \end{pmatrix}
        ,\qquad
        \mat{B} =
        \begin{pmatrix}
            2 & 0 & 0  \\
            0 & 1 & 0  \\
            0 & 0 & -1
        \end{pmatrix}
    \]
    可以看出$\mat{B}$为对角矩阵,根据相似对角化的性质,可以得到$\mat{A}$的三个特征值$2,1,-1$,
    当$\lambda=2$时,$(2\mat{E}-\mat{A})\bm{\alpha}=0(\bm{\alpha}\neq 0)$,
    \[
        2\mat{E}-\mat{A} =
        \begin{pmatrix}
            0 & 0  & 0  \\
            0 & 2  & -1 \\
            0 & -1 & 0  \\
        \end{pmatrix}
        \longrightarrow
        \begin{pmatrix}
            0 & 1 & 0 \\
            0 & 0 & 1 \\
            0 & 0 & 0 \\
        \end{pmatrix}
    \]
    因此一个特征向量为$\bm{\alpha}_1 = (1,0,0)^\intercal$,

    当$\lambda = 1$时,
    \[
        \mat{E}-\mat{A} =
        \begin{pmatrix}
            -1 & 0  & 0  \\
            0  & 1  & -1 \\
            0  & -1 & 1  \\
        \end{pmatrix}
        \longrightarrow
        \begin{pmatrix}
            1 & 0 & 0  \\
            0 & 1 & -1 \\
            0 & 0 & 0  \\
        \end{pmatrix}
    \]
    因此一个特征向量为$\bm{\alpha}_2 = (0,1,1)^\intercal$,

    当$\lambda = -1$时,
    \[
        -\mat{E}-\mat{A} =
        \begin{pmatrix}
            -3 & 0  & 0  \\
            0  & -1 & -1 \\
            0  & -1 & -1 \\
        \end{pmatrix}
        \longrightarrow
        \begin{pmatrix}
            1 & 0 & 0 \\
            0 & 1 & 1 \\
            0 & 0 & 0 \\
        \end{pmatrix}
    \]
    因此一个特征向量为$\bm{\alpha}_3 = (0,-1,1)^\intercal$,
    那么
    \[
        \mat{P} =
        \begin{pmatrix}
            \bm{\alpha}_1 & \bm{\alpha}_2 & \bm{\alpha}_3
        \end{pmatrix}
        =
        \begin{pmatrix}
            1 & 0 & 0  \\
            0 & 1 & -1 \\
            0 & 1 & 1
        \end{pmatrix}
    \]
\end{solution}

\begin{example}
    若
    \[
        \mat{A} =
        \begin{pmatrix}
            1 & 3  & 0 \\
            1 & -1 & 0 \\
            a & -6 & 2
        \end{pmatrix}
    \]
    可相似对角化,求$a$,并求可逆矩阵$\mat{P}$使得$\mat{P}^{-1}\mat{A}\mat{P}=\mat{B}$,$\mat{B}$为对角化矩阵。
\end{example}
\begin{solution}
    $\mat{A}$的特征多项式为
    \[
        |\lambda \mat{E}-\mat{A}| =
        \begin{vmatrix}
            \lambda-1 & -3        & 0         \\
            -1        & \lambda+1 & 0         \\
            -a        & 6         & \lambda-2
        \end{vmatrix}
        =(\lambda-2)[(\lambda-1)(\lambda+1)-3]
        =(\lambda-2)^2(\lambda+2)
    \]
    因此特征值为$-2,2(\text{二重根})$由于$\mat{A}$可相似对角化,所以特征值为$2$的特征向量有$2$个。
    则
    \[
        2\mat{E}-\mat{A} =
        \begin{pmatrix}
            1  & -3 & 0 \\
            -1 & 3  & 0 \\
            -a & 6  & 0
        \end{pmatrix}
        \longrightarrow
        \begin{pmatrix}
            1  & -3 & 0 \\
            -a & 6  & 0 \\
            0  & 0  & 0
        \end{pmatrix}
        \longrightarrow
        \begin{pmatrix}
            1 & -3   & 0 \\
            0 & 6-3a & 0 \\
            0 & 0    & 0
        \end{pmatrix}
    \]
    自由变量有两个,所以$6-3a=0$,即$a=2$。此时$\lambda=2$的两个线性无关特征向量为
    \[ \bm{\alpha}_1 = (3,1,0)^\intercal ,\qquad \bm{\alpha}_2 = (0,0,1)^\intercal \]

    当$\lambda=-2$时,
    \[
        -2\mat{E}-\mat{A} =
        \begin{pmatrix}
            -3 & -3 & 0  \\
            -1 & -1 & 0  \\
            -a & 6  & -4
        \end{pmatrix}
        \longrightarrow
        \begin{pmatrix}
            1  & 1 & 0  \\
            -1 & 3 & -2 \\
            0  & 0 & 0
        \end{pmatrix}
        \longrightarrow
        \begin{pmatrix}
            1 & 1 & 0  \\
            0 & 4 & -2 \\
            0 & 0 & 0
        \end{pmatrix}
        \longrightarrow
        \begin{pmatrix}
            2 & 2 & 0  \\
            0 & 2 & -1 \\
            0 & 0 & 0
        \end{pmatrix}
        \longrightarrow
        \begin{pmatrix}
            2 & 0 & 1  \\
            0 & 2 & -1 \\
            0 & 0 & 0
        \end{pmatrix}
    \]
    所以$\lambda=-2$对应的一个特征向量为$\bm{\alpha}_3 = (-1,1,2)^\intercal$,
    因此矩阵
    \[
        \mat{P} =
        \begin{pmatrix}
            \bm{\alpha}_1 & \bm{\alpha}_2 & \bm{\alpha}_3
        \end{pmatrix}
        =
        \begin{pmatrix}
            3 & 0 & -1 \\
            1 & 0 & 1  \\
            0 & 1 & 2
        \end{pmatrix}
    \]
    对角矩阵为
    \[
        \begin{pmatrix}
            2 & 0 & 0  \\
            0 & 2 & 0  \\
            0 & 0 & -2
        \end{pmatrix}
    \]
\end{solution}

\section{实对称矩阵}
.
\begin{definition}
    对于$n$阶矩阵$\mat{A}$,如果有$\mat{A}^\intercal = \mat{A}$,那么称矩阵$\mat{A}$为实对称矩阵。
\end{definition}
\begin{theorem}
    实对称矩阵一定可以相似对角化。
\end{theorem}
\begin{theorem}
    实对称矩阵不同特征值所对应的特征向量\textcolor{red}{相互正交}。
\end{theorem}
这里相互正交,就会有内积为$0$,也就能引申到齐次方程组相关的知识点。

\begin{theorem}
    实对称矩阵一定存在正交矩阵\ref{sec:正交矩阵}$\mat{Q}$,使得
    \[ \mat{Q}^{-1}\mat{A}\mat{Q} = \mat{Q}^\intercal \mat{A}\mat{Q} = \mat{B} \]
    其中$\mat{B}$为对角化矩阵。
\end{theorem}
在求\textcolor{red}{实对称矩阵}的相似对角化的\textcolor{red}{正交矩阵$\mat{Q}$}时候,步骤在相似对角化的步骤上额外加上向量的正交标准化:
\begin{enumerate}[(1)]
    \item 求出特征值$\lambda_1,\lambda_2,\lambda_3$;
    \item 求出特征向量$\bm{\alpha}_1,\bm{\alpha}_2,\bm{\alpha}_3$;
    \item 由于实对称矩阵中,特征值不同的特征向量相互正交,故只需要单位化;
    \item 当特征值是重根时,如果特征向量已经正交,则只需单位化;
    \item 当特征向量是重根时,且特征向量不正交,需要施密特正交单位化;
    \item 得到新的特征向量为$\bm{\gamma}_1,\bm{\gamma}_2,\bm{\gamma}_3$,则正交矩阵$\mat{Q}=\begin{pmatrix}=\bm{\gamma}_1 & \bm{\gamma}_2 & \bm{\gamma}_3\end{pmatrix}$;
    \item 最后有
          \[
              \mat{Q}^{-1}\mat{A}\mat{Q} =
              \begin{pmatrix}
                  \lambda_1 &           &           \\
                            & \lambda_2 &           \\
                            &           & \lambda_3
              \end{pmatrix}
          \]
\end{enumerate}

\begin{example}
    设$3$阶矩阵$\mat{A}$为实对称矩阵,其特征值为$3,0,0$,且$\lambda=3$的特征向量$\bm{\alpha}_1 = (1,1,1)^\intercal$
    \begin{enumerate}[(1)]
        \item 求$\lambda=0$的特征向量;
        \item 求$\mat{A}$;
        \item 求正交矩阵$\mat{Q}$,使得$\mat{Q}^{-1}\mat{A}\mat{Q}$为对角化矩阵。
    \end{enumerate}
\end{example}
\begin{solution}
    \begin{enumerate}[(1)]
        \item 由于$\mat{A}$是实对称矩阵,设$\lambda=0$的两个特征向量为$\bm{\alpha}_2,\bm{\alpha}_3$,因此有
              \[ (\bm{\alpha}_1,\bm{\alpha}_2) = (\bm{\alpha}_1,\bm{\alpha}_3) = 0 \]
              即$\bm{\alpha}_2,\bm{\alpha}_3$为齐次方程组$\bm{\alpha}_1X=x_1+x_2+x_3=0$的非零解。

              令$x_2=1,x_3=0$得$\bm{\alpha}_2 =(-1,1,0)^\intercal$;
              令$x_2=0,x_3=1$得$\bm{\alpha}_3=(-1,0,1)^\intercal$;

              所以$\lambda=0$得特征向量为$k_2\bm{\alpha}_2+k_3\bm{\alpha}_3 (k_2,k_3\text{不全为}0)$

        \item 由于$\mat{A}$可相似对角化,故有
              \[
                  \mat{A}
                  \begin{pmatrix}
                      \bm{\alpha}_1 & \bm{\alpha}_2 & \bm{\alpha}_3
                  \end{pmatrix}
                  =
                  \begin{pmatrix}
                      \bm{\alpha}_1 & \bm{\alpha}_2 & \bm{\alpha}_3
                  \end{pmatrix}
                  \begin{pmatrix}
                      3 & 0 & 0 \\
                      0 & 0 & 0 \\
                      0 & 0 & 0
                  \end{pmatrix}
                  =
                  \begin{pmatrix}
                      3 & 0 & 0 \\
                      3 & 0 & 0 \\
                      3 & 0 & 0
                  \end{pmatrix}
              \]
              所以
              \[
                  \mat{A} =
                  \begin{pmatrix}
                      3 & 0 & 0 \\
                      3 & 0 & 0 \\
                      3 & 0 & 0
                  \end{pmatrix}
                  \begin{pmatrix}
                      \bm{\alpha}_1 & \bm{\alpha}_2 & \bm{\alpha}_3
                  \end{pmatrix}^{-1}
                  =
                  \begin{pmatrix}
                      3 & 0 & 0 \\
                      3 & 0 & 0 \\
                      3 & 0 & 0
                  \end{pmatrix}
                  \begin{pmatrix}
                      1 & -1 & -1 \\
                      1 & 1  & 0  \\
                      1 & 0  & 1
                  \end{pmatrix}^{-1}
                  =
                  \begin{pmatrix}
                      1 & 1 & 1 \\
                      1 & 1 & 1 \\
                      1 & 1 & 1
                  \end{pmatrix}
              \]
        \item 将$\bm{\alpha}_1$单位化得$\bm{\gamma}_1 = \frac{1}{\sqrt{3}}(1,1,1)^\intercal$,
              对$\bm{\alpha}_2,\bm{\alpha}_3$进行施密特正交化,得
              \begin{align*}
                  \bm{\beta}_2 & = \bm{\alpha}_2 = (-1,1,0)^\intercal                                                           \\
                  \bm{\beta}_3 & = \bm{\alpha}_3 - \frac{(\bm{\alpha}_3,\bm{\beta}_2)}{(\bm{\beta}_2,\bm{\beta}_2)}\bm{\beta}_2
                  = (-1,0,1)^\intercal - \frac{1}{2}(-1,1,0)^\intercal = \frac{1}{2}(-1,-1,2)^\intercal
              \end{align*}
              单位化得
              \begin{align*}
                  \bm{\gamma}_2 & = \frac{1}{\sqrt{2}}(-1,1,0)^\intercal  \\
                  \bm{\gamma}_3 & = \frac{1}{\sqrt{6}}(-1,-1,2)^\intercal
              \end{align*}
              因此正交矩阵为
              \[
                  \mat{Q} =
                  \begin{pmatrix}
                      \frac{1}{\sqrt{3}} & -\frac{1}{\sqrt{2}} & -\frac{1}{\sqrt{6}} \\
                      \frac{1}{\sqrt{3}} & \frac{1}{\sqrt{2}}  & -\frac{1}{\sqrt{6}} \\
                      \frac{1}{\sqrt{3}} & 0                   & \frac{2}{\sqrt{6}}
                  \end{pmatrix}
              \]
    \end{enumerate}
\end{solution}

