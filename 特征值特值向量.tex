\part{特征值、特值向量}
特征值、特征向量无法避开的两个等式如下
\begin{enumerate}[(1)]
    \item $A\alpha = \lambda\alpha, \alpha \neq 0$;
    \item $|\lambda E - A| = 0$;
    \item $(\lambda_i E - A )X = 0$;
\end{enumerate}

以及相似矩阵的概念$A\sim B : \exists \text{可逆矩阵}P, P^{-1}AP = B$,其中的重点是$A$是否相似于对角矩阵。

还有实对称矩阵的概念及相关定理。

\section{特征值、特值向量}
.
\begin{definition}
    设$A$是$n$阶矩阵,$\alpha$是$n$维非零列向量,且
    \[ A\alpha = \lambda\alpha \]
    则称$\lambda$是矩阵$A$的特征值,$\alpha$是矩阵$A$是对应于特征值$\lambda$的特征向量。
\end{definition}
特征值、特征向量的求法主要有定义法、行列式法、基础解系法。


设$n$阶矩阵$A$,非零列向量$\alpha$,则有等式
\[ A\alpha = \lambda\alpha \implies (\lambda E -A)\alpha = 0 \]

\paragraph{行列式法}
这里同齐次线性方程组联系起来,由于$\alpha\neq 0$,所以方程组有非零解,故有行列式法
\begin{equation}
    |\lambda E- A| = 0
\end{equation}
称之为特征多项式。根据代数基本定理,特征多项式一定有$n$个解(允许有重根、复数),即$A$有$n$个特征值。

\paragraph{基础解系法}
另一方面,行列式法得到的特征值$\lambda_i$,带入到等式中,即
\[  (\lambda_i E -A)\alpha = 0 \]
求其基础解系,则可以得到特征值$\lambda_i$对应的一组线性无关的特征向量。
然后得出特征向量的通解(即通过基础解系得出通解)。

\begin{theorem}
    设$\alpha_1,\alpha_2,\cdots,\alpha_t$都是矩阵$A$对应于特征值$\lambda$的特征向量,
    那么当$\alpha = k_1\alpha_1+k_2\alpha_2 + \cdots + k_t\alpha_t \neq 0$时,$\alpha$仍是$A$对应于特征值$\lambda$的特征向量。
\end{theorem}

\begin{theorem}
    如果$\lambda_1,\lambda_2,\cdots,\lambda_m$时矩阵$A$互不相同的特征值,对应的特征向量分别为$\alpha_1,\alpha_2,\cdots,\alpha_m$,
    则$\alpha_1,\alpha_2,\cdots,\alpha_m$\textcolor{red}{线性无关}。
\end{theorem}
这个定理要注意的是,仅仅是线性无关,而不是正交。(正交是实对称矩阵的性质。)

\begin{theorem}
    设$A$为$n$矩阵,特征值为$\lambda_1,\lambda_2,\cdots,\lambda_n$,则
    \begin{enumerate}
        \item $\sum_i^n \lambda_i = \sum_i^n a_{ii}$
        \item $\prod_i^n \lambda_i = |A|$
    \end{enumerate}
\end{theorem}

\begin{example}
    求下面矩阵的特征值,特征向量。
    \[
        A =
        \begin{pmatrix}
            17 & -2 & -2 \\
            -2 & 14 & -4 \\
            -2 & -4 & 14
        \end{pmatrix}
    \]
\end{example}
\begin{solution}
    有$A$的特征多项式
    \begin{align*}
        |\lambda E - A| & =
        \begin{vmatrix}
            \lambda-17 & 2            & 2           \\
            2          & \lambda - 14 & 4           \\
            2          & 4            & \lambda -14
        \end{vmatrix}
        =
        \begin{vmatrix}
            \lambda-17  & 2            & 2           \\
            36-2\lambda & \lambda - 18 & 0           \\
            36-2\lambda & 0            & \lambda -18
        \end{vmatrix}
        =
        \begin{vmatrix}
            \lambda-9 & 2            & 2           \\
            0         & \lambda - 18 & 0           \\
            0         & 0            & \lambda -18
        \end{vmatrix}                    \\
                        & = (\lambda-9)(\lambda-18)^2
    \end{align*}
    得特征值$\lambda_1 = 9, \lambda_2 = 18(\text{重根})$

    当$\lambda=9$时,齐次方程组
    \[
        (9E - A)X =
        \begin{pmatrix}
            -8 & 2  & 2  \\
            2  & -5 & 4  \\
            2  & 4  & -5
        \end{pmatrix}X
        =0
    \]
    则
    \[
        \begin{pmatrix}
            -8 & 2  & 2  \\
            2  & -5 & 4  \\
            2  & 4  & -5
        \end{pmatrix}
        \longrightarrow
        \begin{pmatrix}
            4 & -1  & -1  \\
            4 & -10 & 8   \\
            4 & 8   & -10
        \end{pmatrix}
        \longrightarrow
        \begin{pmatrix}
            4 & -1 & -1 \\
            0 & -9 & 9  \\
            0 & 9  & -9
        \end{pmatrix}
        \longrightarrow
        \begin{pmatrix}
            4 & -1 & -1 \\
            0 & 1  & -1 \\
            0 & 0  & 0
        \end{pmatrix}
        \longrightarrow
        \begin{pmatrix}
            1 & 0 & -\frac{1}{2} \\
            0 & 1 & -1           \\
            0 & 0 & 0
        \end{pmatrix}
    \]
    所以$x_3$是自由变量,令$x_3=1$得$x_1=\frac{1}{2},x_2=1$
    那么基础解系为
    \[
        \alpha = (\frac{1}{2},1,1)^\intercal
    \]
    因此$\lambda = 9$时对应得特征向量为
    \[ t\alpha (t\neq 0) \]
    当$\lambda=18$时,齐次方程组
    \[
        (18E - A)X =
        \begin{pmatrix}
            1 & 2 & 2 \\
            2 & 4 & 4 \\
            2 & 4 & 4
        \end{pmatrix}X
        =0
    \]
    则
    \[
        \begin{pmatrix}
            1 & 2 & 2 \\
            2 & 4 & 4 \\
            2 & 4 & 4
        \end{pmatrix}
        \longrightarrow
        \begin{pmatrix}
            1 & 2 & 2 \\
            0 & 0 & 0 \\
            0 & 0 & 0
        \end{pmatrix}
    \]
    所以$x_2,x_3$是自由变量,令$x_2=1,x_3=0$得$x_1=-2$,令$x_2=0,x_3=1$得$x_1=-2$
    那么基础解系为
    \[
        \alpha_1 = (-2,1,0)^\intercal,\qquad \alpha_2 = (-2,0,1)^\intercal
    \]
    因此$\lambda = 18$时对应得特征向量为
    \[ k_1\beta_1 + k_2\beta_2 (k_1,k_2\text{不全为}0) \]
\end{solution}

特征向量更快的解法由下面给出
\begin{solution}
    当$\lambda=9$时,
    \[
        (9E-A) =
        \begin{pmatrix}
            -8 & 2  & 2  \\
            2  & -5 & 4  \\
            2  & 4  & -5
        \end{pmatrix}
    \]
    由于$|\lambda E - A|=0$,所以它的秩$r<3$,这里可以\textcolor{red}{明显}看出$9E-A$的秩是$2$(若无法明显看出,那么按原始方法作,以减少不必要的麻烦),
    所以一定能经过初等行变换变成两行非零的矩阵,由于$9E-A$的任何两行都不成比例(找出两行不成比例即可,剩余的变为$0$),
    所以直接让某一行变为全$0$,这是因为这一行能被另外两行表示。因此
    \[
        \begin{pmatrix}
            -8 & 2  & 2  \\
            2  & -5 & 4  \\
            2  & 4  & -5
        \end{pmatrix}
        \longrightarrow
        \begin{pmatrix}
            2 & -5 & 4  \\
            2 & 4  & -5 \\
            0 & 0  & 0
        \end{pmatrix}
        \longrightarrow
        \begin{pmatrix}
            2 & -5 & 4  \\
            0 & 9  & -9 \\
            0 & 0  & 0
        \end{pmatrix}
        \longrightarrow
        \begin{pmatrix}
            2 & 0 & -1 \\
            0 & 1 & -1 \\
            0 & 0 & 0
        \end{pmatrix}
        \longrightarrow
        \begin{pmatrix}
            1 & 0 & -\frac{1}{2} \\
            0 & 1 & -1           \\
            0 & 0 & 0
        \end{pmatrix}
    \]
    那么其基础解系为$\alpha = (\frac{1}{2},1,1)^\intercal$
    所以对应的特征向量为$t\alpha (t\neq 0)$
\end{solution}

\begin{example}
    求
    \[
        A =
        \begin{pmatrix}
            -1 & 1 & 0 \\
            -4 & 3 & 0 \\
            1  & 0 & 2
        \end{pmatrix}
    \]
    的特征值、特征向量。
\end{example}
\begin{solution}
    特征多项式
    \[
        |\lambda E - A|
        =
        \begin{vmatrix}
            \lambda+1 & -1        & 0         \\
            4         & \lambda-3 & 0         \\
            -1        & 0         & \lambda-2
        \end{vmatrix}
        =(\lambda-2)
        \begin{vmatrix}
            \lambda+1 & -1        \\
            4         & \lambda-3
        \end{vmatrix}
        = (\lambda-2)(\lambda-1)^2
    \]
    因此特征值为$2,1,1$,
    当$\lambda = 2$时,
    \[
        \lambda E - A =
        \begin{pmatrix}
            3  & -1 & 0 \\
            4  & -1 & 0 \\
            -1 & 0  & 0
        \end{pmatrix}
        \longrightarrow
        \begin{pmatrix}
            1 & 0  & 0 \\
            3 & -1 & 0 \\
            0 & 0  & 0
        \end{pmatrix}
        \longrightarrow
        \begin{pmatrix}
            1 & 0 & 0 \\
            0 & 1 & 0 \\
            0 & 0 & 0
        \end{pmatrix}
    \]
    则$(2E-A)X=0$的基础解系为$\alpha = (0,0,1)^\intercal$
    因此对应的特征向量为$k\alpha (k\neq 0)$

    当$\lambda = 1$时
    \[
        \lambda E - A =
        \begin{pmatrix}
            2  & -1 & 0  \\
            4  & -2 & 0  \\
            -1 & 0  & -1
        \end{pmatrix}
        \longrightarrow
        \begin{pmatrix}
            1 & 0  & 1 \\
            2 & -1 & 0 \\
            0 & 0  & 0
        \end{pmatrix}
        \longrightarrow
        \begin{pmatrix}
            1 & 0  & 1  \\
            0 & -1 & -2 \\
            0 & 0  & 0
        \end{pmatrix}
        \longrightarrow
        \begin{pmatrix}
            1 & 0 & 1 \\
            0 & 1 & 2 \\
            0 & 0 & 0
        \end{pmatrix}
    \]
    所以$(E-A)X=0$基础解系为$\beta=(-1,-2,1)^\intercal$,则对应的特征向量为$t\beta (t\neq 0)$
\end{solution}

注意到这两个例题中,\textcolor{red}{前者重根的特征向量的自由变量有两个,而后者重根的特征向量的自由变量只有一个。}
这个区别关系到矩阵与对角矩阵相似的问题。

根据定义,可以得到下面这个公式
\begin{equation}
    (A+kE)\alpha = (\lambda + k)\alpha
\end{equation}
这表明$(A+kE)$和$A$的特征向量相同,特征值相差$k$

还有下面这个公式
\begin{equation}
    A^n\alpha = \lambda^n\alpha
\end{equation}

\begin{example}
    已知$\alpha=(1,1,-1)^\intercal$是
    \[
        A =
        \begin{pmatrix}
            2  & -1 & 2  \\
            5  & a  & 3  \\
            -1 & b  & -2
        \end{pmatrix}
    \]
    的一个特征向量,求$a,b$
\end{example}
\begin{solution}
    由于$\alpha$是$A$的特征向量,不妨设$\lambda$为$\alpha$对应的特征值,那么有
    \[ A\alpha = \lambda\alpha \]
    即
    \[
        \begin{pmatrix}
            2  & -1 & 2  \\
            5  & a  & 3  \\
            -1 & b  & -2
        \end{pmatrix}
        \begin{pmatrix}
            1 \\1\\-1
        \end{pmatrix}
        =
        \begin{pmatrix}
            -1 \\ 2+a \\ 1+b
        \end{pmatrix}
        =
        \begin{pmatrix}
            \lambda \\ \lambda \\ -\lambda
        \end{pmatrix}
    \]
    因此有
    \[
        \begin{cases}
            \lambda = -1 \\
            a = -3       \\
            b=0
        \end{cases}
    \]
\end{solution}

\subsection{特征值个数与秩的关系}
考虑$n$阶矩阵$A$,根据定义可知特征值满足下面等式
\[ A\alpha = \lambda\alpha \]
同时,由初等变换可知存在$n$阶可逆矩阵$P,Q$使得
\[
    PAQ =
    \begin{pmatrix}
        E_r & 0 \\
        0   & 0
    \end{pmatrix}
\]
其中$r\leq n$,将等式变形可得
\[
    A =
    P^{-1}
    \begin{pmatrix}
        E_r & 0 \\
        0   & 0
    \end{pmatrix}
    Q^{-1}
    =
    P^{-1}
    \begin{pmatrix}
        E_r \\ 0
    \end{pmatrix}
    \begin{pmatrix}
        E_r & 0
    \end{pmatrix}
    Q^{-1}
    =BC
\]
其中$B=P^{-1}\begin{pmatrix}E_r & 0\end{pmatrix}^\intercal, C = \begin{pmatrix}E_r & 0\end{pmatrix}Q^{-1}$,
则$B$为列满秩矩阵,$C$为行满秩矩阵,且$r(A) = r(B) = r(C) = r$。

设$\lambda$为$A$的任意一非零特征值,$\alpha$为对应的特征向量,则有
\[ A\alpha = BC\alpha = \lambda\alpha \]
两边同时乘以矩阵$C$,则有
\[ CBC\alpha = CB(C\alpha) = \lambda(C\alpha) \]
其中$C\alpha\neq 0$,(若$C\alpha = 0$,则有$BC\alpha = 0 = \lambda\alpha = 0$,又$\lambda\neq 0$,则$\alpha=0$,与$\alpha\neq 0$矛盾),
因此矩阵$CB$特征值为$\lambda$,对应特征向量为$C\alpha$。所以$BC$的任意一非零特征值为$CB$的特征值。同理可证$CB$的任意一非零特征值也为$BC$的特征值。
因此$BC,CB$的非零特征值相同。又因为特征值个数不超过矩阵的阶数,所以非零特征值个数不超过$CB$的阶数$r$,即$A=BC$的非零特征值不超过$r$。

综上可得,
\begin{theorem}
    设矩$n$阶矩阵$A$,且$r(A)=r$,则$A$的非零特征值个数小于等于$r$。
\end{theorem}
进一步的,有
\begin{enumerate}[(1)]
    \item 若$A$为可对角化矩阵,根据$A\sim \operatorname{diag}(\lambda_1,\lambda_2,\cdots)$,知非零特征值个数等于矩阵的秩;
    \item 若$A$为满秩矩阵,根据$|A|=\prod \lambda \neq 0$,知$\lambda\neq 0$,则非零特征值等于$n$;
    \item 若$A$为不可对角化的降秩矩阵,则非零特征值个数小于等于它的秩。
\end{enumerate}

\begin{example}
    设$A=E+\alpha\beta^\intercal$,其中$\alpha,\beta$均为$n$阶非零列向量,且有$\alpha^\intercal\beta=3$,求$|A+2E|$
\end{example}
\begin{solution}
    要求$|A+2E|$,则要考虑$A$的特征值,根据特征值的乘积求出行列式。又$A=E+\alpha\beta^\intercal$,因此转为考虑$\alpha\beta^\intercal$的特征值。
    注意到
    \[ r(\alpha\beta^\intercal) \leq \min(r(\alpha),r(\beta^\intercal)) = 1 \]
    且$\alpha\neq 0,\beta\neq 0$,则有$\alpha\beta^\intercal\neq 0$,因此$r(\alpha\beta^\intercal)=1$,所以$\alpha\beta^\intercal$非零特征值不超过$1$。

    假设非零特征值个数为$0$,则特征值为$0$的个数等于$n$,又因为$\Tr(\alpha\beta^\intercal)=\beta^\intercal\alpha = 3\neq 0$,所以非零特征值个数不等于$0$;

    因此非零特征值个数为$1$,此时特征值为$0$的个数等于$n-1$,即$0$为$n-1$重特征值。根据$\Tr(\alpha\beta^\intercal)=3$,得出$3$为单重特征值。综上可得
    \[ |A+2E| = |3E + \alpha\beta^\intercal| = \prod_i^n (3+\lambda_i) = 6\cdot 3^{n-1} = 2\cdot 3^n \]

\end{solution}

\section{相似矩阵}
\subsection{相似矩阵的基本概念和性质}
.
\begin{definition}
    设$A,B$都是$n$阶矩阵,如果存在可逆矩阵$P$,使得
    \[ P^{-1}AP = B \]
    就称矩阵$A$相似于矩阵$B$,$B$是$A$的相似矩阵,记为$A\sim B$
\end{definition}
这里注意与等价概念的区别,等价是$QAP=B$,其中$Q,P$均为可逆矩阵,但$Q,P$可无关系。

下面介绍几个相似的基本性质
\begin{enumerate}[(1)]
    \item (自反性)$A\sim A$;
    \item (对称性)如果$A\sim B$则$B \sim A$;
    \item (传递性)如果$A\sim B, B \sim C$,则$A\sim C$;
    \item (特征值相同)$A\sim B \implies \lambda_A = \lambda_B$;
    \item (秩相等)$A\sim B \implies r(A)=r(B)$;
    \item (行列式相等)$A\sim B \implies |A|=|B|$;
    \item (迹相等)$A\sim B \implies \Tr(A) = \Tr(B)$
\end{enumerate}

下面为相似的运算关系:假设$A\sim B$
\begin{enumerate}[(1)]
    \item $A^2 \sim B^2$;
    \item $A+kE \sim B + kE$;
    \item 如果$A$可逆,则$A^{-1}\sim B^{-1}$
    \item 如果$A_1\sim B_1, A_2\sim B_2$,则
          $
              \begin{pmatrix}
                  A_1 &     \\
                      & A_2
              \end{pmatrix}
              \sim
              \begin{pmatrix}
                  B_1 &     \\
                      & B_2
              \end{pmatrix}
          $
\end{enumerate}

\begin{example}
    已知
    \[
        A =
        \begin{pmatrix}
            2 & 0 & 0 \\
            0 & 0 & 1 \\
            0 & 1 & x
        \end{pmatrix}
        ,\qquad
        B =
        \begin{pmatrix}
            2 & 0 & 0  \\
            0 & y & 0  \\
            0 & 0 & -1
        \end{pmatrix}
    \]
    相似,求$x,y$
\end{example}
\begin{solution}
    由于$A\sim B$,所以$\Tr(A)=\Tr(B),|A|=|B|$,即
    \begin{align*}
        2+x & = 1+y \\
        -2  & = -2y
    \end{align*}
    解得$x=0,y=1$
\end{solution}

\subsection{相似对角化}
.
\begin{definition}
    如果$A$和对角矩阵相似,则称矩阵$A$可相似对角化。
\end{definition}

\begin{theorem}
    $A$和对角矩阵相似$\iff A$有$n$个线性无关特征向量。
\end{theorem}
作为推论有
\begin{theorem}
    $A$有$n$个不同的特征值$\implies A$可相似对角化。
\end{theorem}
这是因为不同的特征值对应的特征向量,一定线性无关。注意这里不是充要条件,这是因为特征值有重根的$A$也是有可能相似对角化。

\begin{theorem}
    $A$可相似对角化$\iff \lambda$是$A$的$k$重根特征值,则$\lambda$有$k$个线性无关的特征向量。
\end{theorem}
简而言之,特征值相同的个数等于对应的特征向量的个数时,矩阵能相似对角化。

相似对角化的解题步骤:假设$3$阶矩阵$A$可相似对角化
\begin{enumerate}[(1)]
    \item 求特征值$\lambda_1, \lambda_2, \lambda_3$;
    \item 求特征向量$\alpha_1,\alpha_2,\alpha_3$;
    \item 构造可逆矩阵$P=\begin{pmatrix} \alpha_1 & \alpha_2& \alpha_3 \end{pmatrix}$;
    \item 最后有对角化矩阵$\operatorname{diag}(\lambda_1,\lambda_2,\lambda_3)$
\end{enumerate}

利用相似对角化来求$A^n$:
\[
    P^{-1}AP = B \implies    P^{-1}A^nP = B^n \implies A^n = PB^nP^{-1}
\]
其中$B$为对角矩阵,因此有
\begin{equation}
    A^n = PB^nP^{-1} =
    P
    \begin{pmatrix}
        b_1 &        &     \\
            & \ddots &     \\
            &        & b_m
    \end{pmatrix}^nP^{-1}
    =P
    \begin{pmatrix}
        b_1^n &        &       \\
              & \ddots &       \\
              &        & b_m^n
    \end{pmatrix}P^{-1}
\end{equation}

\begin{example}
    已知
    \[
        A =
        \begin{pmatrix}
            2 & 0 & 0 \\
            0 & 0 & 1 \\
            0 & 1 & x
        \end{pmatrix}
        ,\qquad
        B =
        \begin{pmatrix}
            2 & 0 & 0  \\
            0 & y & 0  \\
            0 & 0 & -1
        \end{pmatrix}
    \]
    相似,求$x,y$,求可逆矩阵$P$,使得$P^{-1}AP=B$
\end{example}
\begin{solution}
    此例题在相似矩阵部分中解得$x=0,y=1$。因此
    \[
        A =
        \begin{pmatrix}
            2 & 0 & 0 \\
            0 & 0 & 1 \\
            0 & 1 & 0
        \end{pmatrix}
        ,\qquad
        B =
        \begin{pmatrix}
            2 & 0 & 0  \\
            0 & 1 & 0  \\
            0 & 0 & -1
        \end{pmatrix}
    \]
    可以看出$B$为对角矩阵,根据相似对角化的性质,可以得到$A$的三个特征值$2,1,-1$,
    当$\lambda=2$时,$(2E-A)\alpha=0(\alpha\neq 0)$,
    \[
        (2E-A) =
        \begin{pmatrix}
            0 & 0  & 0  \\
            0 & 2  & -1 \\
            0 & -1 & 0  \\
        \end{pmatrix}
        \longrightarrow
        \begin{pmatrix}
            0 & 1 & 0 \\
            0 & 0 & 1 \\
            0 & 0 & 0 \\
        \end{pmatrix}
    \]
    因此一个特征向量为$\alpha_1 = (1,0,0)^\intercal$,

    当$\lambda = 1$时,
    \[
        (E-A) =
        \begin{pmatrix}
            -1 & 0  & 0  \\
            0  & 1  & -1 \\
            0  & -1 & 1  \\
        \end{pmatrix}
        \longrightarrow
        \begin{pmatrix}
            1 & 0 & 0  \\
            0 & 1 & -1 \\
            0 & 0 & 0  \\
        \end{pmatrix}
    \]
    因此一个特征向量为$\alpha_2 = (0,1,1)^\intercal$,

    当$\lambda = -1$时,
    \[
        (-E-A) =
        \begin{pmatrix}
            -3 & 0  & 0  \\
            0  & -1 & -1 \\
            0  & -1 & -1 \\
        \end{pmatrix}
        \longrightarrow
        \begin{pmatrix}
            1 & 0 & 0 \\
            0 & 1 & 1 \\
            0 & 0 & 0 \\
        \end{pmatrix}
    \]
    因此一个特征向量为$\alpha_3 = (0,-1,1)^\intercal$,
    那么
    \[
        P =
        \begin{pmatrix}
            \alpha_1 & \alpha_2 & \alpha_3
        \end{pmatrix}
        =
        \begin{pmatrix}
            1 & 0 & 0  \\
            0 & 1 & -1 \\
            0 & 1 & 1
        \end{pmatrix}
    \]
\end{solution}

\begin{example}
    若
    \[
        A =
        \begin{pmatrix}
            1 & 3  & 0 \\
            1 & -1 & 0 \\
            a & -6 & 2
        \end{pmatrix}
    \]
    可相似对角化,求$a$,并求可逆矩阵$P$使得$P^{-1}AP=B$,$B$为对角化矩阵。
\end{example}
\begin{solution}
    $A$的特征多项式为
    \[
        |\lambda E-A| =
        \begin{vmatrix}
            \lambda-1 & -3        & 0         \\
            -1        & \lambda+1 & 0         \\
            -a        & 6         & \lambda-2
        \end{vmatrix}
        =(\lambda-2)[(\lambda-1)(\lambda+1)-3]
        =(\lambda-2)^2(\lambda+2)
    \]
    因此特征值为$-2,2(\text{二重根})$由于$A$可相似对角化,所以特征值为$2$的特征向量有$2$个。
    则
    \[
        (2E-A) =
        \begin{pmatrix}
            1  & -3 & 0 \\
            -1 & 3  & 0 \\
            -a & 6  & 0
        \end{pmatrix}
        \longrightarrow
        \begin{pmatrix}
            1  & -3 & 0 \\
            -a & 6  & 0 \\
            0  & 0  & 0
        \end{pmatrix}
        \longrightarrow
        \begin{pmatrix}
            1 & -3   & 0 \\
            0 & 6-3a & 0 \\
            0 & 0    & 0
        \end{pmatrix}
    \]
    自由变量有两个,所以$6-3a=0$,即$a=2$。此时$\lambda=2$的两个线性无关特征向量为
    \[ \alpha_1 = (3,1,0)^\intercal ,\qquad \alpha_2 = (0,0,1)^\intercal \]

    当$\lambda=-2$时,
    \[
        (-2E-A) =
        \begin{pmatrix}
            -3 & -3 & 0  \\
            -1 & -1 & 0  \\
            -a & 6  & -4
        \end{pmatrix}
        \longrightarrow
        \begin{pmatrix}
            1  & 1 & 0  \\
            -1 & 3 & -2 \\
            0  & 0 & 0
        \end{pmatrix}
        \longrightarrow
        \begin{pmatrix}
            1 & 1 & 0  \\
            0 & 4 & -2 \\
            0 & 0 & 0
        \end{pmatrix}
        \longrightarrow
        \begin{pmatrix}
            2 & 2 & 0  \\
            0 & 2 & -1 \\
            0 & 0 & 0
        \end{pmatrix}
        \longrightarrow
        \begin{pmatrix}
            2 & 0 & 1  \\
            0 & 2 & -1 \\
            0 & 0 & 0
        \end{pmatrix}
    \]
    所以$\lambda=-2$对应的一个特征向量为$\alpha_3 = (-1,1,2)^\intercal$,
    因此矩阵
    \[
        P =
        \begin{pmatrix}
            \alpha_1 & \alpha_2 & \alpha_3
        \end{pmatrix}
        =
        \begin{pmatrix}
            3 & 0 & -1 \\
            1 & 0 & 1  \\
            0 & 1 & 2
        \end{pmatrix}
    \]
    对角矩阵为
    \[
        \begin{pmatrix}
            2 & 0 & 0  \\
            0 & 2 & 0  \\
            0 & 0 & -2
        \end{pmatrix}
    \]
\end{solution}

\section{实对称矩阵}
.
\begin{definition}
    对于$n$阶矩阵$A$,如果有$A^\intercal = A$,那么称矩阵$A$为实对称矩阵。
\end{definition}
\begin{theorem}
    实对称矩阵一定可以相似对角化。
\end{theorem}
\begin{theorem}
    实对称矩阵不同特征值所对应的特征向量\textcolor{red}{相互正交}。
\end{theorem}
这里相互正交,就会有内积为$0$,也就能引申到齐次方程组相关的知识点。

\begin{theorem}
    实对称矩阵一定存在正交矩阵\ref{sec:正交矩阵}$Q$,使得
    \[ Q^{-1}AQ = Q^\intercal AQ = B \]
    其中$B$为对角化矩阵。
\end{theorem}
在求\textcolor{red}{实对称矩阵}的相似对角化的\textcolor{red}{正交矩阵$Q$}时候,步骤在相似对角化的步骤上额外加上向量的正交标准化:
\begin{enumerate}[(1)]
    \item 求出特征值$\lambda_1,\lambda_2,\lambda_3$;
    \item 求出特征向量$\alpha_1,\alpha_2,\alpha_3$;
    \item 由于实对称矩阵中,特征值不同的特征向量相互正交,故只需要单位化;
    \item 当特征值是重根时,如果特征向量已经正交,则只需单位化;
    \item 当特征向量是重根时,且特征向量不正交,需要施密特正交单位化;
    \item 得到新的特征向量为$\gamma_1,\gamma_2,\gamma_3$,则正交矩阵$Q=\begin{pmatrix}=\gamma_1 & \gamma_2 & \gamma_3\end{pmatrix}$;
    \item 最后有
          \[
              Q^{-1}AQ =
              \begin{pmatrix}
                  \lambda_1 &           &           \\
                            & \lambda_2 &           \\
                            &           & \lambda_3
              \end{pmatrix}
          \]
\end{enumerate}

\begin{example}
    设$3$阶矩阵$A$为实对称矩阵,其特征值为$3,0,0$,且$\lambda=3$的特征向量$\alpha_1 = (1,1,1)^\intercal$
    \begin{enumerate}[(1)]
        \item 求$\lambda=0$的特征向量;
        \item 求$A$;
        \item 求正交矩阵$Q$,使得$Q^{-1}AQ$为对角化矩阵。
    \end{enumerate}
\end{example}
\begin{solution}
    \begin{enumerate}[(1)]
        \item 由于$A$是实对称矩阵,设$\lambda=0$的两个特征向量为$\alpha_2,\alpha_3$,因此有
              \[ (\alpha_1,\alpha_2) = (\alpha_1,\alpha_3) = 0 \]
              即$\alpha_2,\alpha_3$为齐次方程组$\alpha_1X=x_1+x_2+x_3=0$的非零解。

              令$x_2=1,x_3=0$得$\alpha_2 =(-1,1,0)^\intercal$;
              令$x_2=0,x_3=1$得$\alpha_3=(-1,0,1)^\intercal$;

              所以$\lambda=0$得特征向量为$k_2\alpha_2+k_3\alpha_3 (k_2,k_3\text{不全为}0)$

        \item 由于$A$可相似对角化,故有
              \[
                  A
                  \begin{pmatrix}
                      \alpha_1 & \alpha_2 & \alpha_3
                  \end{pmatrix}
                  =
                  \begin{pmatrix}
                      \alpha_1 & \alpha_2 & \alpha_3
                  \end{pmatrix}
                  \begin{pmatrix}
                      3 & 0 & 0 \\
                      0 & 0 & 0 \\
                      0 & 0 & 0
                  \end{pmatrix}
                  =
                  \begin{pmatrix}
                      3 & 0 & 0 \\
                      3 & 0 & 0 \\
                      3 & 0 & 0
                  \end{pmatrix}
              \]
              所以
              \[
                  A =
                  \begin{pmatrix}
                      3 & 0 & 0 \\
                      3 & 0 & 0 \\
                      3 & 0 & 0
                  \end{pmatrix}
                  \begin{pmatrix}
                      \alpha_1 & \alpha_2 & \alpha_3
                  \end{pmatrix}^{-1}
                  =
                  \begin{pmatrix}
                      3 & 0 & 0 \\
                      3 & 0 & 0 \\
                      3 & 0 & 0
                  \end{pmatrix}
                  \begin{pmatrix}
                      1 & -1 & -1 \\
                      1 & 1  & 0  \\
                      1 & 0  & 1
                  \end{pmatrix}^{-1}
                  =
                  \begin{pmatrix}
                      1 & 1 & 1 \\
                      1 & 1 & 1 \\
                      1 & 1 & 1
                  \end{pmatrix}
              \]
        \item 将$\alpha_1$单位化得$\gamma_1 = \frac{1}{\sqrt{3}}(1,1,1)^\intercal$,
              对$\alpha_2,\alpha_3$进行施密特正交化,得
              \begin{align*}
                  \beta_2 & = \alpha_2 = (-1,1,0)^\intercal                                  \\
                  \beta_3 & = \alpha_3 - \frac{(\alpha_3,\beta_2)}{(\beta_2,\beta_2)}\beta_2
                  = (-1,0,1)^\intercal - \frac{1}{2}(-1,1,0)^\intercal = \frac{1}{2}(-1,-1,2)^\intercal
              \end{align*}
              单位化得
              \begin{align*}
                  \gamma_2 & = \frac{1}{\sqrt{2}}(-1,1,0)^\intercal  \\
                  \gamma_3 & = \frac{1}{\sqrt{6}}(-1,-1,2)^\intercal
              \end{align*}
              因此正交矩阵为
              \[
                  Q =
                  \begin{pmatrix}
                      \frac{1}{\sqrt{3}} & -\frac{1}{\sqrt{2}} & -\frac{1}{\sqrt{6}} \\
                      \frac{1}{\sqrt{3}} & \frac{1}{\sqrt{2}}  & -\frac{1}{\sqrt{6}} \\
                      \frac{1}{\sqrt{3}} & 0                   & \frac{2}{\sqrt{6}}
                  \end{pmatrix}
              \]
    \end{enumerate}
\end{solution}

