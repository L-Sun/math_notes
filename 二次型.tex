\part{二次型}
二次型得知识点应当与特征值、特征向量相联系。
\section{二次型得基本概念、定理}
.
\begin{definition}
    设$F$是一数域,一个系数$F$中含有$n$个变元$x_1,x_2,\cdots,x_n$的\textcolor{red}{二次}齐次多项式
    \begin{align*}
        f(x_1,x_2,\cdots,x_n) = a_{11}x_1^2 + 2a_{12}x_1x_2 + \cdots + 2a_{1n}x_1x_n & \\
        +a_{22}x_2^2 + \cdots + 2a_{2n}x_2x_n                                        & \\
        +\cdots                                                                      & \\
        +a_{nn}x_n^2                                                                 &
    \end{align*}
    称为数域$F$上的一个$n$元二次型,简称为二次型。
\end{definition}

对于二次型,可以将其表示为矩阵形式
\begin{equation}
    f(x_1,x_2,\cdots,x_n) = \mat{X}^\intercal \mat{A}\mat{X}
\end{equation}
其中$\mat{A}$为对称矩阵,即
\[
    \mat{A} =\mat{A}^\intercal =
    \begin{pmatrix}
        a_{11} & a_{12} & \cdots & a_{1n} \\
        a_{12} & a_{22} & \cdots & a_{2n} \\
        \vdots & \vdots & \ddots & \vdots \\
        a_{1n} & a_{2n} & \cdots & a_{nn}
    \end{pmatrix}
\]

将一个二次型的$\mat{A}$矩阵写法为:
\begin{enumerate}
    \item 将平方项$x_i^2$的系数写在对角线元素$a_{ii}$上;
    \item 将混合项$x_ix_j$的系数一半写在$a_{ij}$,另一半写在$a_{ji}$上。
\end{enumerate}

\begin{definition}
    (二次型的秩)
    二次型矩阵$\mat{A}$的秩称为二次型的秩,记为$r(f)=r(\mat{A})$
\end{definition}

\section{标准型}
.
\begin{definition}
    (标准型)
    如果二次型中只有平方项$x_i^2$,而没有混合项$x_{ij}$,则称为标准型。
    \[ f(x_1,x_2,\cdots,x_n) = a_1x_1^2 + a_2x_2^2 + \cdots + a_nx_n^2 \]
\end{definition}
显然标准型的矩阵是一个对角矩阵。

\begin{definition}
    (规范型)
    对于标准型
    \[ f(x_1,x_2,\cdots,x_n) = a_1x_1^2 + a_2x_2^2 + \cdots + a_nx_n^2 \]
    如果$a_i \in \{-1,0,1\}$,则称这个二次型为规范型。
\end{definition}

\begin{definition}
    (惯性指数)
    当一个二次型化为标准型后,如果系数为正的个数为\textcolor{red}{\textbf{\textsf{正惯性指数}}},记为$p$;
    如果系数为负的个数为\textcolor{red}{\textbf{\textsf{负惯性指数}}},记为$q$。
\end{definition}
\begin{theorem}
    二次型矩阵$\mat{A}$的特征值为正的个数等于二次型的正惯性指数,二次型矩阵$\mat{A}$的特征值为负的个数等于二次型的负惯性指数。
\end{theorem}
作为推论有
\begin{theorem}
    正负惯性指数之和等于矩阵的秩。
\end{theorem}
\begin{example}
    若二次型
    \[ f(x_1,x_2,x_3) = x_1^2 + ax_2^2 + x_3^2 + 2x_1x_2 - 2ax_1x_3 - 2x_2x_3 \]
    的正负惯性指数都是$1$,求$a$的值。
\end{example}
\begin{solution}
    由于$p=q=1$,所以$r(f) = p+q = 2 < 3$,则有二次型的矩阵的行列式等于$0$,
    \begin{align*}
        |\mat{A}| & =
        \begin{vmatrix}
            1  & 1  & -a \\
            1  & a  & -1 \\
            -a & -1 & 1
        \end{vmatrix}
        =
        \begin{vmatrix}
            0  & 1-a & 1-a \\
            1  & a   & -1  \\
            -a & -1  & 1
        \end{vmatrix}
        =(1-a)
        \begin{vmatrix}
            0  & 1  & 1  \\
            1  & a  & -1 \\
            -a & -1 & 1
        \end{vmatrix}
        \\
                  & =(1-a)
        \begin{vmatrix}
            1   & 1+a & 0  \\
            1   & a   & -1 \\
            1-a & a-1 & 0
        \end{vmatrix}
        =(1-a)^2
        \begin{vmatrix}
            1 & 1+a & 0  \\
            1 & a   & -1 \\
            1 & -1  & 0
        \end{vmatrix}
        =-(1-a)^2(a+2)
    \end{align*}
    因此$a=1,-2$,当$a=1$时,矩阵的秩$r(\mat{A})=1$不满足$r(f)=2$,而$a=-2$时,满足$r(f)=r(\mat{A})=2$,故$a=-2$

\end{solution}

\begin{definition}
    (坐标变换)
    \[
        \begin{cases}
            x_1 = c_{11}y_1 + c_{12}y_2 + \cdots + c_{1n}y_n \\
            x_2 = c_{21}y_1 + c_{22}y_2 + \cdots + c_{2n}y_n \\
            ...                                              \\
            x_n = c_{n1}y_1 + c_{n2}y_2 + \cdots + c_{nn}y_n
        \end{cases}
    \]
    且矩阵
    \[
        \mat{C} =
        \begin{pmatrix}
            c_{11} & c_{12} & \cdots & c_{1n} \\
            c_{21} & c_{22} & \cdots & c_{2n} \\
            \vdots & \vdots & \ddots & \vdots \\
            c_{n1} & c_{n2} & \cdots & c_{nn}
        \end{pmatrix}
    \]
    可逆(行列式不为$0$),则称$\mat{X}$为$\mat{Y}$经过$\mat{C}$的坐标变换。
\end{definition}

对于一个二次型$\mat{X}^\intercal \mat{A}\mat{X}$进行任意的坐标变换$\mat{X}=\mat{CY}$,则有
\[ \mat{X}^\intercal \mat{A}\mat{X} = (\mat{CY})^\intercal \mat{A} (\mat{CY}) = \mat{Y}^\intercal (\mat{C}^\intercal \mat{A} \mat{C})\mat{Y} \]
记$\mat{B}=\mat{C}^\intercal \mat{A} \mat{C}$,显然有
\[ \mat{B}^\intercal = (\mat{C}^\intercal \mat{A} \mat{C})^\intercal = \mat{C}^\intercal (\mat{C}^\intercal \mat{A})^\intercal = \mat{C}^\intercal \mat{A}^\intercal \mat{C} = \mat{C}^\intercal \mat{A}\mat{C} = \mat{B} \]
所以$\mat{B}$是新的二次型$\mat{Y}^\intercal \mat{B}\mat{Y}$的矩阵。

\begin{definition}
    如果两个$n$阶可逆矩阵$\mat{C}$,使得$n$阶矩阵$\mat{A},\mat{B}$满足
    \[ \mat{B} = \mat{C}^\intercal \mat{A} \mat{C} \]
    则称$\mat{A},\mat{B}$合同,记为$\mat{A}\simeq \mat{B}$
\end{definition}
注意这里的定义并没有$\mat{A}=\mat{A}^\intercal$。

\begin{theorem}
    对于任意一个$n$元二次型
    \[ f(x_1,x_2,\cdots,x_n) = \mat{X}^\intercal \mat{A}\mat{X} \]
    存在正交变换$\mat{X}=\mat{QY}$,使得$f$化为标准型。
\end{theorem}
\begin{proof}
    设二次型矩阵为$\mat{A}$,由于$\mat{A}$为实对称矩阵,所以存在正交矩阵$\mat{Q}$,使得$\mat{Q}^{-1}\mat{A}\mat{Q} =  \mat{Q}^\intercal \mat{A}\mat{Q}$为对角矩阵$\mat{B}$。
    令$\mat{X}=\mat{QY}$,则有
    \[f = \mat{X}^\intercal \mat{A}\mat{X} = (\mat{QY})^\intercal \mat{A} (\mat{QY}) = \mat{Y}^\intercal (\mat{Q}^\intercal \mat{A}\mat{Q}) \mat{Y} = \mat{Y}^\intercal \mat{B}\mat{Y}\]
    由于$\mat{B}$为对角矩阵,那么新的二次型$\mat{Y}^\intercal \mat{B} \mat{Y}$为标准型。
\end{proof}

根据这条定理,就可以与可对角化矩阵的特征值联系起来。即
\begin{equation}
    \mat{X}^\intercal \mat{A}\mat{X} = (\mat{QY})^\intercal \mat{A} (\mat{QY}) = \mat{Y}^\intercal \mat{B} \mat{Y} = \lambda_1y_1^2+\lambda_2y_2^2+\cdots+\lambda_ny_n^2
\end{equation}
这里$\lambda_1,\lambda_2,\cdots,\lambda_n$为$\mat{A}$的特征值,坐标变换就是$\mat{Q}$矩阵,也就是特征值的特征向量。

因此一般的二次型变标准型的方法步骤为
\begin{enumerate}
    \item 计算二次型矩阵$\mat{A}$的特征值,特征值就是标准型的二次型系数;
    \item 特征值对应的特征向量组成的矩阵,就是二次型的坐标变换矩阵。
\end{enumerate}

\begin{theorem}
    对于任意一个$n$元二次型$f=\mat{X}^\intercal \mat{A} \mat{X}$都可以通过(配方法)可逆线性变换$\mat{X}=\mat{CY}$化为标准型,其中$\mat{C}$为可逆矩阵。
\end{theorem}
\begin{example}
    用配方法化二次型
    \[ 2x_1^2 + 3x_2^2 + 5x_3^2 + 4x_1x_2 - 8x_2x_3 - 4x_3x_1 \]
    为标准型,并写成所用的坐标变换。
\end{example}
\begin{solution}
    先将$x_1$集中起来,然后为$x_1$配方,之后$x_2,x_3$同理
    \begin{align*}
        f & = 2[x_1^2 + 2x_1(x_2-x_3)] + 3x_2^2 + 5x_3^2 -8x_2x_3                              \\
          & = 2[x_1^2 + 2x_1(x_2-x_3) + (x_2-x_3)^2] - 2(x_2-x_3)^2 + 3x_2^2 + 5x_3^2 -8x_2x_3 \\
          & = 2(x_1+x_2-x_3)^2 - 2x_2^2 + 4x_2x_3 - 2x_3^2 + 3x_2^2 + 5x_3^2 -8x_2x_3          \\
          & = 2(x_1+x_2-x_3)^2 + (x_2^2-4x_2x_3) + 3x_3^2                                      \\
          & = 2(x_1+x_2-x_3)^2 + (x_2^2-4x_2x_3 +4x_3^2) - 4x_3^2 + 3x_3^2                     \\
          & = 2(x_1+x_2-x_3)^2 + (x_2-2x_3)^2 - x_3^2
    \end{align*}
    令
    \[
        \begin{cases}
            y_1 = x_1+x_2-x_3 \\
            y_2 = x_2-2x_3    \\
            y_3 = x_3
        \end{cases}
    \]
    则有
    \[
        C  =
        \begin{pmatrix}
            1 & 1 & -1 \\
            0 & 1 & -2 \\
            0 & 0 & 1
        \end{pmatrix}^{-1}
    \]
    对$C^{-1}$作行初等变换
    \[
        \left(\begin{array}{ccc|ccc}
                1 & 1 & -1 & 1 & 0 & 0 \\
                0 & 1 & -2 & 0 & 1 & 0 \\
                0 & 0 & 1  & 0 & 0 & 1
            \end{array}\right)
        \longrightarrow
        \left(\begin{array}{ccc|ccc}
                1 & 1 & 0 & 1 & 0 & 1 \\
                0 & 1 & 0 & 0 & 1 & 2 \\
                0 & 0 & 1 & 0 & 0 & 1
            \end{array}\right)
        \longrightarrow
        \left(\begin{array}{ccc|ccc}
                1 & 0 & 0 & 1 & -1 & -1 \\
                0 & 1 & 0 & 0 & 1  & 2  \\
                0 & 0 & 1 & 0 & 0  & 1
            \end{array}\right)
    \]
    因此坐标变换为
    \[
        \mat{X} =
        \begin{pmatrix}
            1 & -1 & -1 \\
            0 & 1  & 2  \\
            0 & 0  & 1
        \end{pmatrix}\mat{Y}
    \]
\end{solution}

\begin{example}
    二次型$x_1^2 + 3x_2^2 + x_3^2 +2ax_1x_2 + 2x_1x_3 + 2x_2x_3$,
    经过正交变换$\mat{X}=\mat{PY}$化为标准型$y_1^2 + 4y_2^2$,求$a$的值。
\end{example}
\begin{solution}
    二次型的矩阵为
    \[
        \mat{A} =
        \begin{pmatrix}
            1 & a & 1 \\
            a & 3 & 1 \\
            1 & 1 & 1
        \end{pmatrix}
    \]
    由于二次型经过正交变换变为标准型$y_1^2 + 4y_2^2$,故$\mat{A}$的特征值为$1,4,0$,
    因此有$|\lambda \mat{E} - \mat{A}| = 0$,或则$\prod \lambda = |\mat{A}|$
    \[
        1\times 4\times 0 = 0 = |\mat{A}| =
        \begin{vmatrix}
            1 & a & 1 \\
            a & 3 & 1 \\
            1 & 1 & 1
        \end{vmatrix}
        =
        \begin{vmatrix}
            1   & a   & 1 \\
            a-1 & 3-a & 0 \\
            0   & 1-a & 0
        \end{vmatrix}
        =(a-1)(1-a)
    \]
    因此$a=1$
\end{solution}
\section{正定矩阵}
.
\begin{definition}
    若$\forall \mat{X} \neq 0$,恒有二次型
    \[f(x_1,x_2,\cdots,x_n) = \mat{X}^\intercal \mat{A} \mat{X} >0\]
    则称二次型$f$为正定二次型,其中$\mat{A}$称为正定矩阵。
\end{definition}
举个例子
\[
    \begin{cases}
        f(x_1,x_2,x_3) = x_1^2 + 3x_2^2 + 5x_3^2 & \text{是正定二次型}   \\
        f(x_1,x_2,x_3) = x_1^2 + 3x_2^2          & \text{不是正定二次型}
    \end{cases}
\]
后者当$\mat{X}=(0,0,1)^\intercal\neq 0$时,$f=0$,所以不是正定二次型。


\begin{theorem}
    (正定的充分必要条件)
    \begin{align*}
         & f=\mat{X}^\intercal \mat{A}\mat{X}\text{是正定二次型}                                                                                          \\
         & \iff \text{正惯性系数}p=n                                                                                                                      \\
         & \iff \mat{A}\simeq E (\text{存在可逆矩阵}C\text{,使得}\mat{C}^\intercal \mat{A}C=E\text{亦或是存在可逆矩阵}D\text{使得}\mat{A}=D^\intercal D) \\
         & \iff \mat{A}\text{的特征值全部大于}0
    \end{align*}
\end{theorem}

\begin{theorem}
    (正定的必要条件)
    \begin{enumerate}
        \item 一个二次型是正定二次型$\implies$二次项系数大于$0$;
        \item 二次型矩阵$\mat{A}$主对角线全大于$0$;
        \item $|\mat{A}| = \prod \lambda > 0$
    \end{enumerate}
\end{theorem}
注意这里并不是充要条件,也就是说正定二次型的二次项系数大于$0$,逆反命题则是二次项系数不大于$0$的二次型不是正定的。

\begin{example}
    判断二次型$f(x_1,x_2,x_3)=2x_1^2 + 5x_2^2 + 5x_3^2 + 4x_1x_2 - 4x_1x_3 - 8x_2x_3$
    的正定性。
\end{example}
在判断一个二次型是否正定时,一般采用\textcolor{red}{\textbf{\textsf{顺序主子式}}}的方法。
\begin{solution}
    (顺序主子式)
    \[
        \mat{A} =
        \begin{pmatrix}
            2  & 2  & -2 \\
            2  & 5  & -4 \\
            -2 & -4 & 5
        \end{pmatrix}
    \]
    即$M_i$为$\mat{A}$的$i$阶顺序主子式。即
    \[
        M_1 =
        \begin{vmatrix}
            2
        \end{vmatrix},\qquad
        M_2 =
        \begin{vmatrix}
            2 & 2 \\
            2 & 5
        \end{vmatrix},\qquad
        \begin{vmatrix}
            2  & 2  & -2 \\
            2  & 5  & -4 \\
            -2 & -4 & 5
        \end{vmatrix}
    \]
    那么$M_1 > 0, M_2>0, M_3>0$
    因此二次型式正定的。
\end{solution}

第二种方法就是计算特征值,判断特征值是否全大于$0$;

第三种方法就是将二次型化成标准型(配方法),此时如果二次项系数全部大于$0$,则元二次型式正定的。

顺序主子式也可以用于计算正定二次型中的未知参数。
\begin{example}
    已知正定二次型$f(x_1,x_2,x_3)=x_1^2 + 4x_2^2 + 4x_3^2 + 2tx_1x_2 - -2x_1x_3 + 4x_2x_3$
    求$t$的值。
\end{example}
\begin{solution}
    二次型矩阵为
    \[
        \mat{A} =
        \begin{pmatrix}
            1  & t & -1 \\
            t  & 4 & 2  \\
            -1 & 2 & 4
        \end{pmatrix}
    \]
    则其顺序主子式分别为
    \begin{align*}
         &
        \begin{vmatrix}
            1
        \end{vmatrix}  = 1    \\
         &
        \begin{vmatrix}
            1 & t \\
            t & 4
        \end{vmatrix} = 4-t^2 \\
         &
        \begin{vmatrix}
            1  & t & -1 \\
            t  & 4 & 2  \\
            -1 & 2 & 4
        \end{vmatrix} =
        \begin{vmatrix}
            1   & t    & -1 \\
            t+2 & 4+2t & 0  \\
            -1  & 2    & 4
        \end{vmatrix}
        =
        \begin{vmatrix}
            1   & t+2  & -1 \\
            t+2 & 8+4t & 0  \\
            -1  & 0    & 4
        \end{vmatrix}
        =
        \begin{vmatrix}
            1-t & t+2  & -1 \\
            0   & 8+4t & 0  \\
            0   & 0    & 4
        \end{vmatrix}
        = 16(1-t)(2+t)
    \end{align*}
    因此有
    \[
        \begin{cases}
            1>0     \\
            4-t^2>0 \\
            16(1-t)(2+t) > 0
        \end{cases}
    \]
    因此$t\in(-2,1)$
\end{solution}

\begin{example}
    设
    \[
        \mat{A} =
        \begin{pmatrix}
            0 & 1 & 1 \\
            1 & 2 & 1 \\
            1 & 1 & 0
        \end{pmatrix}
    \]若$\mat{A}+k\mat{E}$是正定矩阵,求$k$的值。
\end{example}
\begin{solution}
    由于出现$\mat{A}+k\mat{E}$,可以联想到特征值的相关概念。首先计算$\mat{A}$的特征值,
    特征多项式
    \begin{align*}
        |\lambda \mat{E} - \mat{A}| & =
        \begin{vmatrix}
            \lambda & -1        & -1      \\
            -1      & \lambda-2 & -1      \\
            -1      & -1        & \lambda
        \end{vmatrix}
        =
        \begin{vmatrix}
            \lambda    & -1        & -1        \\
            -1-\lambda & \lambda-1 & 0         \\
            -1-\lambda & 0         & \lambda+1
        \end{vmatrix}
        =
        \begin{vmatrix}
            \lambda-2 & -1        & -1        \\
            -2        & \lambda-1 & 0         \\
            0         & 0         & \lambda+1
        \end{vmatrix}      \\
                                    & =
        (\lambda+1)[(\lambda-2)(\lambda-1)-2]
        =\lambda(\lambda+1)(\lambda - 3)
    \end{align*}
    因此$\mat{A}$的特征值为$0,-1,3$,则$\mat{A}+k\mat{E}$的特征值为$k,k-1,k+3$,由于$\mat{A}+k\mat{E}$是正定矩阵,故有
    \[
        \begin{cases}
            k>0   \\
            k-1>0 \\
            k+3>0
        \end{cases}
    \]
    因此可得$k>1$
\end{solution}