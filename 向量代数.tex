\part{向量代数}
\section{向量的概念}
.
\begin{definition}
    所谓数域$F$上的一个$n$为向量就是由数域$F$中$n$个数组成的有序数组
    \[ (a_1,a_2,\cdots,a_n) \]
    其中$a_i(i=1,2,\cdots,n)$称为向量的第$i$个分量。
    特别的当所有分量均为$0$时,称为零向量,记作$0$。
\end{definition}
向量有几个个基本的运算,同时满足结合律,交换律。
\begin{enumerate}[(1)]
    \item $\alpha+\beta = (a_1+b_1,a_2+b_2, \cdots, a_n+b_n)$;
    \item $k\alpha = (ka_1,ka_2,\cdots,ka_n)$;
    \item $(\alpha,\beta) = a_1b_1 + a_2b_2 + \cdots + a_n+b_n$;
\end{enumerate}
如果内积运算$(\alpha,\beta)=0$则称$\alpha$与$\beta$正交。

称$\sqrt{(\alpha,\alpha)} = \sqrt{a_1^2 +a_2^2 + \cdots + a_n^2}$为$\alpha$的长度,记为$\norm{\alpha}$

\section{向量的线性相关与线性无关、向量组的秩}
\subsection{向量的线性表出}
.
\begin{definition}
    $m$个$n$维向量$\alpha_1,\alpha_2,\cdots,\alpha_m$及$m$个实数$k_1,k_2,\cdots,k_n$
    称
    \[ k_1\alpha_1 + k_2\alpha_2 + \cdots + k_m\alpha_m \]
    是向量$\alpha_1,\alpha_2,\cdots,\alpha_n$的一个线性组合,称$k_1,k_2,\cdots,k_n$维这个线性组合的系数。
\end{definition}

\begin{definition}
    如果向量$\beta$能表示为$\alpha_1,\alpha_2,\cdots,\alpha_n$的线性组合,即存在一组数$k_1,k_2,\cdots,k_m$使得
    \[ \beta = k_1\alpha_1 + k_2\alpha_2 + \cdots + k_m\alpha_m \]
    则称向量$\beta$可以由$\alpha_1,\alpha_2,\cdots,\alpha_n$线性表出。
\end{definition}

这里一般有三种情况:
\begin{enumerate}[(1)]
    \item 线性表出且惟一;
    \item 线性表出不唯一;
    \item 不能线性表出。
\end{enumerate}
在研究线性表出的时候,通常转化为方程组来研究。
\begin{theorem}
    向量$\beta$可以由$\alpha_1,\alpha_2,\cdots,\alpha_m$线性表出$\iff$存在实数$k_1,k_2,\cdots,k_m$使得
    \[
        k_1\alpha_1 + k_2\alpha_2 +\cdots + k_m\alpha_m =
        \begin{pmatrix}
            \alpha_1 & \alpha_2 & \cdots & \alpha_m
        \end{pmatrix}
        \begin{pmatrix}
            k_1 \\k_2\\\vdots\\k_m
        \end{pmatrix}
        =AX=
        \beta
    \]
    其中
    \[
        A =
        \begin{pmatrix}
            \alpha_1 & \alpha_2 & \cdots & \alpha_m
        \end{pmatrix}
        \qquad
        X =
        \begin{pmatrix}
            k_1 & k_2 & \cdots & k_m
        \end{pmatrix}^\intercal
    \]
    即$AX=\beta$有解$\iff r(A)=r(\overline{A})$。
\end{theorem}

\begin{example}
    设$\alpha_1 = (1,2,3)^\intercal,\alpha_2 = (1,3,4)^\intercal, \alpha_3 = (2,-1,1)^\intercal,\beta = (2,5,t)^\intercal$,
    问$t$取何值时
    \begin{enumerate}[(1)]
        \item 向量$\beta$不能由$\alpha_1,\alpha_2,\alpha_3$线性表出;
        \item 向量$\beta$能由$\alpha_1,\alpha_2,\alpha_3$线性表出,并写出表达式。
    \end{enumerate}
\end{example}
\begin{solution}
    线性表出的表达式为
    \[
        \beta = x_1\alpha_1 + x_2 \alpha_2 + x_3\alpha_3
        = \alpha_1x_1 + \alpha_2x_2 + \alpha_3x_3
        =
        \begin{pmatrix}
            \alpha_1 & \alpha_2 & \alpha_3
        \end{pmatrix}
        \begin{pmatrix}
            x_1 \\x_2\\x_3
        \end{pmatrix}
    \]
    所以变为方程组$AX=b$,对增广矩阵进行行变换
    \[
        \overline{A}=
        \left(\begin{array}{ccc|c}
                1 & 1 & 2  & 2 \\
                2 & 3 & -1 & 5 \\
                3 & 4 & 1  & t \\
            \end{array}\right)
        \longrightarrow
        \left(\begin{array}{ccc|c}
                1 & 1 & 2  & 2   \\
                0 & 1 & -5 & 1   \\
                0 & 1 & -5 & t-6 \\
            \end{array}\right)
        \longrightarrow
        \left(\begin{array}{ccc|c}
                1 & 1 & 2  & 2   \\
                0 & 1 & -5 & 1   \\
                0 & 0 & 0  & t-7 \\
            \end{array}\right)
        \longrightarrow
        \left(\begin{array}{ccc|c}
                1 & 0 & 7  & 1   \\
                0 & 1 & -5 & 1   \\
                0 & 0 & 0  & t-7 \\
            \end{array}\right)
    \]
    所以当$t\neq 7$时,方程组无解,$\beta$无法被$\alpha_1,\alpha_2,\alpha_3$线性表出;
    当$t=7$时,方程组变为
    \[
        \systeme{
            x_1 + 7x_3 = 1,
            x_2 -5x_3 = 1
        }
    \]
    令自由变量$x_3=k$其中$k$为任意常数,则有
    \[
        \begin{cases}
            x_1 = 1 - 7k \\
            x_2 = 1+ 5k
        \end{cases}
    \]
    因此有线性表出
    \[ \beta = (1-7k)\alpha_1 + (1+5k)\alpha_2 + k\alpha_3 \]
\end{solution}
\begin{example}
    设$\alpha_1=(1+\lambda,1,1)^\intercal,\alpha_2=(1,1+\lambda,1)^\intercal,\alpha_3 = (1,1,1+\lambda)^\intercal,\beta=(0,\lambda,\lambda^2)^\intercal$
    当$\lambda$为何值时
    \begin{enumerate}[(1)]
        \item 向量$\beta$不能由$\alpha_1,\alpha_2,\alpha_3$线性表出;
        \item 向量$\beta$能由$\alpha_1,\alpha_2,\alpha_3$线性表出,并写出表达式。
    \end{enumerate}
\end{example}
\begin{solution}
    设$\beta = \alpha_1 x_1 + \alpha_2 x_2 + \alpha_3 x_3 = AX$,则对增广矩阵作初等行变换
    \[
        \overline{A}=
        \left(\begin{array}{ccc|c}
                1+\lambda & 1         & 1         & 0         \\
                1         & 1+\lambda & 1         & \lambda   \\
                1         & 1         & 1+\lambda & \lambda^2
            \end{array}\right)
        \xrightarrow{\text{减去第一行}}
        \left(\begin{array}{ccc|c}
                1+\lambda & 1       & 1       & 0         \\
                -\lambda  & \lambda & 0       & \lambda   \\
                -\lambda  & 0       & \lambda & \lambda^2
            \end{array}\right)
    \]
    此时得到一个爪形矩阵,对于爪形矩阵一帮通过消去左侧元素,或则时顶部元素。由于这里只能用行变换,故只能消除顶部元素
    而为了消去$1$,必须要加上后面两行的$-\frac{1}{\lambda}$,所以进行分类讨论:
    \begin{enumerate}[(1)]
        \item 当$\lambda=0$时,方程组变为$x_1+x_2+x_3 = 0$,则令$x_2 = s,x_3=t$,得$x_1 = -s-t$所以
              \[ \beta = (-s-t)\alpha_1 + s\alpha_2 + t\alpha_3 \]
        \item 当$\lambda\neq 0 $时
              \[
                  \overline{A}
                  \longrightarrow
                  \left(\begin{array}{ccc|c}
                          1+\lambda & 1 & 1 & 0       \\
                          -1        & 1 & 0 & 1       \\
                          -1        & 0 & 1 & \lambda
                      \end{array}\right)
                  \longrightarrow
                  \left(\begin{array}{ccc|c}
                          3+\lambda & 0 & 0 & -1-\lambda \\
                          -1        & 1 & 0 & 1          \\
                          -1        & 0 & 1 & \lambda
                      \end{array}\right)
              \]
              此时需要将第一行化为$1$,那么又进行分类讨论
              \subitem 当$\lambda=-3$时,$-1-\lambda=2\neq 0$,此时方程组无解,故不能线性表出
              \subitem 当$\lambda\neq-3$时,
              \[
                  \overline{A}
                  \longrightarrow
                  \left(\begin{array}{ccc|c}
                          1  & 0 & 0 & -\frac{1+\lambda}{3+\lambda} \\
                          -1 & 1 & 0 & 1                            \\
                          -1 & 0 & 1 & \lambda
                      \end{array}\right)
                  \longrightarrow
                  \left(\begin{array}{ccc|c}
                          1 & 0 & 0 & -\frac{1+\lambda}{3+\lambda}         \\
                          0 & 1 & 0 & 1 -\frac{1+\lambda}{3+\lambda}       \\
                          0 & 0 & 1 & \lambda -\frac{1+\lambda}{3+\lambda}
                      \end{array}\right)
              \]
              所以方程组得解为
              \[
                  \begin{dcases}
                      x_1 = -\frac{1+\lambda}{3+\lambda} \\
                      x_2 = \frac{2}{3+\lambda}          \\
                      x_3 = \frac{\lambda^2+2\lambda-1}{3+\lambda}
                  \end{dcases}
              \]
              因此有线性表出
              \[\beta = -\frac{1+\lambda}{3+\lambda}\alpha_1 + \frac{2}{3+\lambda}\alpha_2 + \frac{\lambda^2+2\lambda-1}{3+\lambda}\alpha_3 \]
    \end{enumerate}
    综上有
    \begin{enumerate}[(1)]
        \item $\lambda=0$时,$\beta = (-s-t)\alpha_1+s\alpha_2+t\alpha_3$其中$s,t$为任意常数;
        \item $\lambda=-3$时,$\beta$不能由$\alpha_1,\alpha_2,\alpha_3$线性表出;
        \item $\lambda\notin\{0,3\}$时,
              \[ \beta = -\frac{1+\lambda}{3+\lambda}\alpha_1 + \frac{2}{3+\lambda}\alpha_2 + \frac{\lambda^2+2\lambda-1}{3+\lambda}\alpha_3 \]
    \end{enumerate}
\end{solution}

\subsection{向量组的线性相关、线性无关}
.
\begin{theorem}
    如果向量组$(\alpha_1,\alpha_2,\cdots,\alpha_s)$和$(\beta_1,\beta_2,\cdots,\beta_t)$可以互相线性表示,则称这两个向量组等价。
\end{theorem}

\begin{theorem}
    如果向量组$A = (\alpha_1,\alpha_2,\cdots,\alpha_s)$可由$B = (\beta_1,\beta_2,\cdots,\beta_t)$线性表出,则$r(A)\leq r(B)$
    如果两个向量组等价,那么两个向量组得秩相同。
\end{theorem}

\begin{definition}
    对于$m$个$n$维向量$\alpha_1,\alpha_2,\cdots,\alpha_m$,若存在不全为$0$的实数$k_1,k_2,\cdots,k_m$使得
    \[ k_1\alpha_1 + k_2\alpha_2 + \cdots + k_m\alpha_m = 0 \]
    成立,则称向量组$\alpha_1,\alpha_2,\cdots,\alpha_m$线性相关,否则称其为线性无关。
\end{definition}
由线性相关的定义可以得到下面的定理
\begin{theorem}
    .
    \begin{enumerate}
        \item 包含零向量的向量组必定线性相关;
        \item 包含有成比例的向量的向量组必定线性相关;
        \item 若一个向量组线性无关,则它任意部分组必定线性无关。
    \end{enumerate}
\end{theorem}

根据方程组是否有解,可以得出下面定理
\begin{theorem}
    $n$维向量$\alpha_1,\alpha_2,\cdots,\alpha_m$\textcolor{red}{线性无关}的充要条件为
    \[AX=0\]
    只有零解。
    其中$A=(\alpha_1,\alpha_2,\cdots,\alpha_m),\alpha_i$为$n$维列向量。
\end{theorem}
特别的,两个向量线性相关,其向量的对应坐标分量成比例。

当向量组中向量的个数超过其向量维度时$(m>n)$,向量组必定线性相关,这是因为未知数的个数多于方程的个数,所以$AX=0$一定有非零解。
作为推论有
\begin{theorem}
    .
    \begin{enumerate}[(1)]
        \item $n$个$n$维向量$\alpha_1,\alpha_2,\cdots,\alpha_n$线性相关$\iff$行列式$|\alpha_1 \alpha_2 \cdots \alpha_n|=0$;
        \item $n+1$个$n$维向量必定线性相关。
    \end{enumerate}
\end{theorem}
\begin{example}
    设
    \[
        A =
        \begin{pmatrix}
            \alpha_1 & \alpha_2 & \cdots & \alpha_n
        \end{pmatrix}
        \qquad
        B =
        \begin{pmatrix}
            \beta_1 & \beta_2 & \cdots & \beta_n
        \end{pmatrix}
        \qquad
        AB =
        \begin{pmatrix}
            \gamma_1 & \gamma_2 & \cdots & \gamma_n
        \end{pmatrix}
    \]
    记向量组
    \begin{tasks}[label=(\Roman*),label-width = 2em](3)
        \task $\alpha_1,\alpha_2,\cdots,\alpha_n$;
        \task $\beta_1,\beta_2,\cdots,\beta_n$;
        \task $\gamma_1,\gamma_2,\cdots,\gamma_n$.
    \end{tasks}
    若向量组(III)线性相关,则
    \begin{tasks}[label=(\Alph*),label-width = 2em](2)
        \task (I),(II)均相关
        \task (I)或(II)至少有一个相关
        \task (I)必相关
        \task (II)必相关
    \end{tasks}
\end{example}
\begin{solution}
    由于(III)线性相关所以$|AB|=|A||B|=0$,因此$|A|,|B|$至少有一个为$0$,所以(I)或(II)至少有一个相关,选(B)
\end{solution}

\begin{theorem}
    如果$\alpha_1,\alpha_2,\cdots,\alpha_s$线性无关,$\alpha_1,\alpha_2,\cdots,\alpha_s,\beta$线性相关,
    则$\beta$可以由$\alpha_1,\alpha_2,\cdots,\alpha_s$线性表示,且表示方法唯一。
\end{theorem}
\begin{theorem}
    向量组$\alpha_1,\alpha_2,\cdots,\alpha_s(s\geq 2)$线性相关
    $\iff$\textcolor{red}{存在}$\alpha_i(1\leq i \leq s)$可由其余向量线性表出。
\end{theorem}
这里需要注意的是,不是任何一个$\alpha_i$都能由其余的向量表出。也就是说下面这种写法为错误写法
\textcolor{red}{
    \begin{align*}
        \because   \, & \alpha_1,\alpha_2,\alpha_3\text{线性相关}              \\
        \therefore \, & \alpha_3 \text{可以由}\alpha_1,\alpha_2\text{线性表出}
    \end{align*}
}
这里只能说有存在一个向量可以被其余向量线性表出,但是是否是$\alpha_3$是不确定的。
同理,下面这种写法也是错误的
\textcolor{red}{
    \begin{align*}
        \because   \, & \alpha_3 \text{不能由}\alpha_1,\alpha_2 \text{线性表出} \\
        \therefore \, & \alpha_1,\alpha_2,\alpha_3 \text{线性无关}
    \end{align*}
}
这都是因为线性相关的向量组,其线性系数可以为零,但不是全为零。举个简单的例子
\[
    \alpha_1 =
    \begin{pmatrix}
        1 \\0
    \end{pmatrix}
    \qquad
    \alpha_2 =
    \begin{pmatrix}
        2 \\0
    \end{pmatrix}
    \qquad
    \alpha_3 =
    \begin{pmatrix}
        0 \\1
    \end{pmatrix}
\]
显然这三个向量线性相关,但$\alpha_3$无法由$\alpha_1,\alpha_2$线性表出。即
\[ k_1\alpha_1 + k_2\alpha_2 + k_3\alpha_3 = 0 \]
时,$k_3=0$且唯一,刚好就是无法由其余向量表示的$\alpha_3$

\begin{theorem}
    如果$\alpha_1,\alpha_2,\cdots,\alpha_s$可由$\beta_1,\beta_2,\cdots,\beta_t$线性表出,且$s>t$,
    则$\alpha_1,\alpha_2,\cdots,\alpha_s$必定线性相关。
\end{theorem}
这里假设$\beta_1,\beta_2,\cdots,\beta_t$线性无关,那么可以将其看作$t$维空间的一组基,
而$s>t$,则必然有$\alpha_1,\alpha_2,\cdots,\alpha_s$线性相关;而$\beta_1,\beta_2,\cdots,\beta_t$线性相关,根据传递性,
$\alpha_1,\alpha_2,\cdots,\alpha_s$也线性相关。

作为推论有
\begin{theorem}
    如果$\alpha_1,\alpha_2,\cdots,\alpha_s$线性无关,且$\alpha_1,\alpha_2,\cdots,\alpha_s$可以由$\beta_1,\beta_2,\cdots,\beta_t$线性表出,
    则$s\leq t$
\end{theorem}
与上个定理解释用同样的思想,$\alpha$向量组线性无关,则可以将其看作$s$维空间的一组基,而$\alpha$可以由$\beta$线性表出,
那么$\beta$中的向量要么是在同维空间,要么就是在更高维的空间,所以会有$s\leq t$。

\subsection{向量组的秩}
.
\begin{definition}
    在向量组$\alpha_1,\alpha_2,\cdots,\alpha_s$中,如果存在$r$个向量$\alpha_{i1},\alpha_{i2},\cdots,\alpha_{ir}$线性无关,
    且再添加任意一个$\alpha_j(j=1,2,\cdots,s)$向量组$\alpha_{i1},\alpha_{i2},\cdots,\alpha_{ir},\alpha_j$变为线性相关,
    则称$\alpha_{i1},\alpha_{i2},\cdots,\alpha_{ir}$为原向量组的一个极大线性无关组。
\end{definition}

对于一个向量组,可能存在多个极大线性无关组,也可能只有一个,特别的,当只有零向量的向量组没有极大线性无关组。
\begin{theorem}
    对于一个向量组,其所有极大线性无关组中的向量个数相同。
\end{theorem}
\begin{proof}
    假设向量组$\alpha_1,\alpha_2,\cdots,\alpha_s$,有两个极大线性无关组$\alpha_{i1},\alpha_{i2},\cdots,\alpha_{ir}$和$\alpha_{j1},\alpha_{j2},\cdots,\alpha_{jd}$
    其个数分别为$r,d$,则有
    \[ \alpha_{i1},\alpha_{i2},\cdots,\alpha_{ir},\alpha_{jk}\quad (k=1,2,\cdots,d) \]
    线性相关。同时由于$\alpha_{i1},\alpha_{i2},\cdots,\alpha_{ir}$线性无关,因此$\alpha_{jk}$能由$\alpha_{i1},\alpha_{i2},\cdots,\alpha_{ir}$线性表出。

    所以向量组$\alpha_{j1},\alpha_{j2},\cdots,\alpha_{jd}$能由向量组$\alpha_{i1},\alpha_{i2},\cdots,\alpha_{ir}$线性表出,因此$s\geq d$;

    反过来,同理有向量组$\alpha_{i1},\alpha_{i2},\cdots,\alpha_{ir}$能由向量组$\alpha_{j1},\alpha_{j2},\cdots,\alpha_{jd}$线性表出,因此$d\geq s$;

    那么有$d=s$,所以所有的极大线性无关组的向量个数相等。
\end{proof}

\begin{definition}
    向量组$\alpha_1,\alpha_2,\cdots,\alpha_s$的极大线性无关组所含的向量的个数$r$称为向量组的秩,记
    \[ r(\alpha_1,\alpha_2,\cdots,\alpha_s) = r \]
    只有零向量的向量组,其秩为$0$
\end{definition}
对于一个向量组,找出其极大线性无关组的方法为下面两个步骤:
\begin{enumerate}
    \item 将列向量组写成矩阵;
    \item 进行初等行变换,化为行最简矩阵;
    \item 找到只含一个$1$的所有列,其对应位置的所有向量,组成极大线性无关组。
\end{enumerate}
其原理为
\[
    PA = B
    \implies P
    \begin{pmatrix}
        \alpha_1 & \cdots & \alpha_n
    \end{pmatrix}
    =
    \begin{pmatrix}
        P\alpha_1 & \cdots & P\alpha_n
    \end{pmatrix}
    =
    \begin{pmatrix}
        \beta_1 & \cdots & \beta_n
    \end{pmatrix}
\]
其中$A$为原列向量组组成的矩阵,$P$为一系列初等行变换的变换矩阵,$B$为行最简矩阵。
所以有
\begin{align*}
    k_1\alpha_1 + \cdots + k_n\alpha_n = 0
     & \iff P(k_1\alpha_1 + \cdots + k_n\alpha_n) = 0 \\
     & \iff k_1P\alpha_1 + \cdots + k_nP\alpha_n = 0  \\
     & \iff k_1\beta_1 + \cdots + k_n\beta_n = 0
\end{align*}
因此两个向量组等价,且系数对应位置相同,则$\beta$向量组的极大线性无关组所对应得位置就是$\alpha$向量组的极大线性无关组对应位置。
那么显然$B$中只含一个$1$的列,就是极大线性无关组对应的位置。系数一样,所以表出形式都是一样。

\begin{example}
    设向量组$\alpha_1 = (1,-1,2,4)^\intercal, \alpha_2 = (0,3,1,2)^\intercal, \alpha_3 = (3,0,7,14)^\intercal, \alpha_4 = (1,-2,2,0)^\intercal,\alpha_5 = (2,-1,5,2)^\intercal$
    求其向量组的一个极大线性无关组。
\end{example}
\begin{solution}
    \begin{align*}
        \begin{pmatrix}
            \alpha_1 & \alpha_2 & \alpha_3 & \alpha_4 & \alpha_5
        \end{pmatrix}
         & =
        \begin{pmatrix}
            1  & 0 & 3  & 1  & 2  \\
            -1 & 3 & 0  & -2 & -1 \\
            2  & 1 & 7  & 2  & 5  \\
            4  & 2 & 14 & 0  & 2
        \end{pmatrix}
        \longrightarrow
        \begin{pmatrix}
            1 & 0 & 3 & 1  & 2  \\
            0 & 3 & 3 & -1 & 1  \\
            0 & 1 & 1 & 0  & 1  \\
            0 & 2 & 2 & -4 & -6
        \end{pmatrix}
        \longrightarrow
        \begin{pmatrix}
            1 & 0 & 3 & 1  & 2  \\
            0 & 1 & 1 & 0  & 1  \\
            0 & 3 & 3 & -1 & 1  \\
            0 & 1 & 1 & -2 & -3
        \end{pmatrix} \\
         & \longrightarrow
        \begin{pmatrix}
            1 & 0 & 3 & 1  & 2  \\
            0 & 1 & 1 & 0  & 1  \\
            0 & 0 & 0 & -1 & -2 \\
            0 & 0 & 0 & -2 & -4
        \end{pmatrix}
        \longrightarrow
        \begin{pmatrix}
            1 & 0 & 3 & 1 & 2 \\
            0 & 1 & 1 & 0 & 1 \\
            0 & 0 & 0 & 1 & 2 \\
            0 & 0 & 0 & 0 & 0
        \end{pmatrix}
        \longrightarrow
        \begin{pmatrix}
            1 & 0 & 3 & 0 & 0 \\
            0 & 1 & 1 & 0 & 1 \\
            0 & 0 & 0 & 1 & 2 \\
            0 & 0 & 0 & 0 & 0
        \end{pmatrix}
    \end{align*}
    其中第$1,2,4$列只含有一个$1$,故原向量组的极大线性无关组为$\alpha_1,\alpha_2,\alpha_4$,
    此时其余列中的元素即为由极大线性无关组线性表出的系数,即$\beta_3 = 3\beta_1 + \beta_2, \beta_5 = \beta_2 + 2\beta_4$,
    所以原向量组对应的表出也是相同的$\alpha_3 = 3\alpha_1 + \alpha_2, \alpha_5 = \alpha_2 + 2\alpha_4$
\end{solution}

\section{施密特正交化}
设$\alpha_1,\alpha_2,\alpha_3$线性无关,则施密特正交化为下面几个步骤:
\begin{enumerate}[(1)]
    \item 正交化
          \begin{align*}
              \beta_{\textcolor{red}{1}}    & = \alpha_{\textcolor{red}{1}}                                                                                                                                                        \\
              \beta_{\textcolor{orange}{2}} & = \alpha_{\textcolor{orange}{2}} - \frac{(\alpha_{\textcolor{orange}{2}},\beta_1)}{(\beta_1, \beta_1)}\beta_1                                                                        \\
              \beta_{\textcolor{blue}{3}}   & = \alpha_{\textcolor{blue}{3}} - \frac{(\alpha_{\textcolor{blue}{3}},\beta_1)}{(\beta_1, \beta_1)}\beta_1 - \frac{(\alpha_{\textcolor{blue}{3}},\beta_2)}{(\beta_2, \beta_2)}\beta_2
          \end{align*}
    \item 单位化
          \[ \gamma_1 = \frac{\beta_1}{\norm{\beta_1}},\qquad \gamma_2 = \frac{\beta_2}{\norm{\beta_2}},\qquad \gamma_3 = \frac{\beta_3}{\norm{\beta_3}} \]
\end{enumerate}

\begin{example}
    设$\alpha_1=(1,3,0)^\intercal, \alpha_2 = (-2,2,1)^\intercal$
    将其正交单位化。
\end{example}
\begin{solution}
    首先正交化
    \begin{align*}
        \beta_1 & = \alpha_1   =   (1,3,0)^\intercal                                                                                                                       \\
        \beta_2 & = \alpha_2 - \frac{(\alpha_2,\beta_1)}{(\beta_1, \beta_1)}\beta_1  = (-2,2,1)^\intercal - \frac{4}{10}(1,3,0)^\intercal = \frac{1}{5}(-12,4,5)^\intercal
    \end{align*}
    然后单位化
    \[
        \beta_1 = \frac{1}{\sqrt{10}}(1,3,0)^\intercal,\qquad \beta_2 = \frac{1}{\sqrt{185}}(-12,4,5)^\intercal
    \]
\end{solution}