\part{矩阵代数}
由于矩阵代数中的每个知识点都有相互联系,故此部分无法顺序地学习,应该往复看多几遍,同时学习过程中应该相互联系。
\section{矩阵的概念和运算}
形如
\[
    \begin{pmatrix}
        a_{11} & a_{12} & \cdots & a_{1n} \\
        a_{21} & a_{22} & \cdots & a_{2n} \\
        \vdots & \vdots & \ddots & \vdots \\
        a_{m1} & a_{m2} & \cdots & a_{mn}
    \end{pmatrix}
\]
成为$m$行$n$列的矩阵,记为$\mat{A}_{m\times n}$,简记为$\mat{A}$,当$m=n$时,成为\textbf{\textsf{方阵}}。
当矩阵内的所用元素均为$0$,则矩阵称为零矩阵,简记为$0$

如果$\mat{A}$和$\mat{B}$都是$m\times n$的矩阵,则称$\mat{A},\mat{\mat{B}}$为\textbf{\textsf{同型矩阵}}。进一步的,当$a_{ij}=b_{ij}$,
则有$\mat{A}=\mat{B}$

矩阵有如下的运算:
\begin{enumerate}[(1)]
    \item 矩阵加法$\mat{A}+\mat{B}=(a_{ij}+b_{ij})$;
    \item 矩阵数乘$k\mat{A} = (ka_{ij})$;
\end{enumerate}

同时有加法的交换律和加法的结合律,数乘的交换律和结合律:
\begin{enumerate}[(1)]
    \item $\mat{A}+\mat{B}=\mat{B}+\mat{A}$;
    \item $(\mat{A}+\mat{B})+\mat{C}=\mat{A}+(\mat{B}+\mat{C})=\mat{A}+\mat{B}+\mat{C}$;
    \item $k(m\mat{A}) = m(k\mat{A}) = (mk)\mat{A}$;
    \item $(k+m)\mat{A} = k\mat{A} + m\mat{A}$;
    \item $k(\mat{A}+\mat{B}) = k\mat{A} + k\mat{B}$;
    \item $1\mat{A}=\mat{A}, 0\mat{A} = 0$
\end{enumerate}

\subsection{矩阵的乘法}
矩阵的乘法与数的乘法有所不同,首要就是其不满足交换律。
\begin{definition}
    设矩阵$\mat{A}=(a_{ij})$是一个$m\times n$矩阵,$\mat{B}=(b_{ij})$是一个$n\times p$矩阵,
    则定义运算
    \[ \mat{AB} = \mat{C} \]
    其中$\mat{C}=(c_{ij})$为$m\times p$矩阵,且满足
    \[ c_{ij} = a_{i1}b_{1j} + a_{a2}b_{2j} + \cdots + a_{an}b_{nj} = a_i\vdot b_j \]
    其中$a_i$为$\mat{A}$的第$i$行行向量,$b_j$为$\mat{B}$的第$j$列列向量。
\end{definition}

对于矩阵的乘法有结合律和分配律
\begin{enumerate}[(1)]
    \item $(\mat{AB})\mat{C}=\mat{A}(\mat{BC})=\mat{ABC}$;
    \item $\mat{A}(\mat{B}+\mat{C})=\mat{AB}+\mat{AC}$;
    \item $(\mat{A}+\mat{B})\mat{C}=\mat{AC}+\mat{BC}$;
    \item $\mat{AA}=\mat{A}^2$
\end{enumerate}

需要重视的是
\begin{enumerate}[(1)]
    \item \textcolor{red}{$\mat{AB} \neq \mat{BA}$},即矩阵乘法不满足交换律;
    \item \textcolor{red}{$\mat{AB} \not\implies \mat{A}=\mat{0} \text{或} \mat{B}=\mat{0}$};
    \item \textcolor{red}{$\mat{AB}=\mat{AC},\mat{A}\neq \mat{0} \not\implies \mat{B} = \mat{C}$}.
\end{enumerate}

\begin{definition}
    若$\mat{A},\mat{B}$为$n$阶矩阵,且满足
    \[ \mat{AB}=\mat{BA} \]
    则称$\mat{A},\mat{B}$可交换。
\end{definition}

\begin{theorem}
    \label{th:行列式的乘法公式}
    (行列式的乘法公式)
    如果$\mat{A},\mat{B}$为$n$阶矩阵,则有
    \[ |\mat{AB}| = |\mat{A}|\cdot|\mat{B}| \]
\end{theorem}

若将$\mat{A}$的行列互换,所得矩阵称为$\mat{A}$的\textbf{\textsf{转置}},记为$\mat{A}^\intercal$,其运算法则如下
\begin{enumerate}[(1)]
    \item $(\mat{A}+\mat{B})^\intercal = \mat{A}^\intercal + \mat{B}^\intercal$;
    \item $(k\mat{A})^\intercal = k\mat{A}^\intercal$;
    \item $(\mat{AB})^\intercal = \mat{B}^\intercal \mat{A}^\intercal$;
    \item $(\mat{A}^\intercal)^\intercal=\mat{A}$
\end{enumerate}
若$\mat{A}^\intercal = \mat{A}$,则称$\mat{A}$为\textbf{\textsf{对称矩阵}},若$\mat{A}^\intercal = - \mat{A}$,则称$\mat{A}$为\textbf{\textsf{反对称矩阵}}。
其中反对称矩阵的对角线一定是$0$。


设$\bm{\alpha},\bm{\beta}$为$n$维列向量,可以将其看作$n\times 1$的矩阵,则有
\begin{align}
    \bm{\alpha}\bm{\beta}^\intercal  & =
    \begin{pmatrix}
        a_1b_1 & \cdots & a_1b_n \\
        \vdots & \ddots & \vdots \\
        a_nb_1 & \cdots & a_nb_1
    \end{pmatrix}                                           \\
    \bm{\alpha}^\intercal\bm{\beta}  & = a_1b_1+a_2b_2 + \cdots + a_nb_n \\
    \bm{\alpha}\bm{\alpha}^\intercal & =
    \begin{pmatrix}
        a_1a_1 & \cdots & a_1a_n \\
        \vdots & \ddots & \vdots \\
        a_na_1 & \cdots & a_na_1
    \end{pmatrix}\qquad\text{(对称矩阵)}                    \\
    \bm{\alpha}^\intercal\bm{\alpha} & = a_1^2 + a_2^2 + \cdots a_n^2
\end{align}
由于$\bm{\alpha},\bm{\beta}$是列向量,故必有
\begin{equation}
    r(\bm{\alpha}\bm{\beta}^\intercal) \leq 1
\end{equation}

\begin{example}
    已知列向量$\bm{\alpha} = (1,2,3),\bm{\beta}=(1,\frac{1}{2},\frac{1}{3}),\mat{A}=\bm{\alpha}^\intercal\bm{\beta}$,
    求$\mat{A}^n$
\end{example}
\begin{solution}
    \[
        \mat{A}^n = (\bm{\alpha}^\intercal\bm{\beta})(\bm{\alpha}^\intercal\bm{\beta})\cdots(\bm{\alpha}^\intercal\bm{\beta})
        =\bm{\alpha}^\intercal(\bm{\beta}\bm{\alpha}^\intercal)\cdots(\bm{\beta}\bm{\alpha}^\intercal)\bm{\beta}
        =\bm{\alpha}^\intercal(3)^{n-1}\bm{\beta}
        =3^{n-1}
        \begin{pmatrix}
            1 & \frac{1}{2} & \frac{1}{3} \\
            2 & 1           & \frac{2}{3} \\
            3 & \frac{3}{2} & 1
        \end{pmatrix}
    \]
\end{solution}

\begin{example}
    设$\mat{A}=\mat{E}-\bm{\bm{\xi}}\bm{\bm{\xi}}^\intercal$,其中$\bm{\xi}$是$n$维非$0$列向量,证明
    $\mat{A}^2=\mat{A}\iff \bm{\xi}\bm{\xi}^\intercal = 1$
\end{example}
\begin{proof}
    \begin{align*}
        \mat{A}^2-\mat{A} & = \mat{A}(\mat{A}-\mat{E})                                                             \\
                          & = (\mat{E}-\bm{\xi}\bm{\xi}^\intercal)(-\bm{\xi}\bm{\xi}^\intercal)                    \\
                          & = -\bm{\xi}\bm{\xi}^\intercal + \bm{\xi}\bm{\xi}^\intercal\bm{\xi}\bm{\xi}^\intercal   \\
                          & = -\bm{\xi}\bm{\xi}^\intercal + (\bm{\xi}^\intercal\bm{\xi})\bm{\xi}\bm{\xi}^\intercal \\
                          & = (\bm{\xi}^\intercal\bm{\xi}-1)\bm{\xi}\bm{\xi}^\intercal                             \\
    \end{align*}
    由于$\bm{\xi}$是$n$维非$0$列向量,故$\bm{\xi}\bm{\xi}^\intercal\neq 0$,所以有
    \[\bm{\xi}^\intercal\bm{\xi} = 1 \iff \mat{A}^2 = \mat{A} \]
\end{proof}

形如
\[
    \begin{pmatrix}
        1 &        &   \\
          & \ddots &   \\
          &        & 1
    \end{pmatrix}
\]
称为\textbf{\textsf{单位矩阵}},简记为$E$,其在矩阵的乘法中起到$1$的作用,即
\[ \mat{AE} = \mat{EA} = \mat{A} \]
其中$\mat{A},\mat{E}$均为$n$阶方阵。

形如
\[
    \mat{D}=
    \begin{pmatrix}
        a_1 &        &     \\
            & \ddots &     \\
            &        & a_n
    \end{pmatrix}
\]
称为\textbf{\textsf{对角矩阵}},对角矩阵的乘法可满足交换律,即
\[
    \mat{D_1D_2} = \mat{D_2D_1}
    =
    \begin{pmatrix}
        a_1b_1 &        &        \\
               & \ddots &        \\
               &        & a_nb_n
    \end{pmatrix}
\]
同时有
\begin{equation}
    \begin{pmatrix}
        a_1 &        &     \\
            & \ddots &     \\
            &        & a_m
    \end{pmatrix}^n
    =
    \begin{pmatrix}
        a_1^n &        &       \\
              & \ddots &       \\
              &        & a_m^n
    \end{pmatrix}
\end{equation}
此通常用于求复杂矩阵的$n$次方,常出现于特征值相关的题目中。

对于对角矩阵的逆,可以通过下面的性质得出:
\begin{equation}
    \begin{pmatrix}
        a_1 &        &     \\
            & \ddots &     \\
            &        & a_n
    \end{pmatrix}
    \begin{pmatrix}
        \frac{1}{a_1} &        &               \\
                      & \ddots &               \\
                      &        & \frac{1}{a_n}
    \end{pmatrix}
    = \mat{E}
\end{equation}

\subsection{矩阵的秩}
在引入矩阵的秩的定义前,先说明矩阵的子式概念。
\begin{definition}
    对于矩阵$\mat{A}_{m\times n}$,任取$k$行$k$列,所得交点元素在相对位置不变的情况下组成的行列式,称为$\mat{A}_{m \times n}$的$k$阶子式。
\end{definition}

\begin{definition}
    对于矩阵$\mat{A}_{m\times n}$,如果\textcolor{red}{存在}$r$阶子式$D\neq 0$,且所有$r+1$阶子式(如果存在)全为$0$,则称矩阵$\mat{A}$的秩为$r$,
    记作$r(\mat{A})=r$,并且规定零矩阵的秩为$0$
\end{definition}
可以通过这个定理可以缩小求秩的范围。
\begin{example}
    设
    \[
        \mat{A} =
        \begin{pmatrix}
            1 & 2 & 3 \\
            2 & 4 & t \\
            3 & 5 & 8
        \end{pmatrix}
    \]
    求$r(\mat{A})$
\end{example}
\begin{solution}
    由于矩阵左下角的$2$阶子式
    \[
        \begin{vmatrix}
            2 & 4 \\
            3 & 5
        \end{vmatrix}
        \neq 0
    \]
    所以$2\leq r(\mat{A}) \leq 3$。注意到$t=6$时,上下两行成比例,此时$|\mat{A}|=0,r(\mat{A})=2$,
    那么当$t\neq 6$时,$r(\mat{A})=3$
\end{solution}

\begin{theorem}
    经过初等变换的矩阵,其秩不变。
\end{theorem}
由这个定理可以将矩阵进行初等变换,得到阶梯矩阵或最简矩阵,主元的个数就是矩阵的秩。秩相关的公式在附录\ref{sec:矩阵的秩相关公式}给出
\begin{theorem}
    矩阵的秩、矩阵列向量组的秩、矩阵行向量组的秩,三秩相等。
\end{theorem}
\begin{theorem}
    $n$阶矩阵$\mat{A}$的行列式$|\mat{A}|\neq 0 \iff r(\mat{A}) = n$,反过来$|\mat{A}|=0\iff r(\mat{A})< n$
\end{theorem}
\begin{example}
    \[
        \mat{A} =
        \begin{pmatrix}
            k & 1 & 1 & 1 \\
            1 & k & 1 & 1 \\
            1 & 1 & k & 1 \\
            1 & 1 & 1 & k
        \end{pmatrix}
    \]
    若$r(\mat{A})=3$,求$k$
\end{example}
\begin{solution}
    由于$r(\mat{A})=3<4$,所以$|\mat{A}|=0$,则
    \[
        |\mat{A}| =
        \begin{vmatrix}
            k & 1 & 1 & 1 \\
            1 & k & 1 & 1 \\
            1 & 1 & k & 1 \\
            1 & 1 & 1 & k
        \end{vmatrix}
        =(k+3)
        \begin{vmatrix}
            1 & 1 & 1 & 1 \\
            1 & k & 1 & 1 \\
            1 & 1 & k & 1 \\
            1 & 1 & 1 & k
        \end{vmatrix}
        =(k+3)
        \begin{vmatrix}
            1 & 1   & 1   & 1   \\
            0 & k-1 & 0   & 0   \\
            0 & 0   & k-1 & 0   \\
            0 & 0   & 0   & k-1
        \end{vmatrix}
        =(k+3)(k-1)^3
    \]
    所以$k=-3$或$k=1$,当$k=1$时$r(\mat{A})=1$。当$k=-3$时
    \[
        \begin{vmatrix}
            -3 & 1  & 1  \\
            1  & -3 & 1  \\
            1  & 1  & -3 \\
        \end{vmatrix}
        =
        \begin{vmatrix}
            -3 & 1  & 1  \\
            4  & -4 & 0  \\
            4  & 0  & -4 \\
        \end{vmatrix}
        =
        \begin{vmatrix}
            -1 & 1  & 1  \\
            0  & -4 & 0  \\
            0  & 0  & -4 \\
        \end{vmatrix}
        =-16\neq 0
    \]
    所以这个三阶子式不等于$0$,因此$k=-3$
\end{solution}
利用初等变换,有如下解法
\begin{solution}
    \[
        \mat{A} =
        \begin{pmatrix}
            k & 1 & 1 & 1 \\
            1 & k & 1 & 1 \\
            1 & 1 & k & 1 \\
            1 & 1 & 1 & k
        \end{pmatrix}
        \longrightarrow
        \begin{pmatrix}
            k   & 1   & 1   & 1   \\
            1-k & k-1 & 0   & 0   \\
            1-k & 0   & k-1 & 0   \\
            1-k & 0   & 0   & k-1
        \end{pmatrix}
        \longrightarrow
        \begin{pmatrix}
            k+3 & 1   & 1   & 1   \\
            0   & k-1 & 0   & 0   \\
            0   & 0   & k-1 & 0   \\
            0   & 0   & 0   & k-1
        \end{pmatrix}
    \]
    当$k=1$时,$r(\mat{A})=1$,当$k=-3$时,列主元有三个,故$r(\mat{A})=3$,所以$k=-3$
\end{solution}

\section{可逆矩阵}
在研究逆矩阵前,需要引入伴随矩阵的概念。
\subsection{伴随矩阵}
\begin{definition}
    设$\mat{A}$为$n$阶矩阵,则$\mat{A}$的伴随矩阵为
    \[
        \mat{A}^* =
        \begin{pmatrix}
            A_{\textcolor{red}{1}1} & A_{\textcolor{red}{2}1} & \cdots & A_{\textcolor{red}{n}1} \\
            A_{\textcolor{red}{1}2} & A_{\textcolor{red}{2}2} & \cdots & A_{\textcolor{red}{n}2} \\
            \vdots                  & \vdots                  & \ddots & \vdots                  \\
            A_{\textcolor{red}{1}n} & A_{\textcolor{red}{2}n} & \cdots & A_{\textcolor{red}{n}n} \\
        \end{pmatrix}
    \]
    其中$A_{ij}$为$\mat{A}$中$a_{ij}$的代数余子式。此处要注意\textcolor{red}{下标}。
\end{definition}
对于伴随矩阵有下面的等式
\begin{equation}
    \mat{A}\mat{A}^* = \mat{A}^*\mat{A} = |\mat{A}|\mat{E}
\end{equation}
其证明如下
\begin{proof}
    下面只讨论$n=2$的情况,对于$n$为任意正整数的情况同理。
    利用行列式的展开公式,有
    \begin{align*}
        \mat{A}\mat{A}^* & =
        \begin{pmatrix}
            a_{11} & a_{12} \\
            a_{21} & a_{22}
        \end{pmatrix}
        \begin{pmatrix}
            A_{11} & A_{21} \\
            A_{12} & A_{22}
        \end{pmatrix}            \\
                         & =
        \begin{pmatrix}
            a_{11}A_{11} + a_{12}A_{12} &  & a_{11}A_{21} + a_{12}A_{22} \\
            a_{21}A_{11} + a_{22}A_{12} &  & a_{21}A_{21} + a_{22}A_{22}
        \end{pmatrix}            \\
                         & =
        \begin{pmatrix}
            |\mat{A}| & 0         \\
            0         & |\mat{A}|
        \end{pmatrix}            \\
                         & = |\mat{A}|\mat{E}
    \end{align*}
\end{proof}

对于二阶矩阵的伴随矩阵,可以将主对角线互换,副对角线变号,即
\begin{equation}
    \begin{pmatrix}
        a & b \\
        c & d
    \end{pmatrix}^*
    =
    \begin{pmatrix}
        d  & -b \\
        -c & a
    \end{pmatrix}
\end{equation}
由此可以很简单地通过定义法求出二阶矩阵的逆矩阵。

有关伴随矩阵的公式在附录\ref{sec:伴随矩阵的公式}中给出。
\subsubsection{伴随矩阵的秩与原矩阵的秩}
当原矩阵可逆时,伴随矩阵也可逆,那么此时$r(\mat{A})=r(\mat{A}^*)$,而需要着重讨论的是,当原矩阵不是满秩时,两者之间的关系。

\begin{enumerate}[(1)]
    \item $r(\mat{A})=n$,则$r(\mat{A}^*)=n$;
    \item $r(\mat{A})=n-1$,则$r(\mat{A}^*)=1$;
    \item $r(\mat{A})<n-1$,则$r(\mat{A}^*)=0$。
\end{enumerate}
\begin{proof}
    第一种情况显然是成立的。对于后面两种情况,讨论如下。
    \begin{enumerate}[(1)]
        \item $r(\mat{A})=n$,则$\mat{A},\mat{A}^*$可逆,所以$r(\mat{A}^*)=n$;
        \item 当$r(\mat{A})=n-1$时,行列式$|\mat{A}|=0$,当矩阵中存在$n-1$阶不为$0$的子式,故存在代数余子式$A_{ij}\neq 0$,则显然有$r(\mat{A}^*)\geq 1$,
              又$\mat{A}^*\mat{A} = |\mat{A}|\mat{E} = 0$,所以$r(\mat{A})+r(\mat{A}^*)-n\leq r(\mat{A}^*A) = 0$,即$r(\mat{A}^*)\leq n - r(\mat{A}) = 1$,因此$r(\mat{A}^*)=1$;
        \item 当$r(\mat{A})<n-1$时,则任意$n-1$阶子式等于$0$,即所有代数余子式等于$0$,所以$\mat{A}^*=0$,则有$r(\mat{A}^*)=0$。
    \end{enumerate}
\end{proof}

\begin{example}
    设$\mat{A}$为$n$阶矩阵,且满足$\sum_j^n a_{ij} = 0$,求$r(\mat{A}^*),\mat{A}^*$
\end{example}
\begin{solution}
    由于$\sum_j^n a_{ij} = 0$,故方程组$AX=0$有一个非零解$(1,1,\cdots,1)^\intercal$,所以$|\mat{A}|=0$,那么有$r(\mat{A})\leq n-1$
    \begin{enumerate}[(1)]
        \item 当$r(\mat{A})=n-1$时,$r(\mat{A}^*)=1$,利用列向量的秩不超过$1$,设任意非零列向量$\bm{\alpha},\bm{\beta}$,
              则有$\bm{\alpha}\bm{\beta}^\intercal\neq 0$,且$r(\bm{\alpha}\bm{\beta}^\intercal)=1$,此时只需令$\mat{A}^*=\bm{\alpha}\bm{\beta}^\intercal$即可。
        \item 当$r(\mat{A})<n-1$,则$r(\mat{A}^*)=0,\mat{A}=0$
    \end{enumerate}
\end{solution}

\subsection{可逆矩阵}
.
\begin{definition}
    对于$n$阶矩阵$\mat{A}$,如果存在$n$阶矩阵$\mat{B}$,使得
    \[ \mat{AB} = \mat{BA} = \mat{E} \]
    则称矩阵$\mat{A}$是可逆的,称$\mat{B}$是$\mat{A}$的逆矩阵,
    记$\mat{A}$的逆矩阵为$\mat{A}^{-1}$,且$\mat{A}^{-1}$惟一。其中$\mat{E}$为$n$阶单位矩阵。
\end{definition}
在有关逆矩阵的证明中,常需要用单位矩阵的恒等变形,例如$\mat{A}=\mat{AE}=\mat{A}(\mat{B}\mat{B}^{-1})$。
\begin{example}
    若矩阵$\mat{A}$可逆,证明$\mat{A}$的逆矩阵唯一。
\end{example}
\begin{proof}
    假设$\mat{A}$的逆矩阵不唯一,记$\mat{B},\mat{C}$分别为$\mat{A}$的逆矩阵。则有$\mat{AB}=\mat{BA}=\mat{E}, \mat{AC}=\mat{CA}=\mat{E}$。
    因此有
    \[ \mat{B} = \mat{BE} = \mat{BAC} = \mat{EC} = \mat{C} \]
    显然与假设矛盾,故假设$\mat{A}$的逆矩阵唯一。
\end{proof}

利用伴随矩阵可以计算逆矩阵:
\begin{theorem}
    若$n$阶矩阵$\mat{A}$的行列式不为$0$,则$\mat{A}$的逆矩阵为
    \[
        \mat{A}^{-1} = \frac{1}{|\mat{A}|}\mat{A}^*
    \]
\end{theorem}

\begin{theorem}
    $n$阶矩阵$\mat{A}$可逆$\iff |\mat{A}|\neq 0$。
\end{theorem}
\begin{proof}
    (充分性)
    \[ \mat{A}\mat{A}^{-1} = \mat{E} \implies |\mat{A}\mat{A}^{-1}| = |\mat{A}|\cdot|\mat{A}^{-1}| = |\mat{E}| = 1 \]
    所以$|\mat{A}|\neq 0$。
    (必要性)
    \[ \mat{A}\mat{A}^* = |\mat{A}|\mat{E} \]
    由于$|\mat{A}|\neq 0$
    所以$\mat{A} \frac{\mat{A}^*}{|\mat{A}|} = \mat{E}$,所以$\mat{A}$可逆。
\end{proof}

可逆矩阵的一些性质在附录\ref{sec:可逆矩阵的性质}给出。

\subsection{逆矩阵的求解}
如果矩阵$\mat{A}$可逆,一般有如下几种方法求逆矩阵。
\begin{enumerate}[(1)]
    \item 定义法;
    \item 伴随矩阵法;
    \item 初等行变换
          \[ (\mat{A}|\mat{E}) \xrightarrow{\text{由上往下}} (\text{上三角矩阵}|*)\xrightarrow{\text{由下往上}} (\text{对角矩阵}|*)\xrightarrow{\text{乘系数}}(\mat{E}|\mat{A}^{-1}) \]
    \item 分块法
          \[
              \begin{pmatrix}
                  \mat{A} & 0       \\
                  0       & \mat{B}
              \end{pmatrix}^{-1}
              =
              \begin{pmatrix}
                  \mat{A}^{-1} & 0            \\
                  0            & \mat{B}^{-1}
              \end{pmatrix}
          \]
          \[
              \begin{pmatrix}
                  0       & \mat{A} \\
                  \mat{B} & 0
              \end{pmatrix}^{-1}
              =
              \begin{pmatrix}
                  0            & \mat{B}^{-1} \\
                  \mat{A}^{-1} & 0
              \end{pmatrix}
          \]
\end{enumerate}

\begin{example}
    设
    \[
        \mat{A} =
        \begin{pmatrix}
            1 & 1 & 1 \\
            1 & 2 & 1 \\
            1 & 1 & 3
        \end{pmatrix}
    \]
    求$\mat{A}^{-1}$
\end{example}
\begin{solution}
    \begin{align*}
         & \left(
        \begin{array}{ccc|ccc}
            1 & 1 & 1 & 1 & 0 & 0 \\
            1 & 2 & 1 & 0 & 1 & 0 \\
            1 & 1 & 3 & 0 & 0 & 1
        \end{array}
        \right)
        \rightarrow
        \left(
        \begin{array}{ccc|ccc}
            1 & 1 & 1 & 1  & 0 & 0 \\
            0 & 1 & 0 & -1 & 1 & 0 \\
            0 & 0 & 2 & -1 & 0 & 1
        \end{array}
        \right)
        \rightarrow
        \left(
        \begin{array}{ccc|ccc}
            1 & 0 & 1 & 2  & -1 & 0 \\
            0 & 1 & 0 & -1 & 1  & 0 \\
            0 & 0 & 2 & -1 & 0  & 1
        \end{array}
        \right)
        \rightarrow
        \left(
        \begin{array}{ccc|ccc}
            1 & 0 & 0 & \frac{5}{2} & -1 & -\frac{1}{2} \\
            0 & 1 & 0 & -1          & 1  & 0            \\
            0 & 0 & 2 & -1          & 0  & 1
        \end{array}
        \right)        \\
         & \rightarrow
        \left(
        \begin{array}{ccc|ccc}
            1 & 0 & 0 & \frac{5}{2}  & -1 & -\frac{1}{2} \\
            0 & 1 & 0 & -1           & 1  & 0            \\
            0 & 0 & 1 & -\frac{1}{2} & 0  & \frac{1}{2}
        \end{array}
        \right)
    \end{align*}
    所以
    \[
        \mat{A}^{-1} =
        \begin{pmatrix}
            \frac{5}{2}  & -1 & -\frac{1}{2} \\
            -1           & 1  & 0            \\
            -\frac{1}{2} & 0  & \frac{1}{2}
        \end{pmatrix}
        =\frac{1}{2}
        \begin{pmatrix}
            5  & -2 & -1 \\
            -2 & 2  & 0  \\
            -1 & 0  & 1
        \end{pmatrix}
    \]
\end{solution}

\begin{example}
    $\mat{A}$为$n$阶矩阵,且满足$\mat{A}^2 - 3\mat{A} - 2\mat{E} =0$
    求$(\mat{A}+\mat{E})^{-1}$
\end{example}
\begin{solution}
    对于这类题目,首先写下$\mat{A}+\mat{E}$,然后根据已知条件进行凑因式的形式来求解。
    \[ \mat{A}^2 - 3\mat{A} - 2\mat{E} = (\mat{A}+\mat{E})(\mat{A} - 4\mat{E}) + 2\mat{E} = 0 \]
    所以
    \[ (\mat{A}+\mat{E})^{-1} = \frac{1}{2}(4\mat{E}-\mat{A}) \]
\end{solution}
\begin{example}
    $\mat{A},\mat{B}$均为$3$阶矩阵,$\mat{AB}=2\mat{A}+\mat{B}$,其中
    \[
        \mat{B} =
        \begin{pmatrix}
            2 & 0 & 2 \\
            0 & 4 & 0 \\
            2 & 0 & 2
        \end{pmatrix}
    \]
    求$(\mat{A}-\mat{E})^{-1}$
\end{example}
\begin{solution}
    还是从已知条件上往$\mat{A}-\mat{E}$凑
    \[ \mat{AB} - 2\mat{A} - \mat{B} =(\mat{A}-\mat{E})\mat{B} - 2\mat{A} = (\mat{A}-\mat{E})\mat{B} - 2(\mat{A}-\mat{E}) - 2\mat{E} = (\mat{A}-\mat{E})(\mat{B}-2\mat{E}) - 2\mat{E} = 0  \]
    所以
    \[
        (\mat{A}-\mat{E})^{-1} = \frac{1}{2}(\mat{B}-2\mat{E}) =
        \begin{pmatrix}
            0 & 0 & 1 \\
            0 & 1 & 0 \\
            1 & 0 & 0
        \end{pmatrix}
    \]
\end{solution}
\begin{example}
    设$\mat{A}$为$n$阶矩阵,且存在自然数$k$,使得$\mat{A}^k=0$,
    求$\mat{E}-\mat{A}$的逆矩阵。
\end{example}
\begin{solution}
    联想到$a^n-b^n=(a-b)(a^{n-1} + a^{n-2}b + \cdots + ab^{n-2} + b^{n-1})$
    而$\mat{E}$又类似于代数中的$1$,所以有
    \[ -\mat{E} = \mat{A}^k - \mat{E} = \mat{A}^k - \mat{E}^k = (\mat{A}-\mat{E})(\mat{A}^{k-1} + \mat{A}^{k-2} + \cdots + \mat{E}) \]
    移项可得
    \[ (\mat{E}-\mat{A})(\mat{A}^{k-1} + \mat{A}^{k-2} + \cdots + \mat{E}) = \mat{E} \]
    所以$(\mat{E}-\mat{A})^{-1}= \mat{A}^{k-1} + \mat{A}^{k-2} + \cdots + \mat{E}$
\end{solution}

\section{初等变换、初等矩阵}
初等变换有初等行变换、初等列变换。初等行变换有:
\begin{enumerate}[(1)]
    \item 用非$0$常数$k$乘以矩阵的某一行的每个元素;
    \item 互换矩阵中两行元素的位置;
    \item 把$\mat{A}$中某行所有元素的$k$倍加到另一行对应的元素上。
\end{enumerate}
\begin{definition}
    单位矩阵$E$经过\textcolor{red}{一次}初等变换所得矩阵称为初等矩阵。
\end{definition}
\begin{theorem}
    对$m\times n$矩阵$\mat{A}$作一次初等\textcolor{red}{行变换}相当于用\textcolor{red}{相应}的$m$阶初等矩阵\textcolor{red}{左乘}$\mat{A}$;
    对$\mat{A}$作一次初等\textcolor{red}{列变换}相当于用\textcolor{red}{相应}的$n$阶初等矩阵\textcolor{red}{右乘}$\mat{A}$。
\end{theorem}
此处的“\textcolor{red}{相应}”是指同一类的初等矩阵,例如,交换$i,j$两行,所需的初等矩阵是由单位矩阵$i,j$行交换形成。
\begin{theorem}
    初等矩阵都是可逆的,并且每一类初等矩阵的逆矩阵仍为同一类型的初等矩阵。
\end{theorem}
对于初等矩阵的逆矩阵由有如下关系
\begin{enumerate}[(1)]
    \item 交换两行或两列$\mat{P}^{-1}(i,j) = \mat{P}(i,j)$;
    \item 某行或某列乘以非零常数$\mat{P}^{-1}(i(c)) = \mat{P}(i(c^{-1}))$;
    \item 某行(列)加上另一行(列)的$k$倍$\mat{P}^{-1}(i,j(k)) = \mat{P}(i,j(-k))$。
\end{enumerate}

假设$\mat{A}$可逆,则可以通过初等变换来求出逆矩阵,即
\[ (\mat{A}|\mat{E})\xrightarrow{\text{初等行变换}}(\mat{E}|\mat{A}^{-1}) \]
用矩阵表示就是
\begin{align*}
    \mat{P}_s\cdots \mat{P}_1\mat{A} = (\mat{P}_s\cdots \mat{P}_1)\mat{A}  = \mat{E}
\end{align*}
那么一系列的初等行变换矩阵$\mat{P}_s\cdots \mat{P}_1$即为$\mat{A}$的逆矩阵,也就相当于对单位矩阵$\mat{E}$作同样的初等行变换
\[ \mat{P}_s\cdots \mat{P}_1 = \mat{P}_s\cdots \mat{P}_1\mat{E} = \mat{A}^{-1} \]

\begin{definition}
    如果$m\times n$阶矩阵$\mat{A}$满足如下两个条件,则称$\mat{A}$为行阶梯矩阵。
    \begin{enumerate}[(1)]
        \item 矩阵中如果有零行,且都在矩阵的底部;
        \item 每个非零行的主元(该行最左边第一个非零元素)所在列的下面元素都是$0$。
    \end{enumerate}
\end{definition}
\begin{definition}
    如果$m\times n$阶矩阵$\mat{A}$是行阶梯矩阵,且非零行的主元都是$1$,且主元所在列的其它元素都是$0$,
    则称$\mat{A}$为行最简矩阵。
\end{definition}
例如下面这个矩阵就是行最简矩阵。
\[
    \begin{pmatrix}
        1 & 0 & 3  & 0 \\
        0 & 1 & -2 & 0 \\
        0 & 0 & 0  & 1
    \end{pmatrix}
\]

\begin{definition}
    矩阵$\mat{A}$有限此初等变换得到矩阵$\mat{B}$,就称矩阵$\mat{A}$和$\mat{B}$等价。
\end{definition}
\begin{theorem}
    $\mat{A},\mat{B}$都是$m\times n$矩阵,则有
    \[ \mat{A},\mat{B}\text{等价} \iff r(\mat{A}) = r(\mat{B}) \]
\end{theorem}

\begin{theorem}
    $\forall \mat{A}_{m\times n}$都存在$m$阶可逆矩阵$\mat{P}$和$n$阶可逆矩阵$\mat{Q}$使得
    \[
        \mat{PAQ} =
        \begin{pmatrix}
            \mat{E}_r & 0 \\
            0         & 0
        \end{pmatrix}
    \]
    称$\mat{PAQ}$为$\mat{A}$的等价标准型。其中$r$为$\mat{A}$的秩。
\end{theorem}
需要注意的是$\mat{P},\mat{Q}$不唯一。也就是说$\mat{P},\mat{Q}$对应的一系列初等变换不唯一。


在矩阵的乘法的题目中,如果某个矩阵的$0$比较多,就可以利用初等变化的思路来求解,从而节约时间。
\begin{example}
    计算
    \[
        \begin{pmatrix}
            1 & 0 & 2  \\
            0 & 0 & -1 \\
            0 & 1 & 0
        \end{pmatrix}
        \begin{pmatrix}
            1 & 2 & 3 \\
            4 & 5 & 6 \\
            7 & 8 & 9
        \end{pmatrix}
    \]
\end{example}
\begin{solution}
    左侧的矩阵$0$比较多,且易看出其为满秩矩阵,可以将其看成行变换。
    故将右侧矩阵写成行分块矩阵。
    \[
        \begin{pmatrix}
            1 & 0 & 2  \\
            0 & 0 & -1 \\
            0 & 1 & 0
        \end{pmatrix}
        \begin{pmatrix}
            \bm{\alpha}_1 \\
            \bm{\alpha}_2 \\
            \bm{\alpha}_3
        \end{pmatrix}
        =
        \begin{pmatrix}
            \bm{\alpha}_1 + 2\bm{\alpha}_3 \\
            -\bm{\alpha}_3                 \\
            \bm{\alpha}_2
        \end{pmatrix}
    \]
    此时可以看出,结果的第一行为原矩阵第一行加上两倍的第三行,第二行为第三行的相反数,第三行变为了第二行。
    所以结果为
    \[
        \begin{pmatrix}
            15 & 18 & 21 \\
            -7 & -8 & -9 \\
            4  & 5  & 6
        \end{pmatrix}
    \]
\end{solution}

\section{分块矩阵}
对矩阵的分块一般有三种类型
\begin{enumerate}[(1)]
    \item 十字分成四块,且对角线为上的矩阵为方阵(常用于$\mat{AB},\mat{A}^n,\mat{A}^{-1}$的题目中);
    \item 按列分块(一般与向量、秩和方程组相关);
    \item 按行分块(与上面相同)。
\end{enumerate}

对于分块矩阵有下面这些运算
\begin{equation}
    \begin{pmatrix}
        \mat{A}_1 & \mat{A}_2 \\
        \mat{A}_3 & \mat{A}_4
    \end{pmatrix}
    +
    \begin{pmatrix}
        \mat{B}_1 & \mat{B}_2 \\
        \mat{B}_3 & \mat{B}_4
    \end{pmatrix}
    =
    \begin{pmatrix}
        \mat{A}_1+\mat{B}_1 & \mat{A}_2+\mat{B}_2 \\
        \mat{A}_3+\mat{B}_3 & \mat{A}_4+\mat{B}_4
    \end{pmatrix}
\end{equation}
\begin{equation}
    \begin{pmatrix}
        \mat{A} & \mat{B} \\
        \mat{C} & \mat{D}
    \end{pmatrix}
    \begin{pmatrix}
        \mat{X} & \mat{Y} \\
        \mat{Z} & \mat{W}
    \end{pmatrix}
    =
    \begin{pmatrix}
        \mat{AX}+\mat{BZ} & \mat{AY}+\mat{BW} \\
        \mat{CX}+\mat{DZ} & \mat{CY}+\mat{BW}
    \end{pmatrix}
\end{equation}

\begin{equation}
    \begin{pmatrix}
        \mat{A} & \mat{B} \\
        \mat{C} & \mat{D}
    \end{pmatrix}^\intercal
    =
    \begin{pmatrix}
        \mat{A}^\intercal & \mat{C}^\intercal \\
        \mat{B}^\intercal & \mat{D}^\intercal
    \end{pmatrix}
\end{equation}

\begin{equation}
    \begin{pmatrix}
        \mat{A} & \mat{0} \\
        \mat{0} & \mat{B}
    \end{pmatrix}^n
    =
    \begin{pmatrix}
        \mat{A}^n & \mat{0}   \\
        \mat{0}   & \mat{B}^n
    \end{pmatrix}
\end{equation}

\begin{equation}
    \begin{pmatrix}
        \mat{A} & \mat{0} \\
        \mat{0} & \mat{B}
    \end{pmatrix}^{-1}
    =
    \begin{pmatrix}
        \mat{A}^{-1} & \mat{0}      \\
        \mat{0}      & \mat{B}^{-1}
    \end{pmatrix}
    \qquad
    \begin{pmatrix}
        \mat{0} & \mat{A} \\
        \mat{B} & \mat{0}
    \end{pmatrix}^{-1}
    =
    \begin{pmatrix}
        \mat{0}      & \mat{B}^{-1} \\
        \mat{A}^{-1} & \mat{0}
    \end{pmatrix}
\end{equation}

\begin{equation}
    \begin{pmatrix}
        \mat{A}_1 &        &           \\
                  & \ddots &           \\
                  &        & \mat{A}_n
    \end{pmatrix}
    \begin{pmatrix}
        \mat{B}_1 &        &           \\
                  & \ddots &           \\
                  &        & \mat{B}_n
    \end{pmatrix}
    =
    \begin{pmatrix}
        \mat{A}_1\mat{B}_1 &        &                    \\
                           & \ddots &                    \\
                           &        & \mat{A}_n\mat{B}_n
    \end{pmatrix}
\end{equation}
\begin{equation}
    \begin{pmatrix}
        \mat{A}_1 &        &           \\
                  & \ddots &           \\
                  &        & \mat{A}_n
    \end{pmatrix}^{-1}
    =
    \begin{pmatrix}
        \mat{A}_1^{-1} &        &                \\
                       & \ddots &                \\
                       &        & \mat{A}_n^{-1}
    \end{pmatrix}
\end{equation}

\subsection{分块矩阵与线性相关、方程组的关联}
记$\mat{A},\mat{B},\mat{C}$均为$n$阶矩阵,且$\mat{AB}=\mat{C}$。那么下面讨论按行、按列分块后的表述。
\paragraph{按列分块} 若$\mat{A},\mat{C}$按列分块,即
\[
    \begin{pmatrix}
        \bm{\gamma}_1 & \cdots & \bm{\gamma}_n
    \end{pmatrix}
    \begin{pmatrix}
        b_{11} & \cdots & b_{1n} \\
        \vdots & \ddots & \vdots \\
        b_{n1} & \cdots & b_{nn}
    \end{pmatrix}
    =
    \begin{pmatrix}
        \bm{\delta}_1 & \cdots & \bm{\delta}_n
    \end{pmatrix}
\]
那么写成方程组就有
\[
    \begin{cases}
        b_{11}\bm{\gamma}_1 + b_{21}\bm{\gamma}_2 + \cdots + b_{n1}\bm{\gamma}_n = \bm{\delta}_1 \\
        \cdots                                                                                   \\
        b_{1n}\bm{\gamma}_1 + b_{2n}\bm{\gamma}_2 + \cdots + b_{nn}\bm{\gamma}_n = \bm{\delta}_n
    \end{cases}
\]
换句话说,列向量$\bm{\delta}_i$可以被列向量组$(\bm{\gamma}_1,\bm{\gamma}_2,\cdots,\bm{\gamma}_n)$线性表出。

\paragraph{按行分块}如果将$\mat{B},\mat{C}$按行分块,则有
\[
    \begin{pmatrix}
        a_{11} & \cdots & a_{1n} \\
        \vdots & \ddots & \vdots \\
        a_{n1} & \cdots & a_{nn}
    \end{pmatrix}
    \begin{pmatrix}
        \bm{\alpha}_1 \\ \vdots \\ \bm{\alpha}_n
    \end{pmatrix}
    =
    \begin{pmatrix}
        \bm{\beta}_1 \\ \vdots \\ \bm{\beta}_n
    \end{pmatrix}
\]
写成方程组的形式
\[
    \begin{cases}
        a_{11}\bm{\alpha}_1 + a_{12}\bm{\alpha}_2 + \cdots + a_{1n}\bm{\alpha}_n = \bm{\beta}_1 \\
        \cdots                                                                                  \\
        a_{n1}\bm{\alpha}_1 + a_{n2}\bm{\alpha}_2 + \cdots + a_{nn}\bm{\alpha}_n = \bm{\beta}_n
    \end{cases}
\]
也就是行向量$\bm{\beta}_i$可以由行向量组$(\bm{\alpha}_1,\bm{\alpha}_2,\cdots,\bm{\alpha}_n)$线性表出。

\paragraph{方程组的联系} 如果将$\mat{A}$看作整体,$\mat{B},\mat{C}$按列分块,则有
\[
    \mat{A}
    \begin{pmatrix}
        \bm{\beta}_1 & \cdots & \bm{\beta}_n
    \end{pmatrix}
    =
    \begin{pmatrix}
        \mat{A}\bm{\beta}_1 & \cdots & \mat{A}\bm{\beta}_n
    \end{pmatrix}
    =
    \begin{pmatrix}
        \bm{\gamma}_1 & \cdots & \bm{\gamma}_n
    \end{pmatrix}
\]
所以有
\[ \mat{A}\bm{\beta}_1 = \bm{\gamma}_1, \cdots, \mat{A}\bm{\beta}_n = \bm{\gamma}_n \]
那么$\bm{\beta}_i$是方程组$\mat{AX}=\bm{\gamma}_i$的解。

\begin{example}
    设
    \[
        \mat{A} =
        \begin{pmatrix}
            1 & 2  & -2 \\
            4 & t  & 3  \\
            3 & -1 & 1
        \end{pmatrix}
    \]
    $\mat{B}$为$3$阶非零矩阵,且$\mat{AB}=\mat{0}$,求$t$的值。
\end{example}
\begin{solution}
    将$\mat{B}$按列分块,再与方程组联系,则有
    \[ A\bm{\beta}_1 = \bm{0},\qquad A\bm{\beta}_2 = \bm{0},\qquad A\bm{\beta}_3 = \bm{0} \]
    所以$\bm{\beta}_1,\bm{\beta}_2,\bm{\beta}_3$是齐次方程组$\mat{AX}=\mat{0}$的解。
    又$\mat{B}$为非零矩阵,故$\bm{\beta}_1,\bm{\beta}_2,\bm{\beta}_3$中至少有一个非零解。
    那么系数矩阵不满秩,故而
    \[
        0 = |\mat{A}| =
        \begin{vmatrix}
            1 & 2  & -2 \\
            4 & t  & 3  \\
            3 & -1 & 1
        \end{vmatrix}
        =
        \begin{vmatrix}
            7 & 0  & 0 \\
            4 & t  & 3 \\
            3 & -1 & 1
        \end{vmatrix}
        =
        7\begin{vmatrix}
            t  & 3 \\
            -1 & 1
        \end{vmatrix}
        =
        7(t+3)
    \]
    所以$t=-3$
\end{solution}

\section{正交矩阵}
\label{sec:正交矩阵}
.
\begin{definition}
    $n$阶矩阵$\mat{A}$,如果满足$A\mat{A}^\intercal = \mat{A}^\intercal \mat{A} = \mat{E}$
    则称$\mat{A}$是正交矩阵。
\end{definition}
根据正交矩阵的定义,可以得到
\begin{theorem}
    $\mat{A}$是正交矩阵$\iff \mat{A}^\intercal = \mat{A}^{-1} \iff A$的列向量都是\textcolor{red}{单位向量},且两两正交。
\end{theorem}
\begin{theorem}
    $\mat{A}$是正交矩阵$\implies |\mat{A}|$为$1$或$-1$
\end{theorem}