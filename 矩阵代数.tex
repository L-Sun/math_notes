\part{矩阵代数}
由于矩阵代数中的每个知识点都有相互联系,故此部分无法顺序地学习,应该往复看多几遍,同时学习过程中应该相互联系。
\section{矩阵的概念和运算}
形如
\[
    \begin{pmatrix}
        a_{11} & a_{12} & \cdots & a_{1n} \\
        a_{21} & a_{22} & \cdots & a_{2n} \\
        \vdots & \vdots & \ddots & \vdots \\
        a_{m1} & a_{m2} & \cdots & a_{mn}
    \end{pmatrix}
\]
成为$m$行$n$列的矩阵,记为$A_{m\times n}$,简记为$A$,当$m=n$时,成为\textbf{\textsf{方阵}}。
当矩阵内的所用元素均为$0$,则矩阵称为零矩阵,简记为$0$

如果$A$和$B$都是$m\times n$的矩阵,则称$A,B$为\textbf{\textsf{同型矩阵}}。进一步的,当$a_{ij}=b_{ij}$,
则有$A=B$

矩阵有如下的运算:
\begin{enumerate}[(1)]
    \item 矩阵加法$A+B=(a_{ij}+b_{ij})$;
    \item 矩阵数乘$kA = (ka_{ij})$;
\end{enumerate}

同时有加法的交换律和加法的结合律,数乘的交换律和结合律:
\begin{enumerate}[(1)]
    \item $A+B=B+A$;
    \item $(A+B)+C=A+(B+C)=A+B+C$;
    \item $k(mA) = m(kA) = (mk)A$;
    \item $(k+m)A = kA + mA$;
    \item $k(A+B) = kA + kB$;
    \item $1A=A, 0A = 0$
\end{enumerate}

\subsection{矩阵的乘法}
矩阵的乘法与数的乘法有所不同,首要就是其不满足交换律。
\begin{definition}
    设矩阵$A=(a_{ij})$是一个$m\times n$矩阵,$B=(b_{ij})$是一个$n\times p$矩阵,
    则定义运算
    \[ AB = C \]
    其中$C=(c_{ij})$为$m\times p$矩阵,且满足
    \[ c_{ij} = a_{i1}b_{1j} + a_{a2}b_{2j} + \cdots + a_{an}b_{nj} = a_i\vdot b_j \]
    其中$a_i$为$A$的第$i$行行向量,$b_j$为$B$的第$j$列列向量。
\end{definition}

对于矩阵的乘法有结合律和分配律
\begin{enumerate}[(1)]
    \item $(AB)C=A(BC)=ABC$;
    \item $A(B+C)=AB+AC$;
    \item $(A+B)C=AC+BC$;
    \item $AA=A^2$
\end{enumerate}

需要重视的是
\begin{enumerate}[(1)]
    \item \textcolor{red}{$AB \neq BA$},即矩阵乘法不满足交换律;
    \item \textcolor{red}{$AB \not\implies A=0 \text{或} B=0$};
    \item \textcolor{red}{$AB=AC,A\neq 0 \not\implies B = C$}.
\end{enumerate}

\begin{theorem}
    \label{th:行列式的乘法公式}
    (行列式的乘法公式)
    如果$A,B$为$n$阶矩阵,则有
    \[ |AB| = |A|\cdot|B| \]
\end{theorem}

若将$A$的行列互换,所得矩阵称为$A$的\textbf{\textsf{转置}},记为$A^\intercal$,其运算法则如下
\begin{enumerate}[(1)]
    \item $(A+B)^\intercal = A^\intercal + B^\intercal$;
    \item $(kA)^\intercal = kA^\intercal$;
    \item $(AB)^\intercal = B^\intercal A^\intercal$;
    \item $(A^\intercal)^\intercal=A$
\end{enumerate}
若$A^\intercal = A$,则称$A$为\textbf{\textsf{对称矩阵}},若$A^\intercal = - A$,则称$A$为\textbf{\textsf{反对称矩阵}}。
其中反对称矩阵的对角线一定是$0$。


设$\alpha,\beta$为$n$维列向量,可以将其看作$n\times 1$的矩阵,则有
\begin{align}
    \alpha\beta^\intercal  & =
    \begin{pmatrix}
        a_1b_1 & \cdots & a_1b_n \\
        \vdots & \ddots & \vdots \\
        a_nb_1 & \cdots & a_nb_1
    \end{pmatrix}                                  \\
    \alpha^\intercal\beta  & = a_1b_1+a_2b_2 + \cdots + a_nb_n \\
    \alpha\alpha^\intercal & =
    \begin{pmatrix}
        a_1a_1 & \cdots & a_1a_n \\
        \vdots & \ddots & \vdots \\
        a_na_1 & \cdots & a_na_1
    \end{pmatrix}\qquad\text{(对称矩阵)}          \\
    \alpha^\intercal\alpha & = a_1^2 + a_2^2 + \cdots a_n^2
\end{align}

\begin{example}
    已知列向量$\alpha = (1,2,3),\beta=(1,\frac{1}{2},\frac{1}{3}),A=\alpha^\intercal\beta$,
    求$A^n$
\end{example}
\begin{solution}
    \[
        A^n = (\alpha^\intercal\beta)(\alpha^\intercal\beta)\cdots(\alpha^\intercal\beta)
        =\alpha^\intercal(\beta\alpha^\intercal)\cdots(\beta\alpha^\intercal)\beta
        =\alpha^\intercal(3)^{n-1}\beta
        =3^{n-1}
        \begin{pmatrix}
            1 & \frac{1}{2} & \frac{1}{3} \\
            2 & 1           & \frac{2}{3} \\
            3 & \frac{3}{2} & 1
        \end{pmatrix}
    \]
\end{solution}

\begin{example}
    设$A=E-\xi\xi^\intercal$,其中$\xi$是$n$维非$0$列向量,证明
    $A^2=A\iff \xi\xi^\intercal = 1$
\end{example}
\begin{proof}
    \begin{align*}
        A^2-A & = A(A-E)                                                 \\
              & = (E-\xi\xi^\intercal)(-\xi\xi^\intercal)                \\
              & = -\xi\xi^\intercal + \xi\xi^\intercal\xi\xi^\intercal   \\
              & = -\xi\xi^\intercal + (\xi^\intercal\xi)\xi\xi^\intercal \\
              & = (\xi^\intercal\xi-1)\xi\xi^\intercal                   \\
    \end{align*}
    由于$\xi$是$n$维非$0$列向量,故$\xi\xi^\intercal\neq 0$,所以有
    \[\xi^\intercal\xi = 1 \iff A^2 = A \]
\end{proof}

形如
\[
    \begin{pmatrix}
        1 &        &   \\
          & \ddots &   \\
          &        & 1
    \end{pmatrix}
\]
称为\textbf{\textsf{单位矩阵}},简记为$E$,其在矩阵的乘法中起到$1$的作用,即
\[ AE = EA = A \]
其中$A,E$均为$n$阶方阵。

形如
\[
    D=
    \begin{pmatrix}
        a_1 &        &     \\
            & \ddots &     \\
            &        & a_n
    \end{pmatrix}
\]
称为\textbf{\textsf{对角矩阵}},对角矩阵的乘法可满足交换律,即
\[
    D_1D_2 = D_2D_1
    =
    \begin{pmatrix}
        a_1b_1 &        &        \\
               & \ddots &        \\
               &        & a_nb_n
    \end{pmatrix}
\]
同时有
\begin{equation}
    \begin{pmatrix}
        a_1 &        &     \\
            & \ddots &     \\
            &        & a_m
    \end{pmatrix}^n
    =
    \begin{pmatrix}
        a_1^n &        &       \\
              & \ddots &       \\
              &        & a_m^n
    \end{pmatrix}
\end{equation}
此通常用于求复杂矩阵的$n$次方,常出现于特征值相关的题目中。

对于对角矩阵的逆,可以通过下面的性质得出:
\begin{equation}
    \begin{pmatrix}
        a_1 &        &     \\
            & \ddots &     \\
            &        & a_n
    \end{pmatrix}
    \begin{pmatrix}
        \frac{1}{a_1} &        &               \\
                      & \ddots &               \\
                      &        & \frac{1}{a_n}
    \end{pmatrix}
    = E
\end{equation}

\subsection{矩阵的秩}
在引入矩阵的秩的定义前,先说明矩阵的子式概念。
\begin{definition}
    对于矩阵$A_{m\times n}$,任取$k$行$k$列,所得交点元素在相对位置不变的情况下组成的行列式,称为$A_{m \times n}$的$k$阶子式。
\end{definition}

\begin{definition}
    对于矩阵$A_{m\times n}$,如果\textcolor{red}{存在}$r$阶子式$D\neq 0$,且所有$r+1$阶子式(如果存在)全为$0$,则称矩阵$A$的秩为$r$,
    记作$r(A)=r$,并且规定零矩阵的秩为$0$
\end{definition}
可以通过这个定理可以缩小求秩的范围。
\begin{example}
    设
    \[
        A =
        \begin{pmatrix}
            1 & 2 & 3 \\
            2 & 4 & t \\
            3 & 5 & 8
        \end{pmatrix}
    \]
    求$r(A)$
\end{example}
\begin{solution}
    由于矩阵左下角的$2$阶子式
    \[
        \begin{vmatrix}
            2 & 4 \\
            3 & 5
        \end{vmatrix}
        \neq 0
    \]
    所以$2\leq r(A) \leq 3$。注意到$t=6$时,上下两行成比例,此时$|A|=0,r(A)=2$,
    那么当$t\neq 6$时,$r(A)=3$
\end{solution}

\begin{theorem}
    经过初等变换的矩阵,其秩不变。
\end{theorem}
由这个定理可以将矩阵进行初等变换,得到阶梯矩阵或最简矩阵,主元的个数就是矩阵的秩。秩相关的公式在附录\ref{sec:矩阵的秩相关公式}给出
\begin{theorem}
    矩阵的秩、矩阵列向量组的秩、矩阵行向量组的秩,三秩相等。
\end{theorem}
\begin{theorem}
    $n$阶矩阵$A$的行列式$|A|\neq 0 \iff r(A) = n$,反过来$|A|=0\iff r(A)< n$
\end{theorem}
\begin{example}
    \[
        A =
        \begin{pmatrix}
            k & 1 & 1 & 1 \\
            1 & k & 1 & 1 \\
            1 & 1 & k & 1 \\
            1 & 1 & 1 & k
        \end{pmatrix}
    \]
    若$r(A)=3$,求$k$
\end{example}
\begin{solution}
    由于$r(A)=3<4$,所以$|A|=0$,则
    \[
        |A| =
        \begin{vmatrix}
            k & 1 & 1 & 1 \\
            1 & k & 1 & 1 \\
            1 & 1 & k & 1 \\
            1 & 1 & 1 & k
        \end{vmatrix}
        =(k+3)
        \begin{vmatrix}
            1 & 1 & 1 & 1 \\
            1 & k & 1 & 1 \\
            1 & 1 & k & 1 \\
            1 & 1 & 1 & k
        \end{vmatrix}
        =(k+3)
        \begin{vmatrix}
            1 & 1   & 1   & 1   \\
            0 & k-1 & 0   & 0   \\
            0 & 0   & k-1 & 0   \\
            0 & 0   & 0   & k-1
        \end{vmatrix}
        =(k+3)(k-1)^3
    \]
    所以$k=-3$或$k=1$,当$k=1$时$r(A)=1$。当$k=-3$时
    \[
        \begin{vmatrix}
            -3 & 1  & 1  \\
            1  & -3 & 1  \\
            1  & 1  & -3 \\
        \end{vmatrix}
        =
        \begin{vmatrix}
            -3 & 1  & 1  \\
            4  & -4 & 0  \\
            4  & 0  & -4 \\
        \end{vmatrix}
        =
        \begin{vmatrix}
            -1 & 1  & 1  \\
            0  & -4 & 0  \\
            0  & 0  & -4 \\
        \end{vmatrix}
        =-16\neq 0
    \]
    所以这个三阶子式不等于$0$,因此$k=-3$
\end{solution}
利用初等变换,有如下解法
\begin{solution}
    \[
        A =
        \begin{pmatrix}
            k & 1 & 1 & 1 \\
            1 & k & 1 & 1 \\
            1 & 1 & k & 1 \\
            1 & 1 & 1 & k
        \end{pmatrix}
        \longrightarrow
        \begin{pmatrix}
            k   & 1   & 1   & 1   \\
            1-k & k-1 & 0   & 0   \\
            1-k & 0   & k-1 & 0   \\
            1-k & 0   & 0   & k-1
        \end{pmatrix}
        \longrightarrow
        \begin{pmatrix}
            k+3 & 1   & 1   & 1   \\
            0   & k-1 & 0   & 0   \\
            0   & 0   & k-1 & 0   \\
            0   & 0   & 0   & k-1
        \end{pmatrix}
    \]
    当$k=1$时,$r(A)=1$,当$k=-3$时,列主元有三个,故$r(A)=3$,所以$k=-3$
\end{solution}

\section{可逆矩阵}
在研究逆矩阵前,需要引入伴随矩阵的概念。
\begin{definition}
    设$A$为$n$阶矩阵,则$A$的伴随矩阵为
    \[
        A^* =
        \begin{pmatrix}
            A_{\textcolor{red}{1}1} & A_{\textcolor{red}{2}1} & \cdots & A_{\textcolor{red}{n}1} \\
            A_{\textcolor{red}{1}2} & A_{\textcolor{red}{2}2} & \cdots & A_{\textcolor{red}{n}2} \\
            \vdots                  & \vdots                  & \ddots & \vdots                  \\
            A_{\textcolor{red}{1}n} & A_{\textcolor{red}{2}n} & \cdots & A_{\textcolor{red}{n}n} \\
        \end{pmatrix}
    \]
    其中$A_{ij}$为$A$中$a_{ij}$的代数余子式。此处要注意\textcolor{red}{下标}。
\end{definition}
对于伴随矩阵有下面的等式
\begin{equation}
    AA^* = A^*A = |A|E
\end{equation}
其证明如下
\begin{proof}
    下面只讨论$n=2$的情况,对于$n$为任意正整数的情况同理。
    利用行列式的展开公式,有
    \begin{align*}
        AA^* & =
        \begin{pmatrix}
            a_{11} & a_{12} \\
            a_{21} & a_{22}
        \end{pmatrix}
        \begin{pmatrix}
            A_{11} & A_{21} \\
            A_{12} & A_{22}
        \end{pmatrix} \\
             & =
        \begin{pmatrix}
            a_{11}A_{11} + a_{12}A_{12} &  & a_{11}A_{21} + a_{12}A_{22} \\
            a_{21}A_{11} + a_{22}A_{12} &  & a_{21}A_{21} + a_{22}A_{22}
        \end{pmatrix} \\
             & =
        \begin{pmatrix}
            |A| & 0   \\
            0   & |A|
        \end{pmatrix} \\
             & = |A|E
    \end{align*}
\end{proof}

对于二阶矩阵的伴随矩阵,可以将主对角线互换,副对角线变号,即
\begin{equation}
    \begin{pmatrix}
        a & b \\
        c & d
    \end{pmatrix}^*
    =
    \begin{pmatrix}
        d  & -b \\
        -c & a
    \end{pmatrix}
\end{equation}
由此可以很简单地通过定义法求出二阶矩阵的逆矩阵。

有关伴随矩阵的公式在附录\ref{sec:伴随矩阵的公式}中给出。

\begin{definition}
    对于$n$阶矩阵A,如果存在$n$阶矩阵$B$,使得
    \[ AB = BA = E \]
    则称矩阵$A$是可逆的,称$B$是$A$的逆矩阵,
    记$A$的逆矩阵为$A^{-1}$,且$A^{-1}$惟一。其中$E$为$n$阶单位矩阵。
\end{definition}
在有关逆矩阵的证明中,常需要用单位矩阵的恒等变形,例如$A=AE=A(BB^{-1})$。
\begin{example}
    若矩阵$A$可逆,证明$A$的逆矩阵唯一。
\end{example}
\begin{proof}
    假设$A$的逆矩阵不唯一,记$B,C$分别为$A$的逆矩阵。则有$AB=BA=E, AC=CA=E$。
    因此有
    \[ B = BE = BAC = EC = C \]
    显然与假设矛盾,故假设$A$的逆矩阵唯一。
\end{proof}

利用伴随矩阵可以计算逆矩阵:
\begin{theorem}
    若$n$阶矩阵$A$的行列式不为$0$,则$A$的逆矩阵为
    \[
        A^{-1} = \frac{1}{|A|}A^*
    \]
\end{theorem}

\begin{theorem}
    $n$阶矩阵$A$可逆$\iff |A|\neq 0$。
\end{theorem}
\begin{proof}
    (充分性)
    \[ AA^{-1} = E \implies |AA^{-1}| = |A|\cdot|A^{-1}| = |E| = 1 \]
    所以$|A|\neq 0$。
    (必要性)
    \[ AA^* = |A|E \]
    由于$|A|\neq 0$
    所以$A \frac{A^*}{|A|} = E$,所以$A$可逆。
\end{proof}

可逆矩阵的一些性质在附录\ref{sec:可逆矩阵的性质}给出。

\subsection{逆矩阵的求解}
如果矩阵$A$可逆,一般有如下几种方法求逆矩阵。
\begin{enumerate}[(1)]
    \item 定义法;
    \item 伴随矩阵法;
    \item 初等行变换
          \[ (A|E) \xrightarrow{\text{由上往下}} (\text{上三角矩阵}|*)\xrightarrow{\text{由下往上}} (\text{对角矩阵}|*)\xrightarrow{\text{乘系数}}(E|A^{-1}) \]
    \item 分块法
          \[
              \begin{pmatrix}
                  A & 0 \\
                  0 & B
              \end{pmatrix}^{-1}
              =
              \begin{pmatrix}
                  A^{-1} & 0      \\
                  0      & B^{-1}
              \end{pmatrix}
          \]
          \[
              \begin{pmatrix}
                  0 & A \\
                  B & 0
              \end{pmatrix}^{-1}
              =
              \begin{pmatrix}
                  0      & B^{-1} \\
                  A^{-1} & 0
              \end{pmatrix}
          \]
\end{enumerate}

\begin{example}
    设
    \[
        A =
        \begin{pmatrix}
            1 & 1 & 1 \\
            1 & 2 & 1 \\
            1 & 1 & 3
        \end{pmatrix}
    \]
    求$A^{-1}$
\end{example}
\begin{solution}
    \begin{align*}
         & \left(
        \begin{array}{ccc|ccc}
            1 & 1 & 1 & 1 & 0 & 0 \\
            1 & 2 & 1 & 0 & 1 & 0 \\
            1 & 1 & 3 & 0 & 0 & 1
        \end{array}
        \right)
        \rightarrow
        \left(
        \begin{array}{ccc|ccc}
            1 & 1 & 1 & 1  & 0 & 0 \\
            0 & 1 & 0 & -1 & 1 & 0 \\
            0 & 0 & 2 & -1 & 0 & 1
        \end{array}
        \right)
        \rightarrow
        \left(
        \begin{array}{ccc|ccc}
            1 & 0 & 1 & 2  & -1 & 0 \\
            0 & 1 & 0 & -1 & 1  & 0 \\
            0 & 0 & 2 & -1 & 0  & 1
        \end{array}
        \right)
        \rightarrow
        \left(
        \begin{array}{ccc|ccc}
            1 & 0 & 0 & \frac{5}{2} & -1 & -\frac{1}{2} \\
            0 & 1 & 0 & -1          & 1  & 0            \\
            0 & 0 & 2 & -1          & 0  & 1
        \end{array}
        \right)        \\
         & \rightarrow
        \left(
        \begin{array}{ccc|ccc}
            1 & 0 & 0 & \frac{5}{2}  & -1 & -\frac{1}{2} \\
            0 & 1 & 0 & -1           & 1  & 0            \\
            0 & 0 & 1 & -\frac{1}{2} & 0  & \frac{1}{2}
        \end{array}
        \right)
    \end{align*}
    所以
    \[
        A^{-1} =
        \begin{pmatrix}
            \frac{5}{2}  & -1 & -\frac{1}{2} \\
            -1           & 1  & 0            \\
            -\frac{1}{2} & 0  & \frac{1}{2}
        \end{pmatrix}
        =\frac{1}{2}
        \begin{pmatrix}
            5  & -2 & -1 \\
            -2 & 2  & 0  \\
            -1 & 0  & 1
        \end{pmatrix}
    \]
\end{solution}

\begin{example}
    $A$为$n$阶矩阵,且满足$A^2 - 3A - 2E =0$
    求$(A+E)^{-1}$
\end{example}
\begin{solution}
    对于这类题目,首先写下$(A+E)$,然后根据已知条件进行凑因式的形式来求解。
    \[ A^2 - 3A - 2E = (A+E)(A - 4E) + 2E = 0 \]
    所以
    \[ (A+E)^{-1} = \frac{1}{2}(4E-A) \]
\end{solution}
\begin{example}
    $A,B$均为$3$阶矩阵,$AB=2A+B$,其中
    \[
        B = \begin{pmatrix}
            2 & 0 & 2 \\
            0 & 4 & 0 \\
            2 & 0 & 2
        \end{pmatrix}
    \]
    求$(A-E)^{-1}$
\end{example}
\begin{solution}
    还是从已知条件上往$(A-E)$凑
    \[ AB - 2A - B =(A-E)B - 2A = (A-E)B - 2(A-E) - 2E = (A-E)(B-2E) - 2E = 0  \]
    所以
    \[
        (A-E)^{-1} = \frac{1}{2}(B-2E) =
        \begin{pmatrix}
            0 & 0 & 1 \\
            0 & 1 & 0 \\
            1 & 0 & 0
        \end{pmatrix}
    \]
\end{solution}
\begin{example}
    设$A$为$n$阶矩阵,且存在自然数$k$,使得$A^k=0$,
    求$E-A$的逆矩阵。
\end{example}
\begin{solution}
    联想到$a^n-b^n=(a-b)(a^{n-1} + a^{n-2}b + \cdots + ab^{n-2} + b^{n-1})$
    而$E$又类似于代数中的$1$,所以有
    \[ -E = A^k - E = A^k - E^k = (A-E)(A^{k-1} + A^{k-2} + \cdots + E) \]
    移项可得
    \[ (E-A)(A^{k-1} + A^{k-2} + \cdots + E) = E \]
    所以$(E-A)^{-1}= A^{k-1} + A^{k-2} + \cdots + E$
\end{solution}

\section{初等变换、初等矩阵}
初等变换有初等行变换、初等列变换。初等行变换有:
\begin{enumerate}[(1)]
    \item 用非$0$常数$k$乘以矩阵的某一行的每个元素;
    \item 互换矩阵中两行元素的位置;
    \item 把$A$中某行所有元素的$k$倍加到另一行对应的元素上。
\end{enumerate}
\begin{definition}
    单位矩阵$E$经过\textcolor{red}{一次}初等变换所得矩阵称为初等矩阵。
\end{definition}
\begin{theorem}
    对$m\times n$矩阵$A$作一次初等\textcolor{red}{行变换}相当于用\textcolor{red}{相应}的$m$阶初等矩阵\textcolor{red}{左乘}$A$;
    对$A$作一次初等\textcolor{red}{列变换}相当于用\textcolor{red}{相应}的$n$阶初等矩阵\textcolor{red}{右乘}$A$。
\end{theorem}
此处的“\textcolor{red}{相应}”是指同一类的初等矩阵,例如,交换$i,j$两行,所需的初等矩阵是由单位矩阵$i,j$行交换形成。
\begin{theorem}
    初等矩阵都是可逆的,并且每一类初等矩阵的逆矩阵仍为同一类型的初等矩阵。
\end{theorem}
对于初等矩阵的逆矩阵由有如下关系
\begin{enumerate}[(1)]
    \item 交换两行或两列$P^{-1}(i,j) = P(i,j)$;
    \item 某行或某列乘以非零常数$P^{-1}(i(c)) = P(i(c^{-1}))$;
    \item 某行(列)加上另一行(列)的$k$倍$P^{-1}(i,j(k)) = P(i,j(-k))$。
\end{enumerate}

假设$A$可逆,则可以通过初等变换来求出逆矩阵,即
\[ (A|E)\xrightarrow{\text{初等行变换}}(E|A^{-1}) \]
用矩阵表示就是
\begin{align*}
    P_s\cdots P_1A = (P_s\cdots P_1)A  = E
\end{align*}
那么一系列的初等行变换矩阵$P_s\cdots P_1$即为$A$的逆矩阵,也就相当于对单位矩阵$E$作同样的初等行变换
\[ P_s\cdots P_1 = P_s\cdots P_1E = A^{-1} \]

\begin{definition}
    如果$m\times n$阶矩阵$A$满足如下两个条件,则称$A$为行阶梯矩阵。
    \begin{enumerate}[(1)]
        \item 矩阵中如果有零行,且都在矩阵的底部;
        \item 每个非零行的主元(该行最左边第一个非零元素)所在列的下面元素都是$0$。
    \end{enumerate}
\end{definition}
\begin{definition}
    如果$m\times n$阶矩阵$A$是行阶梯矩阵,且非零行的主元都是$1$,且主元所在列的其它元素都是$0$,
    则称$A$为行最简矩阵。
\end{definition}
例如下面这个矩阵就是行最简矩阵。
\[
    \begin{pmatrix}
        1 & 0 & 3  & 0 \\
        0 & 1 & -2 & 0 \\
        0 & 0 & 0  & 1
    \end{pmatrix}
\]

\begin{definition}
    矩阵$A$有限此初等变换得到矩阵$B$,就称矩阵$A$和$B$等价。
\end{definition}
\begin{theorem}
    $A,B$都是$m\times n$矩阵,则有
    \[ A,B\text{等价} \iff r(A) = r(B) \]
\end{theorem}

\begin{theorem}
    $\forall A_{m\times n}$都存在$m$阶可逆矩阵$P$和$n$阶可逆矩阵$Q$使得
    \[
        PAQ =
        \begin{pmatrix}
            E_r & 0 \\
            0   & 0
        \end{pmatrix}
    \]
    称$PAQ$为$A$的等价标准型。其中$r$为$A$的秩。
\end{theorem}
需要注意的是$P,Q$不唯一。也就是说$P,Q$对应的一系列初等变换不唯一。


在矩阵的乘法的题目中,如果某个矩阵的$0$比较多,就可以利用初等变化的思路来求解,从而节约时间。
\begin{example}
    计算
    \[
        \begin{pmatrix}
            1 & 0 & 2  \\
            0 & 0 & -1 \\
            0 & 1 & 0
        \end{pmatrix}
        \begin{pmatrix}
            1 & 2 & 3 \\
            4 & 5 & 6 \\
            7 & 8 & 9
        \end{pmatrix}
    \]
\end{example}
\begin{solution}
    左侧的矩阵$0$比较多,且易看出其为满秩矩阵,可以将其看成行变换。
    故将右侧矩阵写成行分块矩阵。
    \[
        \begin{pmatrix}
            1 & 0 & 2  \\
            0 & 0 & -1 \\
            0 & 1 & 0
        \end{pmatrix}
        \begin{pmatrix}
            \alpha_1 \\
            \alpha_2 \\
            \alpha_3
        \end{pmatrix}
        =
        \begin{pmatrix}
            \alpha_1 + 2\alpha_3 \\
            -\alpha_3            \\
            \alpha_2
        \end{pmatrix}
    \]
    此时可以看出,结果的第一行为原矩阵第一行加上两倍的第三行,第二行为第三行的相反数,第三行变为了第二行。
    所以结果为
    \[
        \begin{pmatrix}
            15 & 18 & 21 \\
            -7 & -8 & -9 \\
            4  & 5  & 6
        \end{pmatrix}
    \]
\end{solution}

\section{分块矩阵}
对矩阵的分块一般有三种类型
\begin{enumerate}[(1)]
    \item 十字分成四块,且对角线为上的矩阵为方阵(常用于$AB,A^n,A^{-1}$的题目中);
    \item 按列分块(一般与向量、秩和方程组相关);
    \item 按行分块(与上面相同)。
\end{enumerate}

对于分块矩阵有下面这些运算
\begin{equation}
    \begin{pmatrix}
        A_1 & A_2 \\
        A_3 & A_4
    \end{pmatrix}
    +
    \begin{pmatrix}
        B_1 & B_2 \\
        B_3 & B_4
    \end{pmatrix}
    =
    \begin{pmatrix}
        A_1+B_1 & A_2+B_2 \\
        A_3+B_3 & A_4+B_4
    \end{pmatrix}
\end{equation}
\begin{equation}
    \begin{pmatrix}
        A & B \\
        C & D
    \end{pmatrix}
    \begin{pmatrix}
        X & Y \\
        Z & W
    \end{pmatrix}
    =
    \begin{pmatrix}
        AX+BZ & AY+BW \\
        CX+DZ & CY+BW
    \end{pmatrix}
\end{equation}

\begin{equation}
    \begin{pmatrix}
        A & B \\
        C & D
    \end{pmatrix}^\intercal
    =
    \begin{pmatrix}
        A^\intercal & C^\intercal \\
        B^\intercal & D^\intercal
    \end{pmatrix}
\end{equation}

\begin{equation}
    \begin{pmatrix}
        A & 0 \\
        0 & B
    \end{pmatrix}^n
    =
    \begin{pmatrix}
        A^n & 0   \\
        0   & B^n
    \end{pmatrix}
\end{equation}

\begin{equation}
    \begin{pmatrix}
        A & 0 \\
        0 & B
    \end{pmatrix}^{-1}
    =
    \begin{pmatrix}
        A^{-1} & 0      \\
        0      & B^{-1}
    \end{pmatrix}
    \qquad
    \begin{pmatrix}
        0 & A \\
        B & 0
    \end{pmatrix}^{-1}
    =
    \begin{pmatrix}
        0      & B^{-1} \\
        A^{-1} & 0
    \end{pmatrix}
\end{equation}

\begin{equation}
    \begin{pmatrix}
        A_1 &        &     \\
            & \ddots &     \\
            &        & A_n
    \end{pmatrix}
    \begin{pmatrix}
        B_1 &        &     \\
            & \ddots &     \\
            &        & B_n
    \end{pmatrix}
    =
    \begin{pmatrix}
        A_1B_1 &        &        \\
               & \ddots &        \\
               &        & A_nB_n
    \end{pmatrix}
\end{equation}
\begin{equation}
    \begin{pmatrix}
        A_1 &        &     \\
            & \ddots &     \\
            &        & A_n
    \end{pmatrix}^{-1}
    =
    \begin{pmatrix}
        A_1^{-1} &        &          \\
                 & \ddots &          \\
                 &        & A_n^{-1}
    \end{pmatrix}
\end{equation}

\subsection{分块矩阵与线性相关、方程组的关联}
记$A,B,C$均为$n$阶矩阵,且$AB=C$。那么下面讨论按行、按列分块后的表述。
\paragraph{按列分块} 若$A,C$按列分块,即
\[
    \begin{pmatrix}
        \gamma_1 & \cdots & \gamma_n
    \end{pmatrix}
    \begin{pmatrix}
        b_{11} & \cdots & b_{1n} \\
        \vdots & \ddots & \vdots \\
        b_{n1} & \cdots & b_{nn}
    \end{pmatrix}
    =
    \begin{pmatrix}
        \delta_1 & \cdots & \delta_n
    \end{pmatrix}
\]
那么写成方程组就有
\[
    \begin{cases}
        b_{11}\gamma_1 + b_{21}\gamma_2 + \cdots + b_{n1}\gamma_n = \delta_1 \\
        \cdots                                                               \\
        b_{1n}\gamma_1 + b_{2n}\gamma_2 + \cdots + b_{nn}\gamma_n = \delta_n
    \end{cases}
\]
换句话说,列向量$\delta_i$可以被列向量组$(\gamma_1,\gamma_2,\cdots,\gamma_n)$线性表出。

\paragraph{按行分块}如果将$B,C$按行分块,则有
\[
    \begin{pmatrix}
        a_{11} & \cdots & a_{1n} \\
        \vdots & \ddots & \vdots \\
        a_{n1} & \cdots & a_{nn}
    \end{pmatrix}
    \begin{pmatrix}
        \alpha_1 \\ \vdots \\ \alpha_n
    \end{pmatrix}
    =
    \begin{pmatrix}
        \beta_1 \\ \vdots \\ \beta_n
    \end{pmatrix}
\]
写成方程组的形式
\[
    \begin{cases}
        a_{11}\alpha_1 + a_{12}\alpha_2 + \cdots + a_{1n}\alpha_n = \beta_1 \\
        \cdots                                                              \\
        a_{n1}\alpha_1 + a_{n2}\alpha_2 + \cdots + a_{nn}\alpha_n = \beta_n
    \end{cases}
\]
也就是行向量$\beta_i$可以由行向量组$(\alpha_1,\alpha_2,\cdots,\alpha_n)$线性表出。

\paragraph{方程组的联系} 如果将$A$看作整体,$B,C$按列分块,则有
\[
    A
    \begin{pmatrix}
        \beta_1 & \cdots & \beta_n
    \end{pmatrix}
    =
    \begin{pmatrix}
        A\beta_1 & \cdots & A\beta_n
    \end{pmatrix}
    =
    \begin{pmatrix}
        \gamma_1 & \cdots & \gamma_n
    \end{pmatrix}
\]
所以有
\[ A\beta_1 = \gamma_1, \cdots, A\beta_n = \gamma_n \]
那么$\beta_i$是方程组$AX=\gamma_i$的解。

\begin{example}
    设
    \[
        A =
        \begin{pmatrix}
            1 & 2  & -2 \\
            4 & t  & 3  \\
            3 & -1 & 1
        \end{pmatrix}
    \]
    $B$为$3$阶非零矩阵,且$AB=0$,求$t$的值。
\end{example}
\begin{solution}
    将$B$按列分块,再与方程组联系,则有
    \[ A\beta_1 = 0,\qquad A\beta_2 = 0,\qquad A\beta_3 = 0 \]
    所以$\beta_1,\beta_2,\beta_3$是齐次方程组$AX=0$的解。
    又$B$为非零矩阵,故$\beta_1,\beta_2,\beta_3$中至少有一个非零解。
    那么系数矩阵不满秩,故而
    \[
        0 = |A| =
        \begin{vmatrix}
            1 & 2  & -2 \\
            4 & t  & 3  \\
            3 & -1 & 1
        \end{vmatrix}
        =
        \begin{vmatrix}
            7 & 0  & 0 \\
            4 & t  & 3 \\
            3 & -1 & 1
        \end{vmatrix}
        =
        7\begin{vmatrix}
            t  & 3 \\
            -1 & 1
        \end{vmatrix}
        =
        7(t+3)
    \]
    所以$t=-3$
\end{solution}

\section{正交矩阵}
\label{sec:正交矩阵}
.
\begin{definition}
    $n$阶矩阵$A$,如果满足$AA^\intercal = A^\intercal A = E$
    则称$A$是正交矩阵。
\end{definition}
根据正交矩阵的定义,可以得到
\begin{theorem}
    $A$是正交矩阵$\iff A^\intercal = A^{-1} \iff A$的列向量都是\textcolor{red}{单位向量},且两两正交。
\end{theorem}
\begin{theorem}
    $A$是正交矩阵$\implies |A|$为$1$或$-1$
\end{theorem}